\documentclass{homework}

\title{Homework 6}
\author{Kevin Evans}
\studentid{11571810}
\date{March 6, 2020}
\setclass{Physics}{304}
\usepackage{amssymb}
\usepackage{mathtools}

\usepackage{amsthm}
\usepackage{amsmath}
\usepackage{slashed}
\usepackage{relsize}
\usepackage{threeparttable}
\usepackage{float}
\usepackage{booktabs}
\usepackage{boldline}
\usepackage{changepage}
\usepackage{venndiagram}
\usepackage{icomma}
\usepackage{physics}
\usepackage[inter-unit-product =\cdot]{siunitx}
\usepackage{setspace}

\usepackage[makeroom]{cancel}
\usepackage{pgfplots}

\usepackage{multicol}
\usepackage{tcolorbox}
\usepackage{enumitem}
\usepackage{times}
\usepackage{mhchem}

\DeclareSIUnit{\year}{yr}
\begin{document}
	\maketitle
	\subsubsection*{Chapter 15}
	\begin{enumerate}
		\item[6.] Since the lepton numbers $L_\mu$ and $L_e$ must be conserved,
			\begin{align*}
				\mu^+ &\text{: } L_\mu = -1 \\
				\mathrm{e}^- &\text{: } L_\mathrm{e} = 1
			\end{align*}
			The neutrinos must be one muon antineutrino ($\bar{\nu}_\mu$, $L_\mu=-1$) and an electron neutrino ($\nu_e$, $L_e=1$).
		\item[8.] \begin{enumerate}
			\item $\bar{\nu}_\mu$ ($L_\mu = -1$)
			\item $\nu_\mu$ ($L_\mu = 1$)
			\item $\bar{\nu}_e$ ($L_e = -1$)
			\item $\nu_e$ ($L_e = 1$)
			\item $\nu_\mu$ ($L_\mu = 1$)
			\item $\nu_\mu$ ($L_\mu = 1$), $\bar{\nu}_e$ ($L_e = -1$)
		\end{enumerate}
		\item[9.] \begin{enumerate}
			\item Cannot occur; baryon number is not conserved. $B: +1 \to 0$.
			\item Can occur? Wouldn't this violate conservation of energy though?
			\item Cannot occur; baryon number is not conserved.
			\item Can occur.
			\item Can occur.
			\item Cannot occur; muon and baryon number are not conserved.
		\end{enumerate}
		\item[11.] \begin{enumerate}
			\item Muon and electron number are not conserved.
			\item Electron number is not conserved.
			\item Strangeness is not conserved.
			\item Electron and baryon number are not conserved.
			\item Strangeness and the baryon number are not conserved.
		\end{enumerate}
		\item[12.] \begin{enumerate}
			\item Muon and electron lepton numbers are not conserved.
			\item Charge is not conserved.
			\item Baryon number is not conserved.
			\item Baryon number is not conserved.
			\item Charge is not conserved.
		\end{enumerate}
		\item[17.] Using the hadron compositions, \begin{align*}
			\Sigma^0 + p & \to \Sigma^+ + \gamma + ? \\
			\mathrm{uds} + \mathrm{uud} & \to \mathrm{uus} + \gamma + \boxed{\mathrm{udd}}
		\end{align*}
		The particle is a neutron (udd).
		
		\item[18.] If we sum the quark constituents of each, they equal the values of the $\mathrm{K}^0$ and $\mathrm{\Lambda}^0$ particles.
			\begin{center}
				\def\arraystretch{1.5}
				\begin{tabular}{|c||cc|c||ccc|c|}
					\hline
					& $\mathrm{d}$ & $\mathrm{\bar{s}}$ & $\mathrm{K}^0$ & $\mathrm{u}$ & $\mathrm{d}$ & $\mathrm{s}$ & $\mathrm{\Lambda}^0$ \\
					\hline
					$Q$ & $-\frac{1}{3}$ & $\frac{1}{3}$ & $0$ & $\frac{2}{3}$ & $-\frac{1}{3}$ & $-\frac{1}{3}$ & $0$ \\
					\hline
					$B$ & $\frac{1}{3}$ & $-\frac{1}{3}$ & $0$ & $\frac{1}{3}$ & $\frac{1}{3}$ & $\frac{1}{3}$ & $1$ \\
					\hline
					$S$ & $0$ & $1$ & $1$ & $0$ & $0$ & $-1$ & $-1$ \\
					\hline
				\end{tabular}
			\end{center}
		\item[28.] As the momentum of the pion is zero, the total momentum must be conserved. Then $\abs{p_\mu} = \abs{p_{\bar{\nu}}}$. Using the energy-momentum relationship, \begin{align*}
			\left( E_\mu \right)^2 & = (pc)^2 + \left( m_\mu c^2 \right)^2 \\
			\left( E_\nu\right)^2 & = (pc)^2 & \text{Zero rest mass} \\
			\left( E_\mu \right)^2 - \left( E_\nu\right)^2 & = \left( m_\mu c^2 \right)^2 & \text{($*$)}
		\end{align*}
		From conservation of energy, \begin{align*}
			E_\pi & = E_\mu + E_{\bar{\nu}} \\
			E_\pi - E_{\bar{\nu}} & = E_\mu \\
			\left(E_\pi - E_{\bar{\nu}}\right)^2 & = \left(E_\mu\right)^2
			\intertext{Subtracting the square neutrino energy to achieve the form of ($*$), we can then expand the LHS polynomial and substitute in ($*$),}
			\left(E_\pi - E_{\bar{\nu}}\right)^2 - \left(E_{\bar{\nu}}\right)^2 & = \left(E_\mu\right)^2 - \left(E_{\bar{\nu}}\right)^2 \\
			\left( E_{\pi} \right)^2 - 2E_\pi E_{\bar{\nu}} + \cancel{\left(E_{\bar{\nu}}\right)^2} - \cancel{\left(E_{\bar{\nu}}\right)^2} & =  \left( m_\mu c^2 \right)^2 \\
			2 E_\pi E_{\bar{\nu}} & =  \left( E_\pi \right)^2 - \left( m_\mu c^2 \right)^2 \\
			E_{\bar{\nu}} & = \frac{\left( E_\pi \right)^2 - \left( m_\mu c^2 \right)^2}{2 E_\pi}
			\intertext{Using the energies given in the book (and since the pion energy is its rest mass here),}
			E_{\bar{\nu}} & = \frac{\left(\SI{139.5}{\MeV}\right)^2 - \left(\SI{105.7}{\MeV}\right)^2}
			{2 \left(\SI{139.5}{\MeV}\right)} \\
			& \approx \SI{29.71}{\MeV}
		\end{align*}
		\item[30.] \begin{enumerate}
			\item Electron-positron annihilation releasing $2\gamma$. The exchanged particle is $\mathrm{e}^\pm$ (ambiguous).
			\item Neutron decay, $\mathrm{n}\;(\mathrm{udd}) \to \mathrm{p}\;(\mathrm{udu})$. The charge is carried from $\nu_\mu\;(0) \to \mathrm{p}\;(+1)$. As it's a weak force with $+1$ charge, the particle is $\mathrm{W}^+$, and also since the $\nu_\mu \to \mu^-$ primitive uses a $\mathrm{W}^+$.
		\end{enumerate}
		\item[31.] \begin{enumerate}
			\item Since the particles are unchanged, the exchange particle is $\mathrm{Z}^0$.
			\item Gluon as the exchange is between quarks (strong process).
		\end{enumerate}
	\end{enumerate}
\end{document}
	