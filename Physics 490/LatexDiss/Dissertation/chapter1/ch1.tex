\addchapheadtotoc

\chapter{Introduction}
How does chaos seen in classical systems emerge from quantum mechanics, which is fundamentally linear? This paper aims at shedding light on the origins of chaos in quantum mechanics by examining Bose-Einstein condensates modeled by the Gross-Pitaevskii equation. We will begin by providing a shallow background on the fundamentals of quantum mechanics and a primer to chaos theory. Next, we will devise an experiment at characterizing chaos in Bose-Einstein condensates and see the differences in chaos between the Gross-Pitaevskii equation and the normal Schr\"odinger equation.

\begin{center}
	* * *
\end{center}

The Schr\"odinger equation is a linear partial differential equation that describes how quantum mechanical objects evolve in time and space. Any quantum mechanical object can be described by its corresponding wavefunction, governing its probability density.  This complex wavefunction exists in a mathematical space known as Hilbert space \cite{shankar2012principles}.

Bose-Einstein condensates (BECs) are formed by bosonic gases at low densities near absolute zero temperatures, resulting in the high occupation of the lowest quantum state. The collapse to the lowest quantum state is due to the indistinguishability and bosonic nature of these particles \cite{GouldTobochnik+2010}. This additional state of nature was predicted by Einstein and Bose in the early twentieth century \cite{einstein1924quantum}. This state of matter is purely a quantum phenomena and are a recent research interest since its experimental realization of superfluid Helium-4 by Allen and Misener in 1937 \cite{allen1938flow}, and later Rubidium-87 by Cornell, Wieman, and Keterle in 1995 \cite{ensher1996bose}. Although we can model BECs using the Schr\"odinger equation, problems lay when the particle count is increased.

Many-particle systems can be difficult to model as the exact wavefunction would require $N$-variables. This approach quickly becomes computationally infeasible for more than a handful of particles. Several approximations can be used to reduce the modeling complexity. An alternative to a many-body wavefunction is applying density-functional theory (DFT) to model ground-state many-body systems. From the Kohn-Hohenberg theory, it can be shown that there exists a functional of the density that exactly describes the ground state of a system \cite{HK:1964}. The Gross-Pitaevskii equation corresponds to a density functional that closely approximates BECs routinely produced experimentally. Although this is not a perfect model and describes a classical field, it describes many quantum phenomena in superfluids. The nonlinearity of the GPE may give rise to chaos and is the topic of this paper.

Chaos is the sensitivity to initial conditions in the motion of a dynamical system---more formally, the exponential divergence of close-by trajectories in time \cite{vulpiani2009chaos}. Chaos can be characterized through Lyapunov exponents, where the sign of the exponent denotes the chaos in a system. A positive maximal exponent characterizes chaos, as it implies an exponential diverging growth. A non-positive exponent is found when no chaos is present in a system. Chaos theory helps us understand why some dynamical systems are difficult to predict and has broad applications outside of physics, such as describing the difficulties in predicting weather \cite{DeterministicNonperiodicFlow}. In an intuitive sense, we can imagine the Lyapunov exponents as a rough time scale on the predictability of a system.

An common example of chaotic motion is seen in the the Lorenz attractor, resulting in positive Lyapunov exponents \cite{DeterministicNonperiodicFlow}. The Lorenz system is composed of a set of interdependent, nonlinear differential equations, initially used to model atmospheric convection by Edward Lorenz \cite{DeterministicNonperiodicFlow}. As two points with slightly different initial conditions evolve in time in this system, they may have wildly varying trajectories. Using the distance between these two points in time, their Lyapunov exponent can be calculated. This process can be repeated for long durations in time and for several different initial conditions. From this set of Lyapunov exponents, the maximal exponent can be taken to characterize the chaos within this system. As there are no classical trajectories for calculating Lyapunov exponents in Hilbert space, there are several potential metrics that emulate these trajectories and allow a difference to be found. We will use the $L^2$ norm to measure the divergence of the GPE.

The correspondence principle states that quantum mechanics should reproduce classical mechanics in the limit of higher energy states, as stated by Bohr \cite{bohr_n_1920_2051363}. Yet, here lies a puzzling conundrum: chaos abounds in the world of classical mechanics, but the Schr\"odinger equation is fundamentally linear and cannot exhibit chaos, as its Lyapunov exponent is zero. How these ideas are reconciled is still unclear. Chaos seems to emerge from the limit of many particles and the limit of $\hbar$ tending to zero. Although this paper does not discuss the source of quantum chaos, it will instead look towards the effects of Gross-Pitaevskii equation parameters on chaos in BECs.

Previous works have demonstrated chaos in Bose-Einstein condensates through positive Lyapunov exponents \cite{PhysRevA.83.043611,PhysRevLett.62.2065,PhysRevLett.71.2683}. This work replicates these results in one-dimension using finite-difference methods and lays the results to extend these results to higher dimensions. Additionally, we discuss numerical error in both the finite difference method and the adaptive Runge-Kutta solver. We hypothesize the nonlinearity of $g$ gives rise to chaos and the case of $g=0$ should result in a zero Lyapunov exponent. If this is true, it poses additional questions: is the GPE the correct density functional? Does this chaos in BECs exist in reality?

