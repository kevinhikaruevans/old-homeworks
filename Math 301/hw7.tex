\documentclass{homework}

\title{Homework 7}
\author{Kevin Evans}
\studentid{11571810}
\date{October 15, 2020}
\setclass{Math}{301}
\usepackage{amssymb}
\usepackage{mathtools}

\usepackage{amsthm}
\usepackage{amsmath}
\usepackage{slashed}
\usepackage{relsize}
\usepackage{threeparttable}
\usepackage{float}
\usepackage{booktabs}
\usepackage{boldline}
\usepackage{changepage}
\usepackage{physics}
\usepackage[inter-unit-product =\cdot]{siunitx}
\usepackage{setspace}

\usepackage[makeroom]{cancel}
%\usepackage{pgfplots}

\usepackage{enumitem}
\usepackage{times}
\usepackage{multirow}

\usepackage{amsthm}
\newtheorem*{prop}{Proposition}

\renewcommand\qedsymbol{$\blacksquare$}

\begin{document}
	\maketitle
	

	\begin{enumerate}
		\item \begin{minipage}[t]{\linewidth}
			\begin{prop}
				Suppose $n \in \mathbb{Z}$. If $n^2$ is odd, then $n$ is odd.
			\end{prop}
		
			\begin{enumerate}
				\item \begin{proof} Suppose for the sake of contradiction, there is a $n \in \mathbb{Z}$ where $n^2$ is odd and $n$ is even. As $n^2$ is odd, then
					\[ n^2 = 1 \pmod{2} \]
				However, if $n$ is even, then we can write this as $2m$ for some $m \in \mathbb{Z}$. Squaring this expression of $n$, \begin{align*}
					\left(2m\right)^2 & = 2\left(2m^2\right) \\
						& = 0 \pmod{2}
				\end{align*}
				This is a contradiction, as $n^2$ is both even and odd.
				\end{proof}
			
				\item \begin{proof}
					For the contrapositive, we will show if $n$ is even, then $n^2$ is even. Suppose $n$ is even, then it is expressed as twice an integer $m$, \begin{align*}
						n & = 2m \\
						n^2 & = 2(\underbrace{2m^2}_{\in \mathbb{Z}})
					\end{align*}
					Squaring this, it is still twice another integer. Therefore $n^2$ is also even and we have shown the contrapositive to be true.
				\end{proof}
			
				\item \begin{proof}
					Suppose $n^2$ is odd, where $n \in \mathbb{Z}$. As it is odd, it can be expressed as \begin{align*}
						n^2 & = 2m + 1 \\
						n^2 - 1 & = 2m \\
						\left(n - 1\right)\left(n + 1\right) & = 2m
					\end{align*}
					On the LHS, both groups must have the same parity and each group with $n \pm 1$ must have even parity. Therefore, $n$ must be odd as $n - 1$ and $n+1$, and their product will be even.
				\end{proof}
			\end{enumerate}
		\end{minipage}
	
	
		\item \begin{minipage}[t]{\linewidth}
			\begin{prop}
				If $x, y \in \mathbb{Z}$, then $x^2 - 4y - 2 \ne 0$.
			\end{prop}
			\begin{proof}
				For the sake of contradiction, suppose $x, y \in \mathbb{Z}$ and $x^2 - 4y - 2 = 0$. Then solving for $x^2$, \begin{align*}
					x^2 & = 4y - 2 = 2\left(y + 1\right)
					\intertext{As $x^2$ is even, it follows that $x$ is also even. We can express $x = 2z$ for some integer $z$. Using this expression for $x$ in the original equation,}
					0 & = x^2 - 4y - 2 \\
					  & = 4z^2 - 4y - 2 \\
					2 & = 4z^2 - 4y \\
					1 & = 2\left(2z^2 - 2\right)
				\end{align*}
				As $1$ cannot be twice another integer, the original assumptions must be incorrect and the proposition must be true.
			\end{proof}
		\end{minipage}
	
		\item \begin{enumerate}
			\item \begin{minipage}[t]{\linewidth}
				\begin{prop}
					There exists a prime number $p$ such that $p + 4$ and $p+6$ are also prime numbers.
				\end{prop}
				\begin{proof}					
					There exists a prime number $p=7$ where both $p+4=11$ and $p+6=13$ are primes.
				\end{proof}
			\end{minipage}
		
			\item \begin{minipage}[t]{\linewidth}
				\begin{prop}
					There exists prime numbers $p$ and $q$ such that $p+q = 128$.
				\end{prop}
				\begin{proof}					
					There exists prime numbers $p=109$ and $q=19$ such that $p+q = 128$.
				\end{proof}
			\end{minipage}
		\end{enumerate}

		\item \begin{minipage}[t]{\linewidth}
			\begin{prop}
				For integers $a$ and $b > 0$, there exists unique integers $q$ and $r$ such that $a = bq + r$, where $0 \le r < b$.
			\end{prop}
			\begin{proof}
				Suppose there exists two sets of integers $q_i$ and $r_i$ such that \begin{align*}
					a & = bq_1 + r_1 \\
					a & = b q_2 + r_2
				\end{align*}
				and $0 \le r_i < b$. Then, if we subtract the two expressions, \begin{align*}
					0 & = b \left(q_1 - q_2\right) + \left(r_1 - r_2\right) \\
					\frac{r_1 - r_2}{b} & = q_1 - q_2
				\end{align*}
				Because of the original conditions on $r_i$, it must hold true $\abs{r_1 - r_2} / b < 1$. As $q_1 - q_2$ must be an integer as well, it can only be $0$. Therefore $q_1 = q_2$ (and $r_1 = r_2$ as well).
			\end{proof}
		\end{minipage}
	
		\item \begin{minipage}[t]{\linewidth}
			\begin{prop}
				Suppose $a, b, p \in \mathbb{Z}$ and $p$ is a prime number. If $p \mid ab$, then $p \mid a$ or $p \mid b$.
			\end{prop}
			\begin{proof}
				Suppose $a, b, p \in \mathbb{Z}$ and $p$ is a prime number and $p \mid ab$. 
				
				\underline{Case 1:} $p \mid a$.
				
				
				\underline{Case 2:} $p \nmid a$. Then $\gcd(p, a) = 1$ and by Bezout's theorem, \begin{align*}
					1 & = n_1 p + n_2 a && n_1, n_2 \in \mathbb{Z}
					\intertext{Multiplying by $b$ and since $p \mid ba$, we can replace $n_2 ba$ as some integer $n_3$ multiple of $p$,}
					b & = n_1 b p + n_2 b a = n_1 b p + n_3 p \\
					  & = \left(n_1 b + n_3\right)p
				\end{align*}
				Therefore, $p \mid b$.
			\end{proof}
		\end{minipage}
	\end{enumerate}
\end{document}