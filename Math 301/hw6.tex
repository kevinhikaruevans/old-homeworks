\documentclass{homework}

\title{Homework 6}
\author{Kevin Evans}
\studentid{11571810}
\date{October 8, 2020}
\setclass{Math}{301}
\usepackage{amssymb}
\usepackage{mathtools}

\usepackage{amsthm}
\usepackage{amsmath}
\usepackage{slashed}
\usepackage{relsize}
\usepackage{threeparttable}
\usepackage{float}
\usepackage{booktabs}
\usepackage{boldline}
\usepackage{changepage}
\usepackage{physics}
\usepackage[inter-unit-product =\cdot]{siunitx}
\usepackage{setspace}

\usepackage[makeroom]{cancel}
%\usepackage{pgfplots}

\usepackage{enumitem}
\usepackage{times}
\usepackage{multirow}

\usepackage{amsthm}
\newtheorem*{prop}{Proposition}

\renewcommand\qedsymbol{$\blacksquare$}

\begin{document}
	\maketitle
	

	\begin{enumerate}
		\item \begin{minipage}[t]{\linewidth}
			\begin{prop}
				Let $n$ be an integer. Then $n$ is odd if and only if $3n+6$ is odd.
			\end{prop}
			\begin{proof}
				Suppose $n$ is an odd integer. Then $n$ can be represented as \begin{align*}
					n & = 2k + 1 
					\intertext{for some $k \in \mathbb{Z}$. Then the expression $3n+6$ can be written as }
					3n + 6 & = 3 \left(2k + 1\right) + 6 \\
						& = 6k + 9 = 2k' + 1
					\intertext{where $k' = 3k + 4$ and $k' \in \mathbb{Z}$. Therefore, if $n$ is odd, then $3n + 6$ is also odd. Next, we will show the converse is also true. Suppose $3n + 6$ is odd, then}
					3n + 6 & \equiv 1 \pmod{2}
					\intertext{And since $6 \equiv 0 \pmod{2}$, we can subtract this out and}
					3n & \equiv 1 \pmod{2}
				\end{align*}
				For an even $n$, the expression becomes $2(3j) \equiv 0 \pmod{2}$ for $j \in \mathbb{Z}$. For an odd $n=2j'+1$, it equals $2(3j' + 1) \equiv 1 \pmod{2}$ for $j' \in \mathbb{Z}$. Therefore, the converse is only true for odd $n$.
			\end{proof}
		\end{minipage}
	
		\item \begin{minipage}[t]{\linewidth}
			\begin{prop}
				Let $n \in \mathbb{Z}$, then 
				\[n^2 \equiv 0 \pmod{4} \text{ or } n^2 \equiv 1 \pmod{4} \]
			\end{prop}
			\begin{proof}
				Suppose $n$ is an integer. By the division algorithm, $n$ can be expressed as \begin{align*}
					n & = 2q + r
				\end{align*}
				where $q, r \in \mathbb{Z}$ and $0 \le r < 2$, or $r \in \left\{ 0, 1 \right\}$. If we square $n$, then \begin{align*}
					n^2 & = \begin{cases*}
						4q^2 & $r = 0$ \\
						4(q^2 + q) + 1 & $r = 1$
					\end{cases*}
				\intertext{Since $q^2, (q^2 + q) \in \mathbb{Z}$, $n^2$ will either have a remainder of $0$ or $1$ when divided by $4$. Therefore, it holds true that $n^2 \equiv 0 \pmod{4} \text{ or } n^2 \equiv 1 \pmod{4}$. }
				\end{align*}
			\end{proof}
		\end{minipage}

		\item \begin{minipage}[t]{\linewidth}
				\begin{prop}
					If $a, b \in \mathbb{Z}$ and $a^2 + b^2$ is a perfect square, then $a$ and $b$ are not both odd.
				\end{prop}
				\begin{proof}			
					Here, we will show the contrapositive. Suppose $a, b \in \mathbb{Z}$ and both $a$ and $b$ are odd, then using the previous problem, \begin{align*}
						a^2 & \equiv 1 \pmod{4} \\
						b^2 & \equiv 1 \pmod{4}
					\end{align*}
					Then, $\left( a^2 + b^2 \right) \equiv 2 \pmod{4}$. However, this sum cannot be a perfect square, as we have shown in Problem 2: any integer $n$, $n^2 \equiv 0 \pmod{4}$ or $n^2 \equiv 1 \pmod{4}$. 
				\end{proof}
			\end{minipage}
		
		\item \begin{minipage}[t]{\linewidth}
			\begin{prop}
				Suppose the division algorithm applied to $a$ and $b$ yields $a = qb + r$, then \[ \gcd(a, b) = \gcd(r, b) \]
			\end{prop}
			\begin{proof}			
				Suppose $a, b, q, r \in \mathbb{Z}$ and $a = qb + r$, where $0 \le r < b$. Then let $d$ be a divisor of $a$ and $b$. Then it must hold true that $d$ also divides $r$, \begin{align*}
					a & = d x_1 \\
					b & = d x_2 \\
					r & = d(x_1 - x_2 q)
				\end{align*}
				where $x_i \in \mathbb{Z}$. Since the set of divisors are equal between $a, b$ and $r, b$, then there is one greatest common divisor and $\gcd(a, b) = \gcd(r, b)$.
			\end{proof}
		\end{minipage}	
	\end{enumerate}
\end{document}