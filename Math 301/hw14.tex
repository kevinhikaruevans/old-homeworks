\documentclass{homework}

\title{Homework 14}
\author{Kevin Evans}
\studentid{11571810}
\date{December 10, 2020}
\setclass{Math}{301}
\usepackage{amssymb}
\usepackage{mathtools}

\usepackage{amsthm}
\usepackage{amsmath}
\usepackage{slashed}
\usepackage{relsize}
\usepackage{threeparttable}
\usepackage{float}
\usepackage{booktabs}
\usepackage{boldline}
\usepackage{changepage}
\usepackage{physics}
\usepackage[inter-unit-product =\cdot]{siunitx}
\usepackage{setspace}

\usepackage[makeroom]{cancel}
%\usepackage{pgfplots}

\usepackage{enumitem}
\usepackage{times}
\usepackage{multirow}

\usepackage{amsthm}
\newtheorem*{prop}{Proposition}

\renewcommand\qedsymbol{$\blacksquare$}

\begin{document}
	\maketitle
	\begin{enumerate}
		\item \begin{enumerate}
			\item  \begin{minipage}[t]{\linewidth}
				\begin{prop}
					If both $f$ and $g$ are injective, then $g\circ f$ is injective.
				\end{prop}
				\begin{proof} 
					Suppose $f: A \to B$ and $g : B \to C$ are injective. Then for all $a, a' \in A$ where $a \ne a'$, $f(a) \ne f(a')$, i.e. a different input to $f$ results in a different output. Then since $g$ is injective too, $g(f(a)) \ne g(f(a'))$. Therefore $g \circ f$ is injective.
				\end{proof}
			\end{minipage}
		
			\item  \begin{minipage}[t]{\linewidth}
				\begin{prop}
					If both $f$ and $g$ are surjective, then $g\circ f$ is surjective.
				\end{prop}
				\begin{proof} 
					Suppose $f: A \to B$ and $g : B \to C$ are surjective.
				\end{proof}
			\end{minipage}
			
		\end{enumerate}
	\end{enumerate}
\end{document}