\documentclass{homework}

\title{Homework 2}
\author{Kevin Evans}
\studentid{11571810}
\date{September 1, 2020}
\setclass{Math}{301}
\usepackage{amssymb}
\usepackage{mathtools}

\usepackage{amsthm}
\usepackage{amsmath}
\usepackage{slashed}
\usepackage{relsize}
\usepackage{threeparttable}
\usepackage{float}
\usepackage{booktabs}
\usepackage{boldline}
\usepackage{changepage}
\usepackage{physics}
\usepackage[inter-unit-product =\cdot]{siunitx}
\usepackage{setspace}

\usepackage[makeroom]{cancel}
%\usepackage{pgfplots}

\usepackage{enumitem}
\usepackage{times}
\usepackage{multirow}


\begin{document}
	\maketitle
	\begin{enumerate}
		\item \begin{enumerate}
			\item If the determinate of a matrix is not zero, then a matrix is invertable.
			\item If $\abs{r} < 1$, then a geometric series with common ratio $r$ converges.
			\item If a function is continuous, then it is intergrable.
			\item If a function is differentiable, then the function is continuous.
			\item If I'm wearing a hat, then it is sunny.
		\end{enumerate}
	
		\item $\begin{aligned}[t]
			\text{Let } P & = \text{The Curiosity Rover is on Mars.} \\
			Q & = \text{The Curiosity Rover is a good robot.} \\
			R & = \text{The Mars Polar Lander is a good robot.}
		\end{aligned}$
		
		And we know that $P$, $Q \vee R$, and $R \implies \neg P$ are all true. From $R \implies \neg P$,
		\begin{center}
			\begin{tabular}{cc|cl}
				$R$ & $\neg P$ & $R \implies \neg P$ & \\
				\midrule
				T & T & T & \\
				T & F & F & \\
				F & T & T & \\
				F & F & T & ($*$)
			\end{tabular}
		\end{center}
		As $\neg P$ is false, the bottom row ($*$) intersects with $R \implies P$. So, $R$ must be false.
		
		\begin{enumerate}
			\item True. Since $Q \vee R$ is true and $R$ is false, $Q$ must be true.
			\item False. Shown above, $R$ is false.
		\end{enumerate}
	
		\item If $P$ is false, then $P \wedge Q$ is false. If the original statement is true, then both sides of $\iff$ must be equal, i.e. $(R \implies S)$ must be false too. For an implication to be false, then \begin{align*}
			R & = \text{true} \\
			S & = \text{false}
		\end{align*}
	
		It's impossible to know what $Q$ is as $P$ is false and $P \wedge Q$ will always be false regardless of $Q$'s value.
		
		\item For the implication $((P \wedge Q) \vee R) \implies (R \vee S)$ to be false, then \begin{align*}
			((P \wedge Q) \vee R) & = \text{true} \\
			(R \vee S) & = \text{false}
			\intertext{From the latter, both $R$ and $S$ must be false, so}
				(P \wedge Q) & = \text{true}
			\intertext{This is only the case when both are true, thus}
				P = Q & = \text{true} \\
				R = S & = \text{false}
		\end{align*}
	
		\item Building out a truth table, we can see both statements are  equivalent, \begin{center}
			\begin{tabular}{cc|c|c}
				$P$ & $Q$ & $(P \vee Q) \wedge \neg (P \wedge Q)$ & $(P \wedge \neg Q) \vee (Q \wedge \neg P)$ \\ \midrule
				T & T & F & F \\
				T & F & T & T \\
				F & T & T & T \\
				F & F & F & F
			\end{tabular}
		\end{center}
	
		\item Applying de Morgan's law to the second statement, \begin{align*}
			\neg \left( \left( P \wedge Q\right) \wedge \neg R\right) & = \neg \left( P \wedge Q\right) \vee R
			\intertext{For the two original statements to be equivalent, then it would mean}
			\neg \left( P \wedge Q\right) & \stackrel{?}{=} \left(P \implies Q\right)
			\intertext{...which is certainly false. These statements are \underline{not} equivalent.}
		\end{align*}
	
	\end{enumerate}
\end{document}