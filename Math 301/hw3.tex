\documentclass{homework}

\title{Homework 3}
\author{Kevin Evans}
\studentid{11571810}
\date{September 10, 2020}
\setclass{Math}{301}
\usepackage{amssymb}
\usepackage{mathtools}

\usepackage{amsthm}
\usepackage{amsmath}
\usepackage{slashed}
\usepackage{relsize}
\usepackage{threeparttable}
\usepackage{float}
\usepackage{booktabs}
\usepackage{boldline}
\usepackage{changepage}
\usepackage{physics}
\usepackage[inter-unit-product =\cdot]{siunitx}
\usepackage{setspace}

\usepackage[makeroom]{cancel}
%\usepackage{pgfplots}

\usepackage{enumitem}
\usepackage{times}
\usepackage{multirow}


\begin{document}
	\maketitle
	\begin{enumerate}
		\item[1.] \begin{enumerate}
			\item All integers are a multiple of 2.
			\item It's false, since $1 \in \mathbb{Z}$ and $1$ is not a multiple of $2$.
			\item There is an integer that is not a multiple of 2.
			\item $\exists \; x \in \mathbb{Z} : (x \text{ is not a multiple of }2)$
		\end{enumerate}
	
		\item[2.] \begin{enumerate}
			\item $\left(\sqrt{2} < x\right) \wedge \left(x < \sqrt{3}\right)$
			\item $\begin{aligned}[t]
				& \neg \left[\exists x \in \mathbb{Q}, \left(\sqrt{2} < x\right) \wedge \left(x < \sqrt{3}\right)\right] \\
				& = \forall x \in \mathbb{Q}, \left(\sqrt{2} \ge x\right) \vee \left( x \ge \sqrt{3}\right)
			\end{aligned}$
			\item All rational numbers are equal to or less than $\sqrt{2}$, or equal to or greater than $\sqrt{3}$.
		\end{enumerate}
	
		\item[3.] \begin{enumerate}
			\item $\begin{aligned}[t]
				E(x) & = x \text{ is even.} \\
				O(x) & = x \text{ is odd.} \\
				\forall x \in \mathbb{Z}, &  \: E(x) \oplus O(x) \qquad \text{where $\oplus$ is an XOR operator.}
			\end{aligned}$
			\item $\begin{aligned}[t]
				\exists x \in \mathbb{Z}, \left( E(x) \wedge O(x) \right) \vee \left(\neg E(x) \wedge \neg O(x)\right)
			\end{aligned}$
			
			\item There is an integer that is either: both an odd and even integer, or is neither an odd or even integer.
		\end{enumerate}
	
		\item[4.] \begin{enumerate}
			\item $\forall x \in \mathbb{Z}, E(x) \implies O(x + 1)$
			\item $\exists x \in \mathbb{Z}, E(x) \wedge E(x+1)$
			\item There is an integer that even and if you add one to that integer, it's also even.
		\end{enumerate}
	
		\item[5.] \begin{enumerate}
			\item ...$\forall \: x_1, x_2 \in \mathbb{R}, x_1 \le x_2 \implies f(x_1) \ge f(x_2)$
			\item ...$\exists \: x_1, x_2 \in \mathbb{R}, x_1 \le x_2 \wedge f(x_1) < f(x_2)$
		\end{enumerate}
	
		\item[6.] I'm going to define $\mathbb{R}^+$ as the set of positive real integers. \begin{enumerate}
			\item ...$\forall x \in \mathbb{R}, \varepsilon \in \mathbb{R^+}, \exists \delta \in \mathbb{R^+}, \left(\abs{x - a} < \delta\right) \implies \left(\abs{f(x) - f(a)} < \varepsilon \right)$
			
			\item ...$\exists x \in \mathbb{R}, \varepsilon \in \mathbb{R}^+, \forall \delta \in \mathbb{R}^+, \left(\abs{x-a} < \delta\right) \wedge \left(\abs{f(x) - f(a)} \ge \varepsilon\right)$
		\end{enumerate}
	\end{enumerate}
\end{document}