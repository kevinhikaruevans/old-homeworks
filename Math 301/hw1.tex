\documentclass{homework}

\title{Homework 1}
\author{Kevin Evans}
\studentid{11571810}
\date{August 28, 2020}
\setclass{Math}{301}
\usepackage{amssymb}
\usepackage{mathtools}

\usepackage{amsthm}
\usepackage{amsmath}
\usepackage{slashed}
\usepackage{relsize}
\usepackage{threeparttable}
\usepackage{float}
\usepackage{booktabs}
\usepackage{boldline}
\usepackage{changepage}
\usepackage{physics}
\usepackage[inter-unit-product =\cdot]{siunitx}
\usepackage{setspace}

\usepackage[makeroom]{cancel}
%\usepackage{pgfplots}

\usepackage{enumitem}
\usepackage{times}
\usepackage{multirow}


\begin{document}
	\maketitle

	\begin{enumerate}
		\item \begin{enumerate}
			\item This is a false statement, because all square matrices are not invertible.
			\item This is a false statement, as the $\det$ is zero.
			\item This is not a valid statement, since the qualifier of ``happy'' isn't well-defined. Also, ``clap your hands'' is not really a conclusion.
			\item This is not a valid statement, as ``today'' is ambiguous. 
		\end{enumerate}
	
		\item \begin{enumerate}
			\item $\begin{aligned}[t]
				P & : \text{The number $25$ is even} \\
				Q & : \text{The number $25$ is a power of 3} \\
				& \boxed{P \wedge Q}
			\end{aligned}$
			
			\item $\begin{aligned}[t]
			P & : \text{14 is a prime number} \\
			& \boxed{\neg P}
			\end{aligned}$
			
			\item $\begin{aligned}[t]
			P & : \text{A number is even} \\
			Q & : \text{A number is odd} \\
			& \boxed{P \vee Q}
			\end{aligned}$
			
			\item $\begin{aligned}[t]
				P & : \text{I am not here} \\
				Q & : \text{I am probably somewhere else} \\
				& \boxed{P \implies Q}
			\end{aligned}$
		\end{enumerate}
	
		\item If $(P \implies Q) = 1$ and $\neg Q = 1$, then $Q=0$ and $P=0$, \begin{enumerate}
			\item $P$ is false.
			\item $P \vee Q = \left(0 \vee 0\right)$ is false.
			\item $P \wedge Q = \left(0 \wedge 0\right)$ is also false.
		\end{enumerate}
	
		\item \begin{enumerate}
			\item \begin{tabular}[t]{ccc|c|c}
				$P$ & $Q$ & $R$ & $Q \implies R$ & $P \vee \left(Q \implies R\right)$ \\
				\midrule
				$1$ & $1$ & $1$ & $1$ & $1$ \\
				$1$ & $1$ & $0$ & $0$ & $1$\\
				$1$ & $0$ & $1$ & $1$ & $1$ \\
				$1$ & $0$ & $0$ & $1$ & $1$ \\
				$0$ & $1$ & $1$ & $1$ & $1$\\
				$0$ & $1$ & $0$ & $0$ & $0$\\
				$0$ & $0$ & $1$ & $1$ & $1$\\
				$0$ & $0$ & $0$ & $1$ & $1$\\
			\end{tabular}
		
			\item \begin{tabular}[t]{cc|cc|c}
				$P$ & $Q$ & $\neg P$ & $\left(P \wedge \neg P\right)$ & $\left(P \wedge \neg P\right) \vee Q$ \\
				\midrule
				$1$ & $1$ & $0$ & $0$ & $1$ \\
				$1$ & $0$ & $0$ & $0$ & $0$\\
				$0$ & $1$ & $1$ & $0$ & $1$\\
				$0$ & $0$ & $1$ & $0$ & $0$\\
			\end{tabular}
			
			The statement reduces to $Q$.
			
			\item \begin{tabular}[t]{cc|cc|c}
				$P$ & $Q$ & $\neg P$ & $\left(P \vee \neg P\right)$ & $\left(P \vee \neg P\right) \wedge Q$ \\
				\midrule
				$1$ & $1$ & $0$ & $1$ & $1$ \\
				$1$ & $0$ & $0$ & $1$ & $0$\\
				$0$ & $1$ & $1$ & $1$ & $1$\\
				$0$ & $0$ & $1$ & $1$ & $0$\\
			\end{tabular}
		
			The statement reduces to $Q$ again.
		\end{enumerate}
	\end{enumerate}
\end{document}