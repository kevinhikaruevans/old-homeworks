\documentclass{homework}

\title{Homework 4}
\author{Kevin Evans}
\studentid{11571810}
\date{September 17, 2020}
\setclass{Math}{301}
\usepackage{amssymb}
\usepackage{mathtools}

\usepackage{amsthm}
\usepackage{amsmath}
\usepackage{slashed}
\usepackage{relsize}
\usepackage{threeparttable}
\usepackage{float}
\usepackage{booktabs}
\usepackage{boldline}
\usepackage{changepage}
\usepackage{physics}
\usepackage[inter-unit-product =\cdot]{siunitx}
\usepackage{setspace}

\usepackage[makeroom]{cancel}
%\usepackage{pgfplots}

\usepackage{enumitem}
\usepackage{times}
\usepackage{multirow}


\begin{document}
	\maketitle
	\begin{enumerate}
		\item[1.] \begin{enumerate}
			\item $A \cup B = \left\{ 1, 5, 3, 6, 7, 8 \right\}$
			\item $\begin{aligned}[t]
				A \cap \left(B \cup C\right) & = A \cap \left\{ 3, 5, 6, 7, 8\right\} \\
				& = \left\{ 3, 5, 6, 7\right\}
			\end{aligned}$
			\item $A - B = \left\{1, 3 \right\}$
			\item $\begin{aligned}[t]
				C - A & = \left\{ 8 \right\} \\
				\left(C - A\right) \cup B & = \left\{ 5, 6, 7, 8\right\} \\
				\left(\left(C - A\right) \cup B\right) \cap A& = \left\{ 3, 5, 6, 7 \right\}
			\end{aligned}$
			
			\item $\varnothing$ (as both sets are disjoint)
		\end{enumerate}
	
	
		\item[2.] \begin{enumerate}
			\item $\left(A \cup C\right) - B$
			\item $\left(A \cup C\right) - \left(A \cap B\right)$
			\item $\left(A \cup B \cup C\right) - \left(A \cap B \cap C\right)$
			\item $\left(A \cap B\right) \cup \left(A \cap C\right) \cup \left(B \cap C\right)$
			\item $\left(A \cup B \cup C\right) - \left(A \cap B\right) \cup \left(A \cap C\right) \cup \left(B \cap C\right)$
		\end{enumerate}
	
		\item[3.] \begin{enumerate}
			\item $A \cup B - \left(A \cap C\right) - \left(A \cap B \cap \overline{C}\right)$
			\item $\left(A \cap \overline{B} \cap \overline{C}\right)\cup \left(B \cap C\right)$
			\item $\overline{\left(B - \left(A \cup C\right)\right) \cup \left(C - \left(A \cup B\right)\right)}$
			\item $\overline{ \left(A \cup C\right) - B }$
			\item $\overline{B \cup \left(A \cap \overline{C}\right)}$
		\end{enumerate}
	
		\item[4.] \textbf{Proposition:} If $n$ is an odd integer, then $n^2 + 4n + 6$ is odd.
		
		\textit{Proof.} Let $n$ be an odd integer, then $n$ can be expressed as \begin{align*}
			n & = 2a + 1 && \text{...where $a \in \mathbb{Z}$} \\
			n^2 + 4n + 6 & = \left(2a + 1\right)^2 + 4(2a + 1) + 6 && \text{Substition for $n$}\\
				& = 4a^2 + 4a + 1 + 8a + 4 + 6 && \text{Expanding the terms} \\
				& = 2(\underbrace{2a^2 + 6a + 5}_{b}) + 1 && \text{From closure, $b \in \mathbb{Z}$}\\
				& = 2b + 1 && \text{The result is odd.} \qed
		\end{align*}
	
		\item[5.] \textbf{Proposition:} if two integers have the opposite parity, their product is even.
		\textit{Proof.} Let $a$ have even parity and $b$ have odd parity, then $a$ and be can be expressed as \begin{align*}
			a & = 2n \\
			b & = 2m + 1 && \text{where $n, m \in \mathbb{Z}$}
			\intertext{The product $ab$ becomes}
			ab & = \left(2n\right)\left(2m + 1\right) \\
				& = 4nm + 2n \\
				& = 2(\underbrace{2nm + n}_{c}) && \text{Closure, $c \in \mathbb{Z}$}\\
				& = 2c && \text{The result is always even. \qed}
		\end{align*}
		Two cases are not needed since multiplication is commutative, i.e. $ab = ba$.
	\end{enumerate}
\end{document}