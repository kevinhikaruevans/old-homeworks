\documentclass{homework}

\title{Homework 8}
\author{Kevin Evans}
\studentid{11571810}
\date{October 22, 2020}
\setclass{Math}{301}
\usepackage{amssymb}
\usepackage{mathtools}

\usepackage{amsthm}
\usepackage{amsmath}
\usepackage{slashed}
\usepackage{relsize}
\usepackage{threeparttable}
\usepackage{float}
\usepackage{booktabs}
\usepackage{boldline}
\usepackage{changepage}
\usepackage{physics}
\usepackage[inter-unit-product =\cdot]{siunitx}
\usepackage{setspace}

\usepackage[makeroom]{cancel}
%\usepackage{pgfplots}

\usepackage{enumitem}
\usepackage{times}
\usepackage{multirow}

\usepackage{amsthm}
\newtheorem*{prop}{Proposition}

\renewcommand\qedsymbol{$\blacksquare$}

\begin{document}
	\maketitle
	

	\begin{enumerate}
		\item \begin{proof}[Disproof]
			This statement is false, as $\exists \: x, y \in \mathbb{Z} : \abs{x+y} \ne \abs{x} + \abs{y}$.
			
			Counterexample: Suppose $x = -1$ and $y = 1$, then $\abs{x+y} = 0$, but $\abs{x} + \abs{y} = 1$.
		\end{proof} 
	
		\item  \begin{proof}[Disproof]
			This statement is false, as if any two of the integers is zero and the other integer is non-zero and odd, then the products will always be even. 
			
			Counterexample: Suppose $a, c = 0$ and $b = 1$. Then \[ ab = bc = ac = 0 \]
			Therefore although the products have the same parity, the constituents do not.
		\end{proof} 
	
		\item \begin{minipage}[t]{\linewidth}
			\begin{prop}
				Every odd integer is the sum of three odd integers.
			\end{prop}
			\begin{proof}
				Suppose there are three odd integers $o_1$, $o_2$, and $o_3$. These can be expressed of the form $$o_i = 2n_i + 1$$where $n_i \in \mathbb{Z}$. If we sum these three odd integers, it results another odd integer, \begin{align*}
					o_1 + o_2 + o_3 & = 2n_1 + 2n_2 + 2n_3 + 3 \\
						& = 2 \bigg( \underbrace{n_1 + n_2 + n_3 + 1}_{\in \mathbb{Z}} \bigg) + 1
				\end{align*}
				Therefore, every odd integer can be expressed as a sum of three odd integers. 
			\end{proof}
		\end{minipage}
	
		\item  \begin{proof}[Disproof]
			This statement is false, as if there are shared elements between the sets, the LHS will count the intersection once, whereas they will be doublecounted on the RHS.
			
			Counterexample: Suppose $A = \{1\}$, $B = \{1\}$. Then $A \cup B = \{1\}$ and $\abs{ A \cup B } = 1$. However, $\abs{A} + \abs{B} = 2$. And $1 \ne 2$.
		\end{proof} 
	
		\item \begin{minipage}[t]{\linewidth}
			\begin{prop}
				If $a, b, c \in \mathbb{Z}$, then at least one of $a-b$, $a+c$, and $b-c$ is even.
			\end{prop}
			\begin{proof}
				Suppose the proposition is \textit{false}. Then $a-b$, $a+c$ and $b-c$ are all odd. However, if we subtract $b-c$ from $a-b$, it results in an even number \begin{align*}
					\left(a-b\right) - \left(b - c\right) & = \left(2k_1 + 1\right) - \left(2k_2 + 1\right) && k_i \in \mathbb{Z} \\
						& = 2\left(k_1 - k_2\right)
				\end{align*}
				However, this leads to a contradiction because $\left( a-b \right) = \left(b - c\right) = a+c$, and $a+c$ cannot be both even and odd. Thus the original proposition is true.
			\end{proof}
		\end{minipage}
	
		\item \begin{proof}[Disproof]
			The statement is false. 

			Counterexample:	Let $a=6$, $b=3$, $c=4$. Then $6 \mid 12$, but $6 \nmid 3$ and $6\nmid 4$.
		\end{proof}
	
		\item \begin{proof}[Disproof]
			The statement is false.
			
			Counterexample: Let $a = b = 0$. Then $a + b = ab = 0$, and $0 \nless 0$.
		\end{proof}
	
		\item \begin{proof}[Disproof]
			The statement is false. If $x=0$, then $y$ can be any integer and the statement will hold true.
			
			Counterexample: Let $x = 0$, $y=1$. Then $\abs{x + y} = \abs{x - y}$, but $y \ne 0$.
		\end{proof}
	\end{enumerate}
\end{document}