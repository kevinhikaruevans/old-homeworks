%\documentclass{article}
\documentclass[varwidth=true, border=10pt]{standalone}

\pagestyle{empty}
\headheight=0pt
\headsep=0pt

\title{Homework 1}
\author{Kevin Evans}
%\studentid{11571810}
%\date{August 28, 2020}
%\setclass{Math}{301}
\usepackage{amssymb}
\usepackage{mathtools}

\usepackage{amsthm}
\usepackage{amsmath}
\usepackage{slashed}
\usepackage{relsize}
\usepackage{threeparttable}
\usepackage{float}
\usepackage{booktabs}
\usepackage{boldline}
\usepackage{changepage}
\usepackage{physics}
\usepackage[inter-unit-product =\cdot]{siunitx}
\usepackage{setspace}

\usepackage[makeroom]{cancel}
%\usepackage{pgfplots}

\usepackage{enumitem}
\usepackage{times}
\usepackage{multirow}


\begin{document}
	\begin{enumerate}
		
		
		\item[5.] \textbf{Proposition:} if two integers have the opposite parity, their product is even.
\textit{Proof.} Let $a$ have even parity and $b$ have odd parity, then $a$ and be can be expressed as \begin{align*}
	a & = 2n \\
	b & = 2m + 1 && \text{where $n, m \in \mathbb{Z}$}
	\intertext{The product $ab$ becomes}
	ab & = \left(2n\right)\left(2m + 1\right) \\
	& = 4nm + 2n \\
	& = 2(\underbrace{2nm + n}_{c}) && \text{Closure, $c \in \mathbb{Z}$}\\
	& = 2c && \text{The result is always even. \qed}
\end{align*}
Two cases are not needed since multiplication is commutative, i.e. $ab = ba$.



\end{enumerate}
\end{document}