\documentclass{homework}

\title{Problem Set 6}
\author{Kevin Evans}
\studentid{11571810}
\date{March 23, 2021}
\setclass{Physics}{463}
\usepackage{amssymb}
\usepackage{mathtools}
\usepackage{graphicx}
\usepackage{amsthm}
\usepackage{amsmath}
\usepackage{slashed}
\usepackage{boldline}
\usepackage{physics}
\usepackage{tcolorbox}
\usepackage[inter-unit-product =\cdot]{siunitx}

\usepackage[makeroom]{cancel}
\usepackage{booktabs}
\usepackage{multirow}

\usepackage{times}
\usepackage{mhchem}
\usepackage{mathtools}
\usepackage{tabularx}
\usepackage{listings}


\newcommand{\fm}{\femto\meter}
\newcommand{\solution}{	\vspace{1em} \textit{Solution.} \quad }

\begin{document}
	\maketitle
	\begin{enumerate}
		\item % 6.1
			\textbf{\textit{Kinetic energy of electron gas.}} Show that the kinetic energy of a three-dimensional gas of $N$ free electrons at \SI{0}{\K} is $$U_0 = \frac{3}{5} N \epsilon_F.$$
			\solution
			From (12), the energy of the $k$th state is given by (12), \begin{align*}
				\epsilon_K & = \frac{\hbar^2}{2 m } k^2.
				\intertext{The average kinetic energy is then}
				u_0 & = \expval{\epsilon_K} \\
					& = \frac{ \int \epsilon_K f(k) \dd[3]{\bvec{k}} }{\int f(k) \dd[3]{\bvec{k}}}.
				\intertext{As it's fully populated from $0$ to $k_F$ (so $f(k)$ is a step function) and as the non-radial components divide out, the energy per free electron is}
				u_0	& = \frac{\hbar^2}{2m} \frac{\int_0^{k_F} k^4 \dd{k} }{\int_0^{k_F} k^2 \dd{k}} \\
					& = \frac{\hbar^2}{2m} \frac{k_F^5 / 5}{k^3 / 3} \\
					& = \frac{3 \hbar^2 k_F^2}{10 m} = \frac{3}{5} \epsilon_F.
				\intertext{For $N$ free electrons, we can just multiply by $N$,}
				U_0 & = N u_0 = \frac{3}{5} N \epsilon_F. \qed
			\end{align*}
			
		\pagebreak
		
		\item % 6.2, (a) hint
			\textbf{\textit{Pressure and bulk modulus of an electron gas.}} \begin{enumerate}
				\item Derive a relation connecting the pressure and volume of an electron gas at \SI{0}{\K}. Hint: Use the result of Problem 1 and the relation between $\epsilon_F$ and the electron concentration. The result may be written as $p = \frac{2}{3} (U_0 / V)$. Hint: You can use the general relation $p = - \pdv*{U}{V}$ at constant entropy and at absolute zero, all processes are at constant entropy.
				
				\solution From Problem 1 and from the definition of the Fermi energy (17), \begin{align*}
					U_0 & = \frac{3}{5} N \epsilon_F = \frac{3}{5} N \frac{\hbar^2}{2m} \left(3 \pi^2 N\right)^{2/3} V^{-2/3}. \\
					p & = \eval{-\pdv{U_0}{V}}_S \\
						& = -(\dots) \left(-\frac{2}{3}\right) V^{-5/3}. \\
						& = \frac{2}{3} (\dots) V^{-2/3} V^{-3/3} \\
						& = \frac{2}{3} \frac{U_0}{V}. \qed
				\end{align*}
				
				\item Show that the bulk modulus $B=-V(\pdv*{p}{V})$ of an electron gas at \SI{0}{\K} is $$B = 5 p / 3 = 10 U_0 / 9V.$$
				
				\solution From part (a), the bulk modulus is \begin{align*}
					B & = -V \pdv{p}{V} = -V \frac{2}{3}\left( \frac{1}{V} \pdv{U_0}{V} + U_0 \pdv{V} V^{-1} \right) \\
					& = -\frac{2V}{3}  \left(
						-\frac{p}{V}
						- \frac{U_0}{V^2}
					\right) \\
					& = \frac{2p}{3} + \underbrace{ \frac{2}{3} \frac{U_0}{V}}_{p} \\
					& = \frac{5p}{3}. \qed
				\end{align*}
				
				\item Estimate for potassium using Table 1, the value of the electron gas contribution to $B$.
				
				\solution For potassium, we can estimate the bulk modulus as \begin{align*}
					B & = \frac{10 U_0}{9V} = \frac{10}{9} \frac{N}{V} \frac{3}{5} \epsilon_F \\
						& =\frac{2}{3} \left( \SI{1.40e22}{\per\centi\meter\cubed}\right) \left(\SI{2.12}{\eV}\right) \\
						& \approx \SI{2e22}{\eV\per\centi\meter\cubed}. \qquad (\approx \SI{3.1}{\GPa})
				\end{align*}
			\end{enumerate}
		
		\pagebreak
		
		\item % 6.3
			\textbf{\textit{Chemical potential in two dimensions.}} Show that the chemical potential of a Fermi gas in two dimensions is given by $$\mu(T) = k_B T \ln[\exp(\pi n \hbar^2 / m k_B T) - 1],$$
			for $n$ electrons per unit area. Note: The density of orbitals of a free electron gas in two dimensions is independent of energy, $$D(\epsilon) = m / \pi \hbar^2,$$
			per unit area of specimen.
			
			\solution From the Fermi-Dirac distribution, \begin{equation}
				f(\epsilon) = \frac{1}{\exp[(\epsilon - \mu) / k_B T] + 1}, \tag{6.5}
			\end{equation}
			and using the provided density of energy $\epsilon$, the electron area density is \begin{align*}
				n & = \int D(\epsilon) f(\epsilon) \dd{\epsilon} \\
					& = \frac{m}{\pi \hbar^2} \int_0^\infty \frac{1}{\exp[(\epsilon - \mu) / k_B T] + 1} \dd{\epsilon} \\
					& = \frac{m}{\pi \hbar^2} k_B T \ln[\exp(\mu / k_B T) + 1]. && \text{(WolframAlpha)}
				\intertext{Solving for $\mu(T)$,}
				\ln[\dots] & = \frac{n n \hbar^2}{m k_B T} \\
				\exp(\dots) & = \exp(\frac{n n \hbar^2}{m k_B T}) - 1 \\
				\mu(T) & =  k_B T \ln[\exp(\pi n \hbar^2 / m k_B T) - 1]. \qed
			\end{align*}
		
		\item % 6.5
		\textbf{\textit{Liquid \ce{He^3}.}} The atom \ce{He^3} has spin $\frac{1}{2}$ and is a fermion. The density of liquid \ce{He^3} is \SI{0.081}{\g \per \centi\meter\cubed} near absolute zero. Calculate the Fermi energy $\epsilon_F$ and the Fermi temperature $T_F$.
		
		\solution Using the definition of the Fermi energy, \begin{align*}
			\epsilon_F & = \frac{\hbar^2}{2m} \left( 3 \pi^2  \frac{N}{V}\right)^{2/3},
			\intertext{and using the mass of Helium-3 (\SI{3.016}{\atomicmassunit}), }
			\epsilon_F & = \frac{(\SI{1.054e-34}{\J\s})^2}
								{2 (\SI{3.016}{\atomicmassunit} \times \SI{1.66e-24}{\g} )}
								\left(
									3 \pi^2 \times \frac{ \SI{0.081}{\g\per\centi\meter\cubed} }{\SI{3}{\g\per\mol}}
								\right)^{2/3} \\
						& = \SI{9.57e-39}{\J}.
			\intertext{The associated Fermi temperature is}
			T_F & = \epsilon_F / k_B \\
				& = \SI{7e-16}{\K}.
		\end{align*}
	\end{enumerate}
\end{document}