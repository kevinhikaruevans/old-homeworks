\documentclass{homework}

\title{Homework 5}
\author{Kevin Evans}
\studentid{11571810}
\date{September 28, 2021}
\setclass{Math}{364}
\usepackage{amssymb}
\usepackage{mathtools}
\usepackage{graphicx}
\usepackage{amsthm}
\usepackage{amsmath}
\usepackage{slashed}
\usepackage{boldline}
\usepackage{physics}
\usepackage{tcolorbox}
\usepackage[inter-unit-product =\cdot]{siunitx}

\usepackage[makeroom]{cancel}
\usepackage{booktabs}
\usepackage{multirow}

\usepackage{times}
\usepackage{mhchem}
\usepackage{enumitem}
\usepackage[normalem]{ulem}


%\usepackage{calligra}
%\DeclareMathAlphabet{\mathcalligra}{T1}{calligra}{m}{n}
%\DeclareFontShape{T1}{calligra}{m}{n}{<->s*[2.2]callig15}{}
%\newcommand{\scriptr}{\mathcalligra{r}\,}
%\newcommand{\boldscriptr}{\pmb{\mathcalligra{r}}\,}
%\newcommand{\emf}{\mathcal{E}}
\newcommand{\st}{\mathrm{s.t.}}
\newcommand{\solution}{	\vspace{1em} \textit{Solution.} \quad }
\newcommand{\bolditem}[1][YYY]{\item[\textbf{#1}]}
\newenvironment{amatrix}[1]{%
	\left(\begin{array}{@{}*{#1}{c}|c@{}}
	}{%
	\end{array}\right)
}
\begin{document}
	\maketitle
	\begin{enumerate}
		\bolditem[3.3.3] Use the simplex method to do the following problem. The problem is stated in cannonical form with basic variables $x_2$ and $x_3$. Notice that in the first step in the simplex method, either $x_1$ or $x_4$ can enter the basis. \begin{align*}
			\text{Minimize\quad} & -x_1 - 2x_4 + x_5 \\
			\st\quad & x_1 + x_3 + 6x_4 + 3x_5 = 2 \\
				& -3x_1 + x_2 + 3x_4 + x_5 = 3 \\
				& x \ge 0.
		\end{align*}
		
		\solution Rewriting this to the format used in-class, \begin{align*}
			x_1 + x_3 + 6 x_4 + 3x_5 & = 2 \\
			-3x_1 + x_2 + 3 x_4 + x_5 & = 3 \\
			-x_1 - 2x_4 + x_5 & = z
			\intertext{We can try letting $x_1$ enter the basis. This leads to either $x_1=2$ (swap $x_3$) or $x_1 = \text{anything}$ (swap $x_2$). Swapping for $x_3$ is the more limiting constraint, so we can pivot on the first line. The system becomes}
			x_1 + x_3 + 6x_4 + 3x_5 & = 2 \\
			x_2 + 3x_3 + 21 x_4 + 10x_5 & = 9 \\
			x_3 + 4 x_4 + 4 x_5 & = 2 + z
			\intertext{All the coefficients in the objective constraint are positive, so we have arrived at the optimal solution,}
			x^* & = (2, 9, 0, 0, 0) \\
			z^* & = -2.
		\end{align*}
		
		\bolditem[3.4.2] Solve using the simplex method. \begin{enumerate}
			\item[(d)] Minimize $x_3 - x_4$ subject to \begin{align*}
				x_1 - x_4 & = 5 \\x_2 + 2 x_3 & = 10 \\
				x \ge 0
			\end{align*}
			\solution Using the format from class, \begin{align*}
				x_1 - x_4 & = 5 \\
				x_2 + 2x_3 &= 10 \\
				x_3 - x_4 & = z.
			\end{align*} By Theorem 3.4.2, the objective function is unbounded below, as there is an index $s=4$, where $c_s \le 0$ and $a_{is} \le 0 \: \forall \: i$.
			
			\item[(e)] Minimize $-x_3 + x_4$ subject to the constraints of (d). 
			
			\solution  Using the format from class, \begin{align*}
				x_1 - x_4 & = 5 \\
				x_2 + 2x_3 &= 10 \\
				-x_3 + x_4 & = z
				\intertext{We can swap $x_3$ for $x_2$, only needing to change the objective function,}
				x_1 - x_4 & = 5 \\
				x_2 + 2x_3 &= 10 \\
				x_2 + 2x_4 & = 20 + 2z.
				\intertext{The coefficients are positive, so we're at an optimal solution,}
				x^* & = (5, 0, 5, 0) \\
				z^* & = -10.
			\end{align*}
		\end{enumerate}
	\end{enumerate}
\end{document}