\documentclass{homework}

\title{Homework 9}
\author{Kevin Evans}
\studentid{11571810}
\date{November 2, 2021}
\setclass{Math}{364}
\usepackage{amssymb}
\usepackage{mathtools}
\usepackage{graphicx}
\usepackage{amsthm}
\usepackage{amsmath}
\usepackage{slashed}
\usepackage{boldline}
\usepackage{physics}
\usepackage{tcolorbox}
\usepackage[inter-unit-product =\cdot]{siunitx}

\usepackage[makeroom]{cancel}
\usepackage{booktabs}
\usepackage{multirow}

\usepackage{times}
\usepackage{mhchem}
\usepackage{enumitem}
\usepackage[normalem]{ulem}
\usepackage{systeme}
\usepackage{tikz}
\usepackage{mathtools}
\usepackage{tabularx}

%\usepackage{calligra}
%\DeclareMathAlphabet{\mathcalligra}{T1}{calligra}{m}{n}
%\DeclareFontShape{T1}{calligra}{m}{n}{<->s*[2.2]callig15}{}
%\newcommand{\scriptr}{\mathcalligra{r}\,}
%\newcommand{\boldscriptr}{\pmb{\mathcalligra{r}}\,}
%\newcommand{\emf}{\mathcal{E}}
\newcommand{\st}{\mathrm{s.t.}}
\newcommand{\solution}{	\vspace{1em} \textit{Solution.} \quad }
\newcommand{\bolditem}[1][YYY]{\item[\textbf{#1}]}
\newenvironment{amatrix}[1]{%
	\left(\begin{array}{@{}*{#1}{c}|c@{}}
	}{%
	\end{array}\right)
}

%\renewcommand\arraystretch{1.25}


\begin{document}
	\maketitle
	\begin{enumerate}
		\bolditem[5.1.8] A bakery, using flour and sugar, makes cakes, and pastries. Requirements and profits for making and selling a unit of each are as follows: \begin{center}
				\begin{tabular}{lccc}
					\toprule
					& Flour (lb) & Sugar (lb) & Profit (\$) \\
					\midrule
					Cake & 10 & 15 & 40 \\
					Pastry & 3 & 2 & 9 \\
					\bottomrule
				\end{tabular}
			\end{center}
			The bakery has available $b_1$ lb flour and $b_2$ lb of sugar. Assuming that all items can be sold, express the maximum profit attainable as a function of the ratio of $b_1$ to $b_2$.
			
			\solution We can let the decision variables be the number of cakes and pastries to make \begin{align*}
				\text{Let } x_1 & = \text{\# cakes to make,} \\
					x_2 & = \text{\# pastries to make}.
			\end{align*}
			The objective function is the profit to maximize \begin{align*}
				\text{Profit } z & = 40 x_1 + 9 x_2.
			\end{align*}
			The constraints are given by the flour and sugar available, \begin{align*}
				10 x_1 + 3 x_2 \le b_1 && \text{(flour)} \\
				15 x_1 + 2 x_2 \le b_2 && \text{(sugar)}
			\end{align*}
			The linear program is given by \begin{tcolorbox}
				\textit{Primal Problem} \begin{align*}
					\max \quad & z = 40 x_1 + 9 x_2 \\
					\st \quad & 10 x_1 + 3 x_2 \le b_1 \\
							&	15 x_1 + 2 x_2 \le b_2 \\
							& x \ge 0 \\
							& x \in \mathbb{R}^2
				\end{align*}
			\end{tcolorbox}
		
			Using the conversion table from in-class, the dual problem is \begin{tcolorbox}
				\textit{Dual Problem} \begin{align*}
					\min \quad & w = b_1 y_1 + b_2 y_2 \\
					\st \quad & 10y_1 + 15 y_2 \ge 40 \\
						& 3 y_1 + 2 y_2 \ge 9 \\
						& y \ge 0 \\
						& y \in \mathbb{R}^2
				\end{align*}
			\end{tcolorbox}
			
			Sorta following what we did in class, we can let $s=b_1/b_2$ and using the dual problem, we can express the maximum profit as a function of $s$, \begin{center}
				\begin{tabular}{lccc}
					\toprule 
					Condition $s$ & $ s< 10/15$ & $10/15 < s < 3/2$ & $3/2 > s$ \\
					Optimal Point & $(4, 0)$ & $(11/5, 6/5) $ & $(2, 5/2)$ \\
					Optimal Objective & $4b_1$ & $11b_1 / 5 + 6b_2/5$ & $2b_1 + 5b_2/2$ \\
					\bottomrule
				\end{tabular}
			\end{center}
		
		\bolditem[5.3.3] Starting from the final tableau of Table 5.5, complete the problem of (5.3.1) if the objective function coefficient of \begin{enumerate}
			\item $x_3$ is increased from 1 to 4.
				
				\solution As $c_3^* = 2$ and we're increasing $x_3$ by $3$, the coefficient becomes negative. Using the tableau in Table 5.5, the new tableau becomes \begin{align*}
					\begin{array}{cccccc|cc}
						x_1 & x_2 & x_3 & x_4 & x_5 & x_6 & &  \\
						\hline
						-2 & 0 & \boxed{5} & 1 & 2 & -1 & 6 & \boxed{x_4} \\
						11 & 1 & -18 & 0 & -7 & 4 & 4 & x_2 \\
						\hline
						3 & 0 & \boxed{-1} & 0 & 2 & 1 & 106 & z \\
						\midrule
						-2/5 & 0 & 1 & 1/5 & 2/5 & -1/5 & 6/5 & x_3 \\
						19/5 & 1 & 0 & 18/5 & 1/5 & 2/5 & 128/5 & x_2 \\
						\hline
						13/5 & 0 & 0 & 1/5 & 12/5 & 4/5 & 536/5 & z \\
						\bottomrule
					\end{array}
				\end{align*}
				As all the coefficients are non-negative, this is now optimal at the point  $$\boxed{x^* = (0, 128/5, 6/5, 0), z^* = 536/5.}$$
				
			\item $x_4$ is increased from 15 to 16\textonehalf.
			
				\solution As $x_4$ was a basic variable, we're also now changing $c_B$, affecting $r$ and $z$. This results in the tableau \begin{align*}
					\begin{array}{cccccc|cc}
						x_1 & x_2 & x_3 & x_4 & x_5 & x_6 & & \\
						\hline
						-2 & 0 & 5 & \boxed{1} & 2 & -1 & 6 & \boxed{x_4} \\
						11 & 1 & -18 & 0 & -7 & 4 & 4 & x_2 \\
						\hline
						3 & 0 & 2 & \boxed{-3/2} & 2 & 1 & 115 & z \\
						\midrule
						-2 & 0 & 5 & 1 & 2 & -1 & 6 & x_4 \\
						11 & 1 & -18 & 0 & -7 & \boxed{4} & 4 & \boxed{x_2} \\
						\hline
						0 & 0 & 19/2 & 0 & 5 & \boxed{-1/2} & 124 & z \\
						\midrule
						3/4 & 1/4 & 1/2 & 1 & 1/4 & 0 & 7 & x_4\\
						11/4 & 1/4 & -18/4 & 0 & -7/4 & 1 & 1 & x_6 \\
						\hline
						11/8 & 1/8 & 29/4 & 0 & 33/8 & 0 & 249/2 & z
					\end{array}
				\end{align*}
				\sout{ This is optimal at the point $$\boxed{x^* = (0, 0, 0, 7), z^* = 249/2.}$$ } 
				
				I messed this up in the original $z$ value on the RHS in the tableau. It's off by 9. So the optimal value is actually
				$$\boxed{x^* = (0, 0, 0, 7), z^* = 231/2.}$$				
			\item $x_4$ is decreased from 15 to 14 and the coefficient of $x_3$ is decreased from 1 to -2.
			
%			\solution If these values are decreased, we're still in the optimal strategy. Only the objective value is changed, so $$\boxed{x^* = (0, 4, 0, 6), z^*= 84.}$$

			\solution We'll need to enforce $x_4$ being in the basis and leave it out of the objective row in the tableau. The new tableau will be \begin{align*}
				\begin{array}{cccccc|cc}
					x_1 & x_2 & x_3 & x_4 & x_5 & x_6 & & \\
					\hline
					-2 & 0 & 5 & 1 & 2 & -1 & 6 & x_4 \\
					11 & 1 & -18 & 0 & -7 & 4 & 4 & x_2 \\
					\hline
					3 & 0 & 5 & \boxed{1} & 2 & 1 & 106 & z \\
					\midrule
					-2 & 0 & 5 & 1 & 2 & -1 & 6 & x_4 \\
					11 & 1 & -18 & 0 & -7 & 4 & 4 & x_2 \\					
					\hline
					5 & 0 & 0 & 0 & 0 & 2 & 100 & z
				\end{array}
			\end{align*}
			The new optimal point is $$\boxed{x^*=(0, 4, 0, 6), z^*=100.}$$
		\end{enumerate}
		
		
		\bolditem[5.5.2] Consider the linear program of Example 3.5.1 on page 87. Determine the maximum value of the objective function and a point at which this value is attained if \begin{enumerate}
			\item $b_2$ is increased from 10 to 30 units, $b_1$ and $b_3$ remain unchanged.
				
				\solution The basic variables are the slack variables. From the final tableau, the submatrix of these variables is \begin{align*}
					A_B^{-1} & = \begin{pmatrix}
						1/5 & -2/5 & 0 \\
						1/5 & 3/5 & 0 \\
						0 & 2 & 1
					\end{pmatrix}.
				\end{align*}
				The change in the objective value is, where $c_B$ are the basic variable coefficients in the final tableau, \begin{align*}
					\delta z & = c_B A_B^{-1} (\delta b) = c_B A_B^{-1} \begin{pmatrix}
						0 & 20 & 0
					\end{pmatrix}^T \\
					& = -20. \\
					-z^* & = -90.
				\end{align*}
				
			\item $b_1, b_2$, and $b_3$ are each decreased by 10 units from their original values.
				
				\solution Using the same method as (a), \begin{align*}
					\delta b^* & = A_B^{-1} \begin{pmatrix}
						-10 \\ -10 \\ -10
					\end{pmatrix} = \begin{pmatrix}
						2 \\ -8 \\ -30	
					\end{pmatrix} \\
					\delta z & = c_B (\delta b^*) = \begin{pmatrix}
						-2 & -3 & 0
					\end{pmatrix} \begin{pmatrix}
					-2 & -8 & -30
				\end{pmatrix}^T = 20 \\
				-z^* & = 50.
				\end{align*}
		\end{enumerate}
	
		\pagebreak
		
		\bolditem[5.6.7] The aluminum can company of Example 5.1.3 on page 166 has just signed a contract calling for the delivery of an additional 1,800 cases of the Type A can per month (with all other data as stated in the original example). Determine the revised optimal operating schedule and monthly costs, and the new marginal costs for the constraints.
		
		\solution The increase in requirements will affect the first constraint, so $b_1$ will increase by 1800.
			
			The modified $b^*$ can be determined using the existing tableau, \begin{align*}
				b^* & = A_B^{-1} b = A_B^{-1} \begin{pmatrix}
					2400 \\ 2800 \\ 600
				\end{pmatrix} + A_B^{-1} \begin{pmatrix}
					1800 \\ 0 \\ 0
			\end{pmatrix} \\
			& = \begin{pmatrix}
				75 \\ 150 \\ 350
			\end{pmatrix}
			+ \begin{pmatrix}
				3/16 & 0 & -5/8 \\
				-1/8 & 0 & 3/4 \\
				3/8 & -1 & 15/4
			\end{pmatrix} \begin{pmatrix}
			1800 \\ 0 \\ 0
		\end{pmatrix} \\
		& = \begin{pmatrix}
			825/2 \\ -75 \\ 1025
		\end{pmatrix}
			\end{align*}
			These values result in $z^*=-50625$. The modified tableau with these values is \begin{align*}
				\begin{array}{ccccc|ccc|cc}
					\toprule
					9/8 & 1 & 0 & -3/16 & 0 & 3/16 & 0 & -5/8 & 825/2 & x_2 \\
					\boxed{-3/4} & 0 & 1 & 1/8 & 0 & -1/8 & 0 & 3/4 & \boxed{-75} & x_3 \\
					-23/4 & 0 & 0 & -3/8 & 1 & 3/8 & -1 & 15/4 & 1025 & x_5 \\
					\hline
					\boxed{185/4} & 0 & 0 & 25/8 & 0 & -25/8 & 0 & -225/4 & -50625 & -z \\
					\midrule
					0 & 1 & 3/2 & 0 & 0 & 5/16 & 0 & 1/2 & 300 & x_2 \\
					1 & 0 & -4/3 & -1/6 & 0 & 1/6 & 0 & -1 & 100 & x_1 \\
					0 & 0 & -23/3 & -4/3 & 1 & 4/3 & -1 & -2 & 1600 & x_5 \\
					\hline
					0 & 0 & 185/3 & 65/6 & 0 & -65/6 & 0 & -10 & -55250 & -z \\
					\bottomrule
				\end{array}
			\end{align*}
			(I'm not entirely sure if I can end the dual simplex method here, since there are still negatives in the last row. )
			
			The new operating schedule is running all Process 3 for 66.6 hours and the monthly cost is \$55,250. The new marginal costs are \$65.6/case of Type A and \$10/lb of using recycled aluminum. 
			
	\end{enumerate}
\end{document}