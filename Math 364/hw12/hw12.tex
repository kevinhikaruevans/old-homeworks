\documentclass{homework}

\title{Homework 12}
\author{Kevin Evans}
\studentid{11571810}
\date{December 9, 2021}
\setclass{Math}{364}
\usepackage{amssymb}
\usepackage{mathtools}
\usepackage{graphicx}
\usepackage{amsthm}
\usepackage{amsmath}
\usepackage{slashed}
\usepackage{boldline}
\usepackage{physics}
\usepackage{tcolorbox}
\usepackage[inter-unit-product =\cdot]{siunitx}

\usepackage[makeroom]{cancel}
\usepackage{booktabs}
\usepackage{multirow}

\usepackage{times}
\usepackage{mhchem}
\usepackage{enumitem}
\usepackage[normalem]{ulem}
\usepackage{systeme}
\usepackage{tikz}
\usepackage{mathtools}
\usepackage{tabularx}

%\usepackage{calligra}
%\DeclareMathAlphabet{\mathcalligra}{T1}{calligra}{m}{n}
%\DeclareFontShape{T1}{calligra}{m}{n}{<->s*[2.2]callig15}{}
%\newcommand{\scriptr}{\mathcalligra{r}\,}
%\newcommand{\boldscriptr}{\pmb{\mathcalligra{r}}\,}
%\newcommand{\emf}{\mathcal{E}}
\newcommand{\st}{\mathrm{s.t.}}
\newcommand{\solution}{	\vspace{1em} \textit{Solution.} \quad }
\newcommand{\bolditem}[1][YYY]{\item[\textbf{#1}]}
\newenvironment{amatrix}[1]{%
	\left(\begin{array}{@{}*{#1}{c}|c@{}}
	}{%
	\end{array}\right)
}

%\renewcommand\arraystretch{1.25}


\begin{document}
	\maketitle
	\begin{enumerate}
		\bolditem[9.1.1] Determine the payoff matrices for the following two-person zero-sum game. \begin{enumerate}
			\item[(d)] $P_1$ selects a number $n$ from $\{1, 2, 3\}$, and $P_2$ is given two guesses. ($P_2$'s guesses must be from $\{1, 2, 3\}$ but need not be distinct.) After $P_2$ makes her two guesses, $P_1$ reveals his selected number $n$. If $P_2$ did not guess $n$, $P_1$ wins $2n$ from $P_2$; if $P_2$ did guess $n$, $P_2$ wins from $P_1$ an amount equal to $P_2$'s guess.
			
			\solution Considering the distinct permutations, the payoff matrix will look like:
			
			$$\begin{array}{l|rrr}
				  &  (1, 2) & (2, 3)  & (3, 1) \\
				\hline
				1 & -1 & 2 & -1\\
				2 & -2 & -2 & -4 \\
				3 & -6 & -3 & -3
			\end{array}$$
		\end{enumerate}
		
		\bolditem[9.2.1] Find strategy pairs that satisfy Principles I and II for the games with the following payoff matrices: \begin{enumerate}
			\item $\begin{bmatrix}
				3 & 1 & 2 \\
				1 & 0 & 5
			\end{bmatrix}$
		
			\solution Player 1's security level is maximized in row 1. Player 2's security level is maximized with column 2. The corresponding strategy pair is $\boxed{(s_1, t_2).}$
			
			
			\item $\begin{bmatrix}
				7 & 1 & 5 & 9 \\
				1 & 0 & 3 & 2 \\
				6 & 3 & 6 & 4
			\end{bmatrix}$
			
			\solution Similarly, $\boxed{(s_2, t_2).}$
		\end{enumerate}
		
		\bolditem[9.3.2] For each of the following payoff matrices, determine the set of values of $x$ for which game has a saddle point, and for $x$ in this set, determine the saddle point. \begin{enumerate}
			\item[(c)] $\begin{bmatrix}
				x & 1 \\
				3 & x
			\end{bmatrix}$
			
			\solution For Player 1, the minimum of the rows are $\min(1, x)$ and $\min(3, x)$. For Player 2, the maximum of the columns are $\max(x, 3)$ and $\max(1, x)$. 
			
			We can iterate over the first few integers (not sure if this is the best way to do this...),
				$$\begin{array}{cccc}
					\toprule
					x & u_1 & u_2 & \text{Saddle?}\\
					\midrule
					0 & 0 & 1 & \text{no} \\
					1 & 1 & 1 & \text{yes} \\
					2 & 2 & 2 & \text{yes} \\
					3 & 3 & 3 & \text{yes} \\
					4 & 3 & 4 & \text{no} \\
					\bottomrule
				\end{array}$$
			For a saddle point to occur, $x \in \{1, 2, 3\}$ and the saddle point occurs on the $x$.
		\end{enumerate}
		
		\bolditem[9.4.2] For the matrix game $A$, $$A = \begin{bmatrix}
			-1 & 1 & 2 & 0 \\
			4 & -2 & -3 & 2 \\
			0 & 3 & 1 & -2
		\end{bmatrix}$$
		
		\begin{enumerate}
			\item Compute $P_1$'s security level for $X_1=(2/3, 1/3, 0)$ and $X_2=(1/3, 1/3, 1/3)$.
			
			\solution Player 1's security level is given by \begin{align*}
				u_1^{(1)} & = \min_{Y \in T}  \begin{pmatrix}
					2/3 \\ 1/3 \\ 0
				\end{pmatrix}^T A \begin{pmatrix}
				y_1 \\
				y_2 \\
				y_3 \\
				y_4
			\end{pmatrix} \\
				& = \min_{Y\in T} \begin{pmatrix}
					2/3 & 0 & 1/3 & 2/3
				\end{pmatrix} \begin{pmatrix}
				y_1 \\
				y_2 \\
				y_3 \\
				y_4
			\end{pmatrix}
			\intertext{This is minimized for $Y=(0, 1, 0, 0)$ with $\boxed{u_1^{(1)}=0.}$ For $X_2=(1/3, 1/3, 1/3)$, we find}
			u_1^{(2)} & = \min_{Y\in T} \begin{pmatrix}
				1 & 2/3 &0 & 0
			\end{pmatrix} \begin{pmatrix}
				y_1 \\
				y_2 \\
				y_3 \\
				y_4
			\end{pmatrix}
			\intertext{This is minimized with either $Y=(0, 0, 1, 0)$, $Y=(0, 0, 0, 1)$, or a linear combination, with $\boxed{u_1^{(2)}=0.}$}
			\end{align*}
			
			\item Compute $P_2$'s security level for $Y_1=(1/4, 1/4, 1/4, 1/4)$ and $Y_2=(0, 1/2, 0, 1/2)$.
			
			\solution Player 2's security level can be found with \begin{align*}
				u_2^{(1)} & = \max_{X \in S} \begin{pmatrix}
					x_1 & x_2 & x_3
				\end{pmatrix} \begin{pmatrix}
				1/2 \\ 1/4 \\ 1/2
			\end{pmatrix}
				\intertext{This is maximized for either $X=(1, 0, 0)$, $X=(0, 0, 1)$ or a linear combination of these two with $\boxed{u_2^{(1)} = 1/2.}$ For the other mixed strategy, the security level is}
				u_2^{(2)} & = \max_{X \in S} \begin{pmatrix}
					x_1 & x_2 & x_3
				\end{pmatrix} \begin{pmatrix}
					1/2 \\ 0 \\ 1/2
				\end{pmatrix}
				\intertext{This is maximized in the same way with $\boxed{u_2^{(2)} = 1/2.}$}
			\end{align*}
			
			\item What can you now conclude about $v_1$ and $v_2$?
			
			\solution We can say that $v_1 \ge 0$ and $v_2 \le 1/2$, or 
			$$\boxed{0 \le v_1 \le v_2 \le 1/2.}$$
		\end{enumerate}
		
	\end{enumerate}
\end{document}