\documentclass{homework}

\title{Homework 2}
\author{Kevin Evans}
\studentid{11571810}
\date{September 7, 2021}
\setclass{Math}{364}
\usepackage{amssymb}
\usepackage{mathtools}
\usepackage{graphicx}
\usepackage{amsthm}
\usepackage{amsmath}
\usepackage{slashed}
\usepackage{boldline}
\usepackage{physics}
\usepackage{tcolorbox}
\usepackage[inter-unit-product =\cdot]{siunitx}

\usepackage[makeroom]{cancel}
\usepackage{booktabs}
\usepackage{multirow}

\usepackage{times}
\usepackage{mhchem}

%\usepackage{calligra}
%\DeclareMathAlphabet{\mathcalligra}{T1}{calligra}{m}{n}
%\DeclareFontShape{T1}{calligra}{m}{n}{<->s*[2.2]callig15}{}
%\newcommand{\scriptr}{\mathcalligra{r}\,}
%\newcommand{\boldscriptr}{\pmb{\mathcalligra{r}}\,}
%\newcommand{\emf}{\mathcal{E}}
\newcommand{\st}{\mathrm{s.t.}}

\begin{document}
	\maketitle
	\begin{enumerate}
		\item[2.3.15] \underline{Problem}.\quad Using carnations and roses, a florist can make up to three different floral arrangements for the Mother's Day trade. The composition (number of flowers of each type) and selling price (\$)  of a single arrangement of each type are as follows:
		
		\begin{table}[h]
			\centering
			\begin{tabular}{lccc}
				\toprule
				& \textit{Carnations} & \textit{Roses} & \textit{Price (\$)} \\
				\midrule
				Type A & 5 & 2 & 2.75 \\
				Type B & 12 & 4 & 6.50 \\
				Type C & 3 & 6 & 5.25 \\
				\bottomrule
			\end{tabular}
		\end{table}
		
		The florist can purchase from a local wholesaler up to 85 doz carnations at \$1.80/doz and up to 75 roses at \$4.80/doz. The florist can also purchase up to an additional 65 doz carnations at \$3/doz from a distant dealer. Assuming that all arrangements made can be sold, how many of each type should the florist make to maximize income?
		
		\textit{Solution.} \quad First, we can identify the decision variables. It seems like the obvious choices would be: \begin{align*}
			\text{Let } x_k & = \text{the number of type $k$ arrangements made, where } k \in \{A, B, C\} \\
				\ell_C & = \text{dozens of carnations bought locally} \\
				\ell_R & = \text{dozens of roses bought locally} \\
				r_C & = \text{dozens of carnations bought from the distant dealer}.
		\end{align*}
		Next, we can identify the objective function. Here, we'll be trying to maximize the income, which is determined by the number of arrangements sold and the costs of flowers,
		\begin{align*}
			\text{Income } z & = \text{profits} - \text{costs} \\
				& = 2.75 x_a + 6.50 x_b + 5.25 x_c - 1.80 \ell_C - 4.80 \ell_R - 3 r_C.
		\end{align*}
		The constraints are given by the compositions of each arrangement and the purchase limits per dozen of flowers: \begin{align*}
			x_a & = \frac{1}{12} \left[5\left(\ell_C + r_C\right) + 2 \ell_R  \right] \\
			x_b & = \frac{1}{12} \left[12\left(\ell_C + r_C\right) + 4 \ell_R  \right] \\
			x_c & = \frac{1}{12} \left[3\left(\ell_C + r_C\right) + 6 \ell_R  \right] \\
			0 & \le \ell_C \le 85 \\
			0 & \le  \ell_R  \le 75 \\
			0 & \le r_C  \le 65 \\
			& x_k, \ell_C, \ell_R, r_C  \in \mathbb{Z}^+.
		\end{align*}
	
		\pagebreak
		
		The linear program in the standard notation is 
		
		\begin{tcolorbox}
			\vspace{-1em}
			\begin{align*}
				\max\quad  z & = 2.75 x_a + 6.50 x_b + 5.25 x_c - 1.80 \ell_C - 4.80 \ell_R - 3 r_C \\
				\st \quad	x_a  & = \frac{1}{12} \left[5\left(\ell_C + r_C\right) + 2 \ell_R  \right] \\
				x_b & = \frac{1}{12} \left[12\left(\ell_C + r_C\right) + 4 \ell_R  \right] \\
				x_c & = \frac{1}{12} \left[3\left(\ell_C + r_C\right) + 6 \ell_R  \right] \\
				0 & \le \ell_C \le 85 \\
				0 & \le  \ell_R  \le 75 \\
				0 & \le r_C  \le 65 \\
				& x_k, \ell_C, \ell_R, r_C  \in \mathbb{Z}^+.
			\end{align*}
		\end{tcolorbox}
	
		\item[2.4.6] \underline{Problem.}\quad Two sources supply three destinations with a commodity. Each source has a supply of 80 units and each destination has a demand for 50 units. Shipping costs in dollars per unit are:
		
		\begin{center}
			\begin{tabular}{lcccc}
				\toprule
				& & \multicolumn{3}{c}{Destinations} \\
				& & \textit{1} & \textit2 & \textit3 \\
				\midrule
				\multirow{2}{*}{Sources} & \textit1 & 8 & 17 & 19 \\
				& \textit2 & -- & 21 & 22 \\
				\bottomrule
			\end{tabular}
		\end{center}
		The transportation costs from Source 2 to Destination 1 vary. The first 20 units shipped on this route cost \$10/unit, and each unit over 20 cost \$13/unit. Determine a minimal-cost shipping schedule.
		
		\textit{Solution.}\quad The decision variables can be the unit shipments from source $i$ to destination $j$, represented by a matrix $x$ and an additional variable is needed for the number of units from source 2 to destination 1 over 20, \begin{align*}
			\text{Let } x_{ij} & = \text{number of units shipped from source $i$ to destination $j$} \\
			& \qquad \text{where } i \in \{ 1, 2\}, j \in \{ 1, 2, 3 \} \\
			y_{21} & = \text{number of units shipped $2 \to 1$ at the higher cost}
		\end{align*}
		The objective function is the total cost of the shipping schedule, \begin{align*}
			\text{Cost } z & = c^T x + 13 y_{21}\\
			c^T & = \begin{pmatrix}
				8 & 17 & 19 \\
				10 & 21 & 22
			\end{pmatrix}
		\end{align*}
		Lastly the constraints are given by the source supplies, destination demand, and the $2 \to 1$ limitations, \begin{align*}
			0 \le \sum_j x_{ij} \le 80 & \qquad \text{for } i \in \{ 1, 2\} \\
			 \sum_i x_{ij} = 50 & \qquad \text{for } j \in \{1, 2, 3\} \\
			 0 \le x_{21} \le 20 \\
			 x_{ij}, y_{ij} \in \mathbb{Z}^+
		\end{align*}
	
		The linear program in the standard notation is
		
		\begin{tcolorbox}
			\vspace{-1em}
			\begin{align*}
				\min_{x_{ij}, y_{21} \in \mathbb{Z}^+}z & = c^T x + 13 y_{21} \\
				\text{where } c^T & = \begin{pmatrix}
					8 & 17 & 19 \\
					10 & 21 & 22
				\end{pmatrix} \\ \\
				\st \qquad & 0 \le \sum_j x_{ij} \le 80  \qquad \text{for } i \in \{ 1, 2\} \\
				&  \sum_i x_{ij} = 50  \qquad \text{for } j \in \{1, 2, 3\} \\
				& 0 \le x_{21} \le 20 \\
				& x_{ij}, y_{ij} \in \mathbb{Z}^+
			\end{align*}
		\end{tcolorbox}
		\item[2.5.6] \begin{enumerate}
			\item \underline{Problem.}\quad Suppose the agent in (TK) Problem 5 can also by and sell Commodity B at the following prices per unit:
			
			\begin{center}
				\begin{tabular}{lcc}
					\toprule
					\textit{B} & Buy (\$) & Sell (\$) \\
					\midrule
					\textit{Month 1} & 80 & 95 \\
					\textit{Month 2} & 85 & 110 \\
					\textit{Month 3} & 95 & 125 \\
					\bottomrule
				\end{tabular}
			\end{center}
			
			The dealer can buy at most 200 units of B and sell at most 250 units during any one month and can also store B at the local warehouse, but space is limited. Assume the warehouse has 30 cu. yd of space available at \$2/cu. yard and that a unit of A requires 1 cu. yd and a unit of B requires 2 cu. yd. Again, the dealer has no stock on hand and wants none at the end of the 3 months. Determine an optimal buying, selling, and storing program utilizing both commodities. 
			
			\vspace{1em}
			
			\textit{Solution.}\quad Let's begin by first looking at the decision variables. For each month $k$ and for commodity $i$, we'll let \begin{align*}
				b_{ik} & = \text{commodity $i$ bought in month $k$} \\
				s_{ik} & = \text{commodity $i$ sold in month $k$} \\
				w_{ik} & = \text{commodity $i$ stored in month $k$}, \\
				& \text{where } i \in \{ A, B\}, k \in \{1, 2, 3\}.
			\end{align*}
			The objective function is the income generated by both commodities, \begin{align*}
				\text{Income } z & = \text{selling income} - \text{purchasing costs} - \text{storage costs} \\
					& = 40 s_{A1} + 44 s_{A2} + 48 s_{A3} \\
					& \quad + 95 s_{B1} + 110 s_{B2} + 125 s_{B3} \\
					& \quad - 31 b_{A1} - 33 b_{A2} - 36 b_{A3} \\
					& \quad - 80 b_{B1} - 85 b_{B2} - 95 b_{B3} \\
					& \quad - 2 w_{A1} - 2 w_{A2} - 4 w_{B1} - 4 w_{B2}
			\end{align*}
		
			The constraints are given by the commodity flow,
			\begin{align*}
				\text{prev. in storage} + \text{bought} & = \text{sold} + \text{stored} \\
				w_{ik-1} + b_{ik} & = s_{ik} + w_{ik} \\
				\intertext{For $i \in \{A, B\}$,}
				\cancel{w_{i0}} + b_{i1} & = s_{i1} + w_{i1} \\
				w_{i1} + b_{i2} & = s_{i2} + w_{i2} \\
				w_{i2} + b_{i3} & = s_{i3} + \cancel{w_{i3}}.
			\end{align*}
			And the per-month constraints, for month $k \in \{1, 2, 3\}$, \begin{align*}
				0 \le b_{Ak} \le 450 && 0 \le b_{Bk} \le 200 \\
				0 \le s_{Ak} \le 600 && 0 \le s_{Bk} \le 250 \\
				0 \le w_{Ak} + 2 w_{Bk} \le 30. && 
			\end{align*}
		
			\begin{tcolorbox}
				\vspace{-1em}
				\begin{align*}
					\max z & = c^T x \\
					\text{where } x^T & = \begin{pmatrix}
						b_{A1}  & b_{A2} & b_{A3} & s_{A1} & s_{A2} & s_{A3} & w_{A1} & w_{A2} \\
						b_{B1}  & b_{B2} & b_{B3} & s_{B1} & s_{B2} & s_{B3} & w_{B1} & w_{B2}
					\end{pmatrix} \\
				c^T & = \begin{pmatrix}
					-31 & -33 & -36 & 40 & 44 & 48 & -2 & -2 \\
					-80 & -85 & -95 & 95 & 110 & 125 & -4 & -4
				\end{pmatrix} \\
				\st \qquad &	0 \le b_{Ak} \le 450  \qquad 0 \le b_{Bk} \le 200 \\
				& 0 \le s_{Ak} \le 600 \qquad 0 \le s_{Bk} \le 250 \\
				& 0 \le w_{Ak} + 2 w_{Bk} \le 30 \\
			 & b_{i1}  = s_{i1} + w_{i1} \\
				& w_{i1} + b_{i2} = s_{i2} + w_{i2} \\
				& w_{i2} + b_{i3}  = s_{i3}
				\end{align*}
				
			\end{tcolorbox}
		
			\pagebreak
			
			\item \underline{Problem.}\quad In the above problem, any units stored represent an investment of capital. Reconsider the problem, assuming that a maximum of \$10,000 can be borrowed each month for this purpose, with an accompanying 2\% interest rate.
			
			
			\textit{Solution.}\footnote{I'm 75\% sure I'm not understanding this part completely. I'm assuming we can both borrow capital \textit{and} use our liquid money to rent storage.}\quad In this scenario, we can add an additional variable for borrowed capital, 
			$$\text{Let } c_k = \text{capital borrowed for storing units in month $k$}.$$
			We can modify the objective function as 
			\begin{align*}
			\text{Income } z & = \text{selling income} + \text{borrowed} - \text{purchasing costs} - \text{storage costs} - \text{interest} \\
				& = 40 s_{A1} + 44 s_{A2} + 48 s_{A3} \\
				& \quad + 95 s_{B1} + 110 s_{B2} + 125 s_{B3} \\
				& \quad + c_1 + c_2 \\
				& \quad - 31 b_{A1} - 33 b_{A2} - 36 b_{A3} \\
				& \quad - 80 b_{B1} - 85 b_{B2} - 95 b_{B3} \\
				& \quad - 2 w_{A1} - 2 w_{A2} - 4 w_{B1} - 4 w_{B2} \\
				& \quad - 1.02 c_{2} - 1.02 c_1.
			\end{align*}
			Additionally, we'll have to add constraints ensuring the borrowed money at the end is zero and so we cannot borrow more than \$10,000 each month: \begin{align*}
				0 \le c_k & \le 10000 \\
				c_3 & = 0.
			\end{align*}
			The modified linear program is then 
			\begin{tcolorbox}
				\vspace{-1em}
				\begin{align*}
					\max z & = c^T x \\
					\text{where } x^T & = \begin{pmatrix}
						b_{A1}  & b_{A2} & b_{A3} & s_{A1} & s_{A2} & s_{A3} & w_{A1} & w_{A2} & c_1 & c_2 \\
						b_{B1}  & b_{B2} & b_{B3} & s_{B1} & s_{B2} & s_{B3} & w_{B1} & w_{B2} & 0 & 0
					\end{pmatrix} \\
					c^T & = \begin{pmatrix}
						-31 & -33 & -36 & 40 & 44 & 48 & -2 & -2 & -0.02 & -0.02\\
						-80 & -85 & -95 & 95 & 110 & 125 & -4 & -4 & 0 & 0
					\end{pmatrix} \\
					\st \qquad &	\text{for $k \in \{1, 2, 3\}$: } \\
					& 0 \le b_{Ak} \le 450  \qquad 0 \le b_{Bk} \le 200 \\
					& 0 \le s_{Ak} \le 600 \qquad 0 \le s_{Bk} \le 250 \\
					& 0 \le w_{Ak} + 2 w_{Bk} \le 30 \\
					& 0 \le c_k \le 10000 \\
					& b_{i1}  = s_{i1} + w_{i1} \\
					& w_{i1} + b_{i2} = s_{i2} + w_{i2} \\
					& w_{i2} + b_{i3}  = s_{i3} \\
					& c_3 = 0
				\end{align*}
				
			\end{tcolorbox}
		\end{enumerate}
	\end{enumerate}
\end{document}