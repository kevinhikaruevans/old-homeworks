\documentclass{homework}

\title{Homework 6}
\author{Kevin Evans}
\studentid{11571810}
\date{October 5, 2021}
\setclass{Math}{364}
\usepackage{amssymb}
\usepackage{mathtools}
\usepackage{graphicx}
\usepackage{amsthm}
\usepackage{amsmath}
\usepackage{slashed}
\usepackage{boldline}
\usepackage{physics}
\usepackage{tcolorbox}
\usepackage[inter-unit-product =\cdot]{siunitx}

\usepackage[makeroom]{cancel}
\usepackage{booktabs}
\usepackage{multirow}

\usepackage{times}
\usepackage{mhchem}
\usepackage{enumitem}
\usepackage[normalem]{ulem}
\usepackage{systeme}
\usepackage{tikz}
\usepackage{mathtools}
%\usepackage{calligra}
%\DeclareMathAlphabet{\mathcalligra}{T1}{calligra}{m}{n}
%\DeclareFontShape{T1}{calligra}{m}{n}{<->s*[2.2]callig15}{}
%\newcommand{\scriptr}{\mathcalligra{r}\,}
%\newcommand{\boldscriptr}{\pmb{\mathcalligra{r}}\,}
%\newcommand{\emf}{\mathcal{E}}
\newcommand{\st}{\mathrm{s.t.}}
\newcommand{\solution}{	\vspace{1em} \textit{Solution.} \quad }
\newcommand{\bolditem}[1][YYY]{\item[\textbf{#1}]}
\newenvironment{amatrix}[1]{%
	\left(\begin{array}{@{}*{#1}{c}|c@{}}
	}{%
	\end{array}\right)
}
\renewcommand\arraystretch{1.25}



\begin{document}
	\maketitle
	\begin{enumerate}
		\bolditem[3.5.9] Compute the solution to Problem 7 of Section 2.6. A poultry producer has 112 sq. rods of land on which to raise during the next 12-week period chickens, ducks, and turkeys. The space and labor requirements and the profit---excluding labor costs---from the sale after the 12-week breeding period are as follows: \begin{center}
			\begin{tabular}{lccc}
				\toprule
				& Space (sq rod/unit) & Labor (hr/week/unit) & Profit (\$/unit) \\
				\midrule
				Chickens & 1.2 & 3 & 260 \\
				Ducks & 1.0 & 2 & 172 \\
				Turkeys & 0.8 & 1 & 88 \\ \bottomrule
			\end{tabular}
		\end{center}
		The producer has available each week 200 hr of labor at \$13/hr and up to 45 hr of overtime at \$18/hr. What stock should the producer raise over the 12-week period in order to maximize net income (profits less labor costs)?
		
		\solution First, we'll need to bring this to the standard format. The decision variables are: \begin{align*}
			x_1 & = \text{number of chickens to raise} \\
			x_2 & = \text{number of ducks to raise} \\
			x_3 & = \text{number of turkeys to raise} \\
			x_4 & = \text{hours of overtime to use}.
		\end{align*}
		The constraints are first the non-overtime labor/wk is less than 245 hours ($200 + 45$ hours), and the overtime labor is less than 45 hours/wk, \begin{align*}
			0 \le 3 x_1 + 2 x_2 +  x_3 \le 245 \\
			0 \le x_4 \le 45.
		\end{align*}
		The objective function is the net income, \begin{align*}
			\text{net income} & = \text{profits} - \text{labor} \\
			z	& = 260 x_1 + 172 x_2 + 88 x_3 - 12 \times 13 (3 x_1 + 2 x_2 +  x_3) - 12 \times (18-13) x_4 \\
				& = -208x_1 - 140x_2 - 68x_3 - 60x_4.
		\end{align*}
		We can convert this to standard form by adding slack variables, then flipping the sign of the objective function and converting this to a minimization problem, \begin{align*}
			\min  \quad & z = 208 x_1 + 140 x_2 + 68 x_3 + 60 x_4 \\
			\st \quad & 3 x_1 + 2 x_2 + x_3 + x_5 = 245 \\
			& x_4 + x_6 = 45 \\
			& x \ge 0 \\
			& x \in \mathbb{Z}^6
		\end{align*}
		\fbox{\parbox{\linewidth}{ The solution is to produce no livestock, as the labor costs more than the profit. There will always be a net loss in income.}}
		
		
	
		\bolditem[3.6.3] Using a combination of birdseed mixtures $A$, $B$, and $C$, a blend of minimum cost which is at least 20\% thistle and 30\% corn is desired. Given the data which follow, determine the percentage of each of the mixtures in the final blend. \begin{center}
			\begin{tabular}{lccc}
				\toprule
				& \% Thistle & \% Corn & Cost (cents/lb) \\
				\midrule
				A & 25 & 40 & 57 \\
				B & 0 & 30 & 13 \\
				C & 10 & 15 & 20 \\
				\bottomrule
			\end{tabular}
		\end{center}
	
		\solution The decision variables are the amount of each mixture in the final blend. For ease, we'll be doing this as a batch of 100 lbs. \begin{align*}
			\text{Let} \quad x_1 & = \text{lbs of mixture A} \\
				x_2 & = \text{lbs of mixture B} \\
				x_3 & = \text{lbs of mixture C}.
		\end{align*}
		The constraints are given by the thistle and corn requirements, as well as the 100 lb constraint, \begin{align*}
			0.25 x_1 + 0 x_2 + 0.10 x_3 & \ge 20 \\
			0.40 x_1 + 0.30 x_2 + 0.15 x_3 & \ge 30 \\
			x_1 + x_2 + x_3 & = 100.
		\end{align*}
		
		The objective function is the cost in cents to minimize, \begin{align*}
			\text{Cost } z & = 57 x_1 + 13 x_2 + 20 x_3.
		\end{align*}
	
		In standard form, the linear program is \begin{align*}
			\min z  & = 57 x_1 + 13 x_2 + 20 x_3 \\
			\st \quad & 0.25 x_1 + 0 x_2 + 0.10 x_3 - x_4 = 20 \\
			& 0.40 x_1 + 0.30 x_2 + 0.15 x_3 - x_5 = 30 \\
			& x_1 + x_2 + x_3 = 100 \\
			& x \ge 0 \\
			& x \in \mathbb{R}^5
		\end{align*}
		We'll have to add three artificial variables (I think) here, 
		\begin{center}
			\sysdelim..\systeme{
				0.25 x_1 + 0.10 x_3 - x_4 + x_6 = 20,
				0.40 x_1 + 0.30 x_2 + 0.15 x_3 - x_5 + x_7 = 30,
				x_1 + x_2 + x_3 + x_8 = 100,
				57 x_1 + 13 x_2 + 20 x_3 = z,
				x_6 + x_7 + x_8 = w
			}
		\end{center} 
		
		The tableaux with the artificial variables is \[
			\begin{array}{ccccc|ccc|cl}
				x_1 & x_2 & x_3 & x_4 & x_5 & x_6 & x_7 & x_8 & & \\
				\hline
				0.25 & 0 & 0.10 & -1 & 0 & 1 & 0 & 0 & 20 & =x_6 \\
				0.40 & 0.30 & 0.15 & 0 & -1 & 0 & 1 & 0 & 30 & =x_7\\
				1 & 1 & 1 & 0 & 0 & 0 & 0 & 1 & 100 & = x_8\\
				\hline
				57 & 13 & 20 & 0 & 0 & 0 & 0 & 0 & 0 & = -z \\
				0 &  0 & 0 & 0 & 0 & 1 & 1 & 1 & 0 & = -w \\
				\hline \hline
				0.25 & 0 & 0.10 & -1 & 0 & 1 & 0 & 0 & 20 & =x_6 \\
				\boxed{0.40} & 0.30 & 0.15 & 0 & -1 & 0 & 1 & 0 & 30 & =\boxed{x_7}\\
				1 & 1 & 1 & 0 & 0 & 0 & 0 & 1 & 100 & = x_8\\
				\hline
				57 & 13 & 20 & 0 & 0 & 0 & 0 & 0 & 0 & = -z \\
				\boxed{-1.65} & -1.30 & -1.25 & 1 & 1 & 0 & 0 & 0 & 0 & = -w \\
				\hline \hline
				0 & -0.1875 & 0.00625 & -1 & 0.625 & 1 & -0.625 & 0 & 1.25 & = x_6\\
				1 & \boxed{0.75} & 0.375 & 0 & -2.5 & 0 & 2.5 & 0 & 75 & = \boxed{x_1} \\
				0 & 0.25 & 0.625 & 0 & 2.5 & 0 & -2.5 & 1 & 25 & =x_8 \\
				\hline
				0 & -29.75 & -1.375 & 0 & 142.5 & 0 & -142.5 & 0 & -4275 \\
				0 & \boxed{-0.0625} & -0.63125 & 1 & -3.125 & 0 & 4.125 & 0 & 123.75 \\
				\hline \hline
				0.25 & 0 & 0.1 & -1 & 0 & 1 & 0 & 0 & 20 & = x_6\\
				4/3 & 1 & \boxed{0.5} & 0 & -10/3 & 0 & 10/3 & 0 & 100 & = \boxed{x_2}\\
				-1/3 & 0 & 0.5 & 0 & 10/3 & 0 & -10/3 & 1 & 100 & = x_8\\
				\hline
				119/3 & 0 & 13.5 & 0 & 130/3 & 0 & -130/3 &  0 & -1300 \\
				1/12 & 0 & \boxed{-0.6} & 1 & -10/3 & 0 & 13/3 & 0 & 130 \\
				\hline \hline
				-1/60 & -0.2 & 0 & -1 & \boxed{2/3} & 1 & -2/3 & 0 & 0 & = \boxed{x_6}\\
				8/3 & 2 & 1 & 0 & -20/3 & 0 & 20/3 & 0 & 200 & = x_3\\
				-5/3 & -1 & 0 & 0 & 20/37  & 0 & 20/3 & 1 & 0 & = x_8\\
				\hline
				20/3 & -27 & 0 & 0 & 400/3 & 0 & -400/3 & 0 & -4000 \\
				101/60 & 1.2 & 0 & 1 & \boxed{-22/3} & 0 & 25/3 & 0 & 250 \\
				\hline \hline
			\end{array}
		\]
		I think I've made a mistake somewhere, because I'm going in a circular loop after this last tableau. 
%		The initial tableau thing with artificial variables is \[
%			\begin{array}{ccccc|ccc|c}
%				x_1 & x_2 & x_3 & x_4 & x_5 & x_6 & x_7 & x_8 & \\
%				\hline
%				0.25 & 0 & 0.10 & -1 & 0 & 1 & 0 & 0 & 20 \\
%				0.40 & 0.30 & 0.15 & 0 & -1 & 0 & 1 & 0 & 30 \\
%				1 & 1 & 1 & 0 & 0 & 0 & 0 & 1 & 100 \\
%				\hline
%				57 & 13 & 20 & 0 & 0 & 0 & 0 & 0 & -z \\
%				0 &  0 & 0 & 0 & 0 & 1 & 1 & 1 & -w \\
%				\hline \hline
%				0.25 & 0 & 0.10 & -1 & 0 & 1 & 0 & 0 & 20 \\
%				0.40 & 0.30 & 0.15 & 0 & -1 & 0 & 1 & 0 & 30 \\
%				1 & 1 & 1 & 0 & 0 & 0 & 0 & 1 & 100 \\
%				\hline
%				57 & 13 & 20 & 0 & 0 & 0 & 0 & 0 & -z \\
%				-1.65 & -1.30 & -1.25 & 1 & 1 & 0 & 0 & 0 & -w \\
%				\hline
%			\end{array}\]
%			If we take $x_1$ in the $w$ row and perform the ratio test, we'll see the first row gives the minimum ratio of $80$. Therefore, we can pivot on $x_6$ in the first row. This process is repeated for $x_2$ and $x_4$.
%			\[\begin{array}{ccccc|ccc|c}
%				x_1 & x_2 & x_3 & x_4 & x_5 & x_6 & x_7 & x_8 & \\
%				\hline
%				1 & 0 & 0.4 & -4 & 0 & 4 & 0 & 0 & 80 \\
%				0 & 0 & -0.01 & 1.6 & -1 & 0 & 1 & 0 & 22 \\
%				0 & 1 & 0.6 & -4 & 0 & -1 & 0 & 1 & 80 \\
%				\hline
%				0 & 13 & -2.8 & 57 & 0 & -57 & 0 & 0 & -z \\
%				0 & -1.30 & -0.59 & -5.6 & 1 & 6.6 & 0 & 0 & 132 - w \\
%				\hline \hline
%				1 & 0 & 0.4 & -4 & 0 & 4 & 0 & 0 & 80 \\
%				0 & 0 & -0.01 & 1.6 & -1 & 0 & 1 & 0 & 22 \\
%				0 & 1 & 0.6 & -4 & 0 & -1 & 0 & 1 & 80 \\
%				\hline
%				0 & 0 & -10.6 & 109 & 0 & -44 & 0 & -13 & -1040 - z \\
%				0 & 0 & 0.19 & -10.8 & 1 & 5.3&  0 & 1.3 & 236 - w \\
%				\hline \hline
%				1 & 0 & 0.375 & 0 & -2.5 & 4 & 2.5 & 0 & 135 \\
%				0 & 0 & -0.00625 & 1 & -0.625 & 0 & 0.625 & 0 & 13.75 \\
%				0 & 1 & 0.575 & 0 & -2.5 & -1 & 2.5 & 1 & 135 \\
%				\hline
%				0 & 0 & -9.91875 & 0 & 68.125 & -44 & -68.125 & -13 & -2538.75 - z \\
%				0 & 0 & 0.1225 & 0 & -5.75 & 5.3 & 6.75 & 1.3 & 384.5 - w \\
%				\hline \hline
%			\end{array}\]
	%% pp 98
	%% https://tex.stackexchange.com/a/35235/182762
	
		\pagebreak
		\bolditem[3.7.3] \begin{enumerate}
			\item[(b)] Determine the optimal value of the objective function, an optimal solution point, and whether or not the system of constraints contains any redundancies. 
			
			Maximize $5x_1 + 3 x_2 + 3 x_3$
			
			subject to $\begin{array}[t]{rrrl}
				2 x_1 ~ + & x_2 ~ + & x_3 & = 12 \\
				3 x_1 ~ + & x_2 ~ + & 2 x_3 & = 18 \\
			\end{array}$
		
			\solution Putting this in standard form with a couple artificial variables, then sticking it into a tableau,
			\[ 
				\begin{array}{ccc|cc|cl}
					x_1 & x_2 & x_3 & x_4 & x_5 & & \\
					\hline
					\boxed{2} & 1 & 3 & 1 & 0 & 12 & = \boxed{x_4} \\
					3 & 1 & 2 & 0 & 1 & 18 & = x_5 \\
					\hline
					-5 & -3 & -3 & 0 & 0 & 0 & = -z \\
					\boxed{-5} & -2 & -5 & 0 & 0 & 0 & = -w \\
					\hline \hline
					1 & 0.5 & 1.5 & 0.5 & 0 & 6 & = x_1 \\
					0 & -0.5& -2.5& -1.5& 1& 0 & = x_5 \\
					\hline
					0 & -0.5& 4.5& 2.5& 0& 30 \\
					0 & 0.5 & 2.5& 2.5& 0& 30 \\
					\hline \hline
				\end{array}
			\]
			
			Again, I think I've made a mistake somewhere. I've optimized $w$ and am expecting to get $w=0$, but clearly that is not the case. The book says this indicates there is no solution. However, I'm definitely expecting a solution and I can't figure out where I've went wrong.
		\end{enumerate}
	\end{enumerate}
\end{document}