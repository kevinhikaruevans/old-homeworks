\documentclass{homework}

\title{Homework 4}
\author{Kevin Evans}
\studentid{11571810}
\date{September 21, 2021}
\setclass{Math}{364}
\usepackage{amssymb}
\usepackage{mathtools}
\usepackage{graphicx}
\usepackage{amsthm}
\usepackage{amsmath}
\usepackage{slashed}
\usepackage{boldline}
\usepackage{physics}
\usepackage{tcolorbox}
\usepackage[inter-unit-product =\cdot]{siunitx}

\usepackage[makeroom]{cancel}
\usepackage{booktabs}
\usepackage{multirow}

\usepackage{times}
\usepackage{mhchem}
\usepackage{enumitem}
\usepackage[normalem]{ulem}


%\usepackage{calligra}
%\DeclareMathAlphabet{\mathcalligra}{T1}{calligra}{m}{n}
%\DeclareFontShape{T1}{calligra}{m}{n}{<->s*[2.2]callig15}{}
%\newcommand{\scriptr}{\mathcalligra{r}\,}
%\newcommand{\boldscriptr}{\pmb{\mathcalligra{r}}\,}
%\newcommand{\emf}{\mathcal{E}}
\newcommand{\st}{\mathrm{s.t.}}
\newcommand{\solution}{	\vspace{1em} \textit{Solution.} \quad }
\newcommand{\bolditem}[1][YYY]{\item[\textbf{#1}]}
\newenvironment{amatrix}[1]{%
	\left(\begin{array}{@{}*{#1}{c}|c@{}}
	}{%
	\end{array}\right)
}
\begin{document}
	\maketitle
	\begin{enumerate}
		\bolditem[3.1.3] % (f) and (g)
			Put the following problems into standard form.
			
			\begin{enumerate}
				\item[(f)] Maximize $x_1 + 2x_2 + 4x_3$
				
				subject to
				
				$\abs{4x_1 + 3x_2 - 7x_3}  \le x_1 + x_2 + x_3$
				
				$x \ge 0$
				
				\vspace{1em}
				\textit{Solution.} \quad First, we can change the objective from a maximization to a minimization and flipping the sign,
				$$\min_x - x_1 -  2x_2 - 4x_3.$$
				
				Next, the absolute value constraint can be split into two constraints,
				\begin{align*}
					4x_1 + 3x_2 - 7x_3  & \le x_1 + x_2 + x_3 \\
					-4x_1 + -3x_2 + 7x_3  & \le x_1 + x_2 + x_3.
					\intertext{Then, we can add in slack variables $x_4$ and $x_5$, convert these into equalities,}
					4x_1 + 3x_2 - 7x_3  + x_4 & = x_1 + x_2 + x_3 \\
					-4x_1 + -3x_2 + 7x_3 + x_5 & = x_1 + x_2 + x_3.
					\intertext{Subtracting out the extraneous terms,}
					3x_1 + 2x_2 - 8x_3 + x_4  & = 0 \\
					-5x_1 + -4x_2 + 6x_3  + x_5 & = 0.
				\end{align*}
				The linear program in standard form is
				
				\begin{tcolorbox}
					\vspace{-1em}
					\begin{align*}
						\min_x \quad & -x_1 - 2x_2 - 4x_3 \\
						\st \quad & 3x_1 + 2x_2 - 8x_3 + x_4  = 0 \\
						& -5x_1 + -4x_2 + 6x_3  + x_5  = 0 \\
						& x \ge 0
					\end{align*}
				\end{tcolorbox}
			
				\pagebreak
				
				\item[(g)] Maximize $x_1 + 6x_2 + 12 x_3$
				
					subject to
					
					$-x_1 - x_2 + x_4 \ge \text{maximum of } 7 x_1 + 2x_2 \text{ and } 5x_2 + x_3 + x_4$
					
					$x \ge 0$
					
					\vspace{1em}
					
					\textit{Solution.} \quad We can convert the maximization into a minimization by flipping signs again,
					$$\min_x -x_1 - 6x_2 - 12 x_3.$$
					
					The ``maximum of...'' constraint can be split into two separate constraints,
					\begin{align*}
						-x_1 - x_2 + x_4 & \ge 7 x_1 + 2x_2 \\
						-x_1 - x_2 + x_4 & \ge 5x_2 + x_3 + x_4.
						\intertext{Adding in slack variables, we can convert this into equalities,}
						-x_1 - x_2 + x_4 + x_5 & = 7 x_1 + 2x_2 \\
						-x_1 - x_2 + x_4 + x_6 & = 5x_2 + x_3 + x_4.
						\intertext{Making one side a constant, the equality constraints are}
						-8x_1 - 3x_2 + x_4 + x_5 & = 0 \\
						-x_1 - 6x_2  - x_3  + x_6 & = 0.
					\end{align*}
					
					The linear program in standard form is
					\begin{tcolorbox}
						\vspace{-1em}
						\begin{align*}
							\min_x \quad & -x_1 - 6x_2 - 12 x_3 \\
							\st \quad & -8x_1 - 3x_2 + x_4 + x_5  = 0 \\
								& -x_1 - 6x_2  - x_3  + x_6 = 0 \\
								& x \ge 0
						\end{align*}
					\end{tcolorbox}
			\end{enumerate}
		
		\pagebreak
		
		\bolditem[3.2.3] A system of equations is said to be \textit{inconsistent} if the system has no solution. Show by using the pivot operation that the following systems are inconsistent. Is either of these systems equivalent to a system in canonical form?
		
		\begin{enumerate}
			\item $x_1 + 2x_2 = 3$
			
				$x_1 + 2x_2 = 4$
				
				\solution	Putting this into matrix form, it's clear that because the coefficients of $x_1$ and $x_2$ are equal, there is no pivot operation to separate out a single variable,
				\begin{align*}
					\begin{amatrix}{2}
						1 & 2 & 3 \\
						1 & 2 & 4
					\end{amatrix}.
				\end{align*}
				Because the coefficients are equal, the last row will be something like $(0, 0, -1)$, so this system is not solvable. \sout{This system is also not in canonical form, as if we take either $x_1$ or $x_2$ as the basic variable, there is no feasible solution.} I'm not sure, since a canonical form requires an objective function.
				
				
			\item $\begin{aligned}[t]
				x_1 + x_2  -  3x_3  & = 7 \\
				 -2x_1  + x_2  +  5x_3 & = 2 \\
				 3x_2 - x_3 & = 15
			\end{aligned}$
		
			
			\solution In matrix form, this becomes \begin{align*}
				\begin{amatrix}{3}
					1 & 1 & -3 & 7 \\
					-2 & 1 & 5 & 2 \\
					0 & 3 & -1 & 15
				\end{amatrix} & \to \begin{amatrix}{3}
				1 & 1 & -3 & 7 \\
				0 & 3 & -1 & 16 \\
				0 & 3 & -1 & 15
			\end{amatrix} \\
			\to \begin{amatrix}{3}
				1 & 1 & -3 & 7 \\
				0 & 3 & -1 & 16 \\
				0 & 3 & -1 & 15
			\end{amatrix}
			\end{align*}
			Because row 2 and row 3 have equal coefficients, the last row will be something like $(0, 0, 0, 1)$, which means this system is not solvable. \sout{I think this could be in canonical form, if we take $x_3$ as the non-basic variable and set it to $0$, we would have a basic solution.} Can this even be in canonical form if there is no objective function $z$?
		\end{enumerate}
		
	\end{enumerate}
\end{document}