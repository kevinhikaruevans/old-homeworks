\documentclass{homework}

\title{Homework 11}
\author{Kevin Evans}
\studentid{11571810}
\date{November 30, 2021}
\setclass{Math}{364}
\usepackage{amssymb}
\usepackage{mathtools}
\usepackage{graphicx}
\usepackage{amsthm}
\usepackage{amsmath}
\usepackage{slashed}
\usepackage{boldline}
\usepackage{physics}
\usepackage{tcolorbox}
\usepackage[inter-unit-product =\cdot]{siunitx}

\usepackage[makeroom]{cancel}
\usepackage{booktabs}
\usepackage{multirow}

\usepackage{times}
\usepackage{mhchem}
\usepackage{enumitem}
\usepackage[normalem]{ulem}
\usepackage{systeme}
\usepackage{tikz}
\usepackage{mathtools}
\usepackage{tabularx}

%\usepackage{calligra}
%\DeclareMathAlphabet{\mathcalligra}{T1}{calligra}{m}{n}
%\DeclareFontShape{T1}{calligra}{m}{n}{<->s*[2.2]callig15}{}
%\newcommand{\scriptr}{\mathcalligra{r}\,}
%\newcommand{\boldscriptr}{\pmb{\mathcalligra{r}}\,}
%\newcommand{\emf}{\mathcal{E}}
\newcommand{\st}{\mathrm{s.t.}}
\newcommand{\solution}{	\vspace{1em} \textit{Solution.} \quad }
\newcommand{\bolditem}[1][YYY]{\item[\textbf{#1}]}
\newenvironment{amatrix}[1]{%
	\left(\begin{array}{@{}*{#1}{c}|c@{}}
	}{%
	\end{array}\right)
}

%\renewcommand\arraystretch{1.25}


\begin{document}
	\maketitle
	\begin{enumerate}
		\bolditem[6.3.2] Solve using the cutting plane algorithm. \begin{enumerate}
			\item[(b)] 
			 Maximize $3x_1 + 8x_2$ 

				subject to
				
				$\begin{matrix}
					x_1 & + & 2x_2 & \le 9 \\
					& & 2x_2 & \le 5
				\end{matrix}$
			
				
				$x \ge 0, x \in \mathbb{Z}^2$
				
				% solution for relaxation is (4, 2.5)
				\solution Putting this in a tableau and applying the simplex method to solve the LP relaxation, \begin{align*}
					\begin{array}{cc|cc|cc}
						x_1 & x_2 & x_3 & x_4 & \\
						\hline
						\boxed{1} & 2 & 1 & 0 & 9 & \boxed{x_3} \\
						0 & 2 & 0 & 1 & 5 & x_4 \\
						\hline
						\boxed{3} & 8 & 0 & 0 & 0 & -z \\
						\midrule
						1 & 2 & 1 & 0 & 9 & x_1 \\
						0 & \boxed{2} & 0 & 1 & 5 & \boxed{x_4} \\
						\hline
						0 & \boxed{2} & -3 & 0 & -27 & -z \\
						\midrule
						1 & 0 & 1 & -1 & 4 & x_1 \\
						0 & 1 & 0 & 1/2 & 5/2 & x_2 \\
						\hline
						0 & 0 & -3 & -1 & -32 & -z\\
						\midrule
					\end{array}
				\end{align*}			
				The LP relaxation is optimal at $x^* = (4, 5/2)$, $z^* = 32$. The second constraint has a non-integral coefficient, \begin{align*}
					x_2 + (1/2)x_4 & = 5/2.
					\intertext{Rearranging this to integral and nonintegral parts,}
					x_2 + (1/2) x_4 & = 2 + 1/2 \\
					(1/2) x_4 - 1/2 & = -x_2 + 2. \\
					\implies (1/2) x_4 - 1/2 & \ge 0 \quad (\in \mathbb{Z})
					\intertext{We can add a new constraint,}
					\Aboxed{ (-1/2)x_4 + x_5 & = -1/2. }
				\end{align*}
				The modified tableau is  \begin{align*}
					\begin{array}{cc|ccc|cc}
						x_1 & x_2 & x_3 & x_4 & x_5 & \\
						\hline
						1 & 0 & 1 & -1 & 0 & 4 & x_1 \\
						0 & 1 & 0 & 1/2 & 0 & 5/2  & x_2 \\
						0 & 0 & 0 & \boxed{-1/2} & 1 &  \boxed{-1/2} & x_5 \\
						\hline
						0 & 0 & -3 & -1 & 0 & -32 & -z\\
						\midrule
						1 & 0 & 1 & 0 & -2 & 5 & x_1 \\
						0 & 1 & 0 & 0 & 1 & 2 & x_2 \\
						0 & 0 & 0 & 1 & -2 & 1 & x_4\\
						\hline
						0 & 0 & -3 & 0 & -2 & -31 & -z\\
						\bottomrule
					\end{array}
				\end{align*}
				The optimal solution to the IP is $\bar{x} = (5, 2)$, $\bar{z} = 31$.
		\end{enumerate}
	
		\bolditem[6.4.2] Solve using the branch and bound approach. \begin{enumerate}
			\item[(a)] 
				Maximize $4x_1 + 5x_2 + 3x_3$ 
				
				subject to
				
				$\begin{matrix*}[r]
					3x_1 & - & 2x_2 & + & x_3 & \le & 15 \\
					x_1 & + & 2x_2 & + & x_3 & \le & 8
				\end{matrix*}$
			
				$x \le 0, x \in \mathbb{Z}^3$
				
				\solution  Using a computer solver, the LP relaxation $(1)$ solution is $x^{(1)}=(5.75, 1.125, 0)$, $z^{(1)}=28.625$.
				\begin{enumerate}
					\item $\begin{aligned}[t]
						L & = \{(1)\} \\
						B & = \varnothing \\
						\bar{z} & = -\infty
					\end{aligned}$
					
					Looking at $(1)$, we can split the solution on $x^{(1)}_1 = 5.75$, we can add \begin{align}
						x_1^{(2)} & \le 5 \tag{2} \\
						x_1^{(3)} & \ge 6 \tag{3}.
					\end{align}
				
					\item  $\begin{aligned}[t]
						L & = \{(2), (3)\} \\
						B & = \varnothing \\
						\bar{z} & = -\infty
					\end{aligned}$
				
					Looking at $(2)$, $x^{(2)} = (5, 0.75, 1.5)$, $z^{(2)} = 28.25$. We'll add constraints \begin{align}
						x_2^{(4)} & = 0 \tag{4} \\
						x_2^{(5)} & \ge 1 \tag{5}.
					\end{align}
				
					\item $\begin{aligned}[t]
						L & = \{(3), (4), (5)\} \\
						B & = \varnothing \\
						\bar{z} & = -\infty
					\end{aligned}$
				
					Looking at (5), $x^{(5)} = (5, 1, 1)$ and $z^{(5)} = 28$. This is integer and we can add it into $B$ and set $\bar{z}$.
					
					\item  $\begin{aligned}[t]
						L & = \{(3), (4)\} \\
						B & = \{(5)\} \\
						\bar{z} & = 28
					\end{aligned}$
					3 is infeasible and can just be removed from the list.
					
					\item $\begin{aligned}[t]
						L & = \{(4)\} \\
						B & = \{(5)\} \\
						\bar{z} & = 28
					\end{aligned}$
				
					Looking at (4), we'll have $x^(4) = (3.5, 0, 4.5)$ and $z^{(4)} = 27.5$. Since the objective value is worse than (5), we can ignore this one.
					
					\item  $\begin{aligned}[t]
						L & = \varnothing\\
						B & = \{(5)\} \\
						\bar{z} & = 28
					\end{aligned}$
					
					The algorithm is now finished and we're left with an optimal solution from (5): \begin{align*}
						\bar{x} & = (5, 1, 1) \\
						\bar{z} & = 28
					\end{align*}
				\end{enumerate}
		\end{enumerate}
	\end{enumerate}
\end{document}