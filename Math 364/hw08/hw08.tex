\documentclass{homework}

\title{Homework 8}
\author{Kevin Evans}
\studentid{11571810}
\date{October 19, 2021}
\setclass{Math}{364}
\usepackage{amssymb}
\usepackage{mathtools}
\usepackage{graphicx}
\usepackage{amsthm}
\usepackage{amsmath}
\usepackage{slashed}
\usepackage{boldline}
\usepackage{physics}
\usepackage{tcolorbox}
\usepackage[inter-unit-product =\cdot]{siunitx}

\usepackage[makeroom]{cancel}
\usepackage{booktabs}
\usepackage{multirow}

\usepackage{times}
\usepackage{mhchem}
\usepackage{enumitem}
\usepackage[normalem]{ulem}
\usepackage{systeme}
\usepackage{tikz}
\usepackage{mathtools}
\usepackage{tabularx}

%\usepackage{calligra}
%\DeclareMathAlphabet{\mathcalligra}{T1}{calligra}{m}{n}
%\DeclareFontShape{T1}{calligra}{m}{n}{<->s*[2.2]callig15}{}
%\newcommand{\scriptr}{\mathcalligra{r}\,}
%\newcommand{\boldscriptr}{\pmb{\mathcalligra{r}}\,}
%\newcommand{\emf}{\mathcal{E}}
\newcommand{\st}{\mathrm{s.t.}}
\newcommand{\solution}{	\vspace{1em} \textit{Solution.} \quad }
\newcommand{\bolditem}[1][YYY]{\item[\textbf{#1}]}
\newenvironment{amatrix}[1]{%
	\left(\begin{array}{@{}*{#1}{c}|c@{}}
	}{%
	\end{array}\right)
}

%\renewcommand\arraystretch{1.25}


\begin{document}
	\maketitle
	\begin{enumerate}
		\bolditem[4.2.1] Determine the dual of each of the following linear programming problems.
		\begin{enumerate}
			\item[(d)] Minimize $6x_1 + 12x_2 - 18x_3$
			
				subject to
				
				\sysdelim..\systeme{
					x_1 - 3x_2 + 6x_3 = 30,
					2x_1 + 8x_2 - 16 x_3 = 70
				}
			
				$x_1, x_2 \ge 0, x_3$ unrestricted
				
				\solution Using the conversion table from in-class, we have a maximization problem using the $B$'s from the original matrix. Then we can transpose the constraints matrix $A$ and convert the variables to constraints,
				
				Maximize $30 y_1 + 7 y_2$
				
				subject to
				
				$
				\begin{array}{rrrrr}
					y_1 & + & 2 y_2 & \le & 6 \\
					-3y_1 & + & 8 y_2 & \le& 12 \\
					6 y_1 & - & 16 y_2 & = &-18
				\end{array}
				$
				
				$y_1, y_2 \: \mathrm{urs}$
				
			\item[(e)] Maximize $x_1 - 7x_2 + 3 x_3$ 
			
				subject to 
				
				$\begin{array}{rrrrrr}
					& & 2 x_2 & + & 5x_3 & = 20 \\
					8x_1 & & & - & 3x_3 & = 40 \\
					& & x_2 & + & 4 x_3 & \ge 60 
				\end{array}$
				
				$x_1, x_3 \ge 0, x_2$ unrestricted
				
				\solution Again, using the conversion table from in-class, we have 
				
				Minimize $20y_1 + 40 y_2 + 60 y_3$
				
				subject to
				
				$\begin{array}{rrrrrrr}
					 &  & 8 y_2 &  & & \ge & 1 \\
					2 y_1 &  &  & + & 1 y_3 & = & -7 \\
					5 y_1 & - & 3 y_2 & + & 4 y_3 & \ge & 3
				\end{array}$
			
				$y_1, y_2$ unrestricted, $y_3 \le 0$
		\end{enumerate}

	\pagebreak
	
		\bolditem[4.4.6] Consider the problem of \begin{align*}
			\min \quad & z = 13 x_1 + 15 x_2 + 12 x_3 + 8 x_4 \\
			\st \quad & \begin{array}[t]{rrrrrrrrr}
				4 x_1 & + & 8 x_2 & - & 5 x_3 & + & 3 x_4 & = & 32 \\
				3 x_1 & - & 2 x_2 & + & 6 x_3 & - & x_4 & \ge & 3 \\
			\end{array} \\
			& x \ge 0
		\end{align*}
		\begin{enumerate}
			\item Determine which of the following points are feasible solutions to this min problem: $(9, 0, 2, 2)$, $(4, 1, -1, 1)$, $(5, 1, 1, 3)$.
			
			\solution Checking against the three constraints, \begin{itemize}
				\item $(9, 0, 2, 2)$ is feasible as it abides by the constraints;
				\item $(4, 1, -1, 1)$ is not feasible, as $x \ge 0$;
				\item $(5, 1, 1, 3)$ is feasible as it abides by the constraints.
			\end{itemize}
		
			\item Evaluate the function $z$ at those points in part (a) that are feasible solutions to the problem.
			
			\solution For the two feasible points found in (a), the objective function value is\begin{itemize}
				\item $z(9, 0, 2, 2) = 157$,
				\item $z(5, 1, 1, 3) = 116$.
			\end{itemize}
			
			\item Write out the dual to the min problem.
			
			\solution Using the conversion table from class, \begin{align*}
				\max \quad & w = 32 y_1 + 3 y_2 \\
					\st \quad & 4 y_1 + 3 y_2 \le 13 \\
						& 8 y_1 - 2 y_2 \le 15 \\
						& -5y_1 + 6 y_2 \le 12 \\
						& 3 y_1 - 1 y_2 \le 8 \\
						& y_1 \text{ urs}, y_2 \ge 0
			\end{align*}
			
			\item Determine which of the following points are feasible solutions to this dual problem: $(-1, 1)$, $(0, 2)$, $(1, 3)$.
			
			\solution \begin{itemize}
				\item $(-1, 1)$ is feasible;
				\item $(0, 2)$ is feasible;
				\item $(1, 3)$ is not feasible (fails the third constraint).
			\end{itemize}
			
			\item Evaluate the dual objective function at those points in part (d) that are feasible solutions to the problem.
			
			\solution \begin{itemize}
				\item $w(-1, 1) = -32 + 3 = -29$;
				\item $w(0, 2) = 6$.
			\end{itemize}
		
			\item Using only the information above, what can you say about the minimum value of $z$?
			
				\solution It is somewhere between $-29$ and $116$ (by Theorem 4.4.1).
		\end{enumerate}
		
		\pagebreak
		
		\bolditem[4.5.2] Consider the linear program \begin{align*}
			\max \quad & 2x_1 + 2x_2  \\
			\st \quad & \begin{array}[t]{rrrrrrrr}
				x_1 &   &      & + & x_3 & + & x_4 & \le 1 \\
					&   & x_2  & + & x_3 & - & x_4 & \le 1 \\
				x_1 & + & x_2 & + & 2x_3 &   &     & \le 3
			\end{array} \\
			& x \ge 0
		\end{align*}
		\begin{enumerate}
			\item Determine the dual problem.
			
				\solution Using the conversion table from in-class, the dual of the problem is \begin{align*}
					\min \quad & 1 y_1 + 1 y_2 + 3 y_3 \\
					\st \quad &  \begin{array}[t]{rrrrrrr}
						y_1 &  &     & + & y_3 & \ge 2 \\
						    &   & y_2 & + & y_3 & \ge 2 \\
						y_1 & + & y_2 & + & 2 y_3 & \ge 0 \\
						y_1 & - & y_2 &   &       & \ge 0
					\end{array} \\
%					 y_1 + y_3 \ge 2 \\
%						& y_2 + y_3 \ge 2 \\
%						& y_1 + y_2 + 2 y_3 \ge 0 \\
%						& y_1 - y_2 \ge 0 \\
						& y \ge 0
				\end{align*}
			
			\item Show that $X^* = (1, 1, 0, 0)$ and $Y^* = (1, 1, 1)$ are feasible solutions to the original and dual problems, respectively. 
			
				\solution For $X^*=(1, 1, 0, 0)$, \begin{align*}
					1 + 0 + 0 & \le 1 && \checkmark \\
					1 + 0 - 0 & \le 1 && \checkmark \\
					1 + 1 + 2(0) & \le 3 && \checkmark 
					\intertext{And for $Y^*=(1,1,1)$,}
					1 + 1 & \ge 2 && \checkmark \\
					1 + 1 & \ge 2 && \checkmark \\
					1 + 1 + 2 & \ge 2 && \checkmark \\
					1 - 1 & \ge 0 && \checkmark
				\end{align*}
			
			\item Show that for this pair of solutions, for each $j$, $x_j^* > 0$ implies that the slack in the corresponding dual constraint is zero.
			
			\solution Looking at the indices $j=1, 2$, $x_j^* = 1, 1$, the corresponding slacks for $Y^*=(1, 1, 1)$ are:  $2 - (1 + 1) = 0$ and $2 - (1 + 1) = 0$, respectively. 
			
			\item Show that $Y^*$ is not an optimal solution to the dual.
			
			\solution For $X^* = (1, 1, 0, 0)$, the corresponding slack is $(0, 0, 1)$. However, since the last index of $Y^*=(1, 1, 1)$ is not zero, this solution cannot be optimal. 
			
			\item Does this contradict the Complementary Slack Theorem?
			
			\solution No, because both the slack in the primal and dual constraints must either be zero. In this case, we only have it in one direction, i.e. $X^*$ is an optimal solution to the dual, but $Y^*$ is not.
		\end{enumerate}
	\end{enumerate}
\end{document}