\documentclass{homework}

\title{Homework 7}
\author{Kevin Evans}
\studentid{11571810}
\date{October 11, 2021}
\setclass{Math}{364}
\usepackage{amssymb}
\usepackage{mathtools}
\usepackage{graphicx}
\usepackage{amsthm}
\usepackage{amsmath}
\usepackage{slashed}
\usepackage{boldline}
\usepackage{physics}
\usepackage{tcolorbox}
\usepackage[inter-unit-product =\cdot]{siunitx}

\usepackage[makeroom]{cancel}
\usepackage{booktabs}
\usepackage{multirow}

\usepackage{times}
\usepackage{mhchem}
\usepackage{enumitem}
\usepackage[normalem]{ulem}
\usepackage{systeme}
\usepackage{tikz}
\usepackage{mathtools}
\usepackage{tabularx}

%\usepackage{calligra}
%\DeclareMathAlphabet{\mathcalligra}{T1}{calligra}{m}{n}
%\DeclareFontShape{T1}{calligra}{m}{n}{<->s*[2.2]callig15}{}
%\newcommand{\scriptr}{\mathcalligra{r}\,}
%\newcommand{\boldscriptr}{\pmb{\mathcalligra{r}}\,}
%\newcommand{\emf}{\mathcal{E}}
\newcommand{\st}{\mathrm{s.t.}}
\newcommand{\solution}{	\vspace{1em} \textit{Solution.} \quad }
\newcommand{\bolditem}[1][YYY]{\item[\textbf{#1}]}
\newenvironment{amatrix}[1]{%
	\left(\begin{array}{@{}*{#1}{c}|c@{}}
	}{%
	\end{array}\right)
}
%\renewcommand\arraystretch{1.25}


\begin{document}
	\maketitle
	\begin{enumerate}
		\bolditem[2.6.15] Model, but do not solve. A company has plants $P_1$, $P_2$, and $P_3$ which produce sandwiches needed at assembly centers $C_1$, $C_2$, $C_3$, and $C_4$. The annual output capacities of the plants and demands at the assembly centers are:
		
		\vspace{1em}
		
		\begin{minipage}{0.5\textwidth}
			\centering
			\begin{tabular}{ccc}
				\toprule
				\multicolumn{3}{c}{Annual Output of Sandwiches} \\
				$P_1$ & $P_2$ & $P_3$ \\
				\midrule
				10,500 & 18,800 & 13,200 \\
				\bottomrule
			\end{tabular}
		\end{minipage}
		\begin{minipage}{0.5\textwidth}
			\centering
			\begin{tabular}{cccc}
				\toprule
				\multicolumn{4}{c}{Annual Demand of Sandwiches} \\
				$C_1$ & $C_2$ & $C_3$ & $C_4$ \\
				\midrule
				7,700 & 9,900 & 12,200 & 11,100 \\
				\bottomrule
			\end{tabular}
		\end{minipage}
	
		\vspace{1em} 
		
		The units can be delivered from the plants to the centers either by truck or rail. However, for each route for which units are delivered by rail, there is an annual route lease fee, independent of the number of units shipped through the route. The data follow.
		
		\begin{center}
			\begin{tabular}{rcccccc}
				\toprule 
				& & & \multicolumn{4}{c}{To Center} \\
				\cmidrule{4-7}
				& From Plant & By & $C_1$ & $C_2$ & $C_3$ & $C_4$ \\
				\cmidrule{2-7}
				\multirow{6}{*}{ \shortstack{Delivery Cost \\ \small (\$/unit)}} & \multirow{2}{*}{$P_1$} & Truck & 60 & 80 & 50 & 30 \\
					& & Rail & 35 & 62 & 22 & 25 \\
					\cmidrule{2-7}
					&  \multirow{2}{*}{$P_2$} & Truck & 95 & 120 & 65 & 75 \\
					& & Rail & 75 & 89 & 45 & 55 \\
					\cmidrule{2-7}
					& \multirow{2}{*}{$P_3$} & Truck & 110 & 98 & 77 & 88 \\
					& & Rail & 53 & 35 & 32 & 38 \\
					\bottomrule
			\end{tabular}
		
			\begin{tabular}{rccccc}
				\toprule
				& & \multicolumn{4}{c}{To Center} \\
				\cmidrule{3-6}
				& From Plant & $C_1$ & $C_2$ & $C_3$ & $C_4$ \\
				\cmidrule{2-6}
				\multirow{3}{*}{\shortstack{Route Lease Fee \\ \small (in \$1000)}} & $P_1$ & 165 & 220 & 175 & 150 \\
					& $P_2$ & 250 & 200 & 220 & 210 \\
					& $P_3$ & 180 & 190 & 200 & 180 \\
					\bottomrule
			\end{tabular}
		\end{center}
	
		Determine a minimum cost delivery schedule for the next year; that is, for each plant and center, determine how many units are to be shipped from plant to the center by truck, and how many by rail, so that the supplies are not exceeded, demands are met, and total cost is minimized.
		
		\solution Let's begin by determining the decision variables. We'll need to track how many units are shipped for both by truck and by rail, as well as a binary variable to track if one route uses rail. We'll define \begin{align*}
			\mathrm{Let} \quad t_{ij} & = \text{number of units shipped from plant $P_i$ to center $C_j$ by truck,} \\
				r_{ij} & = \text{number of units shipped from plant $P_i$ to center $C_j$ by rail,} \\
				b_{ij} & = \begin{cases*}
					1 & \text{if the shipments from $P_i$ to $C_j$ will be via rail} \\
					0 & \text{otherwise}
				\end{cases*}
		\end{align*}
		
		We can now figure out the objective function. For this problem, we'll be minimizing the total cost \begin{align*}
			\mathrm{Cost} \quad z & = T\cdot t + R \cdot r + B \cdot b, \\
			\mathrm{where} \quad T & = \begin{pmatrix}
				60 & 80 & 50 & 30 \\
				95 & 120 & 65 & 75 \\
				110 & 98 & 77 & 88
			\end{pmatrix}, \\
			R & = \begin{pmatrix}
				35 & 62 & 22 & 25 \\
				75 & 89 & 45 & 55 \\
				53 & 35 & 32 & 38
			\end{pmatrix}, \\
			B & = \begin{pmatrix}
				165 & 220 & 175 & 150 \\
				250 & 200 & 220 & 210 \\
				180 & 190 & 200 & 180
			\end{pmatrix} \times 1000.
		\end{align*}
	
		The constraints will be given by the annual output and demand, where \begin{align*}
			\text{output of plant $i$} & = \text{amount produced by $P_i$} = \sum_{j=1}^4 t_{ij} + r_{ij}, && \text{for $i=1, 2, 3$}\\
			\text{demand by center $j$} & = \text{amount sent to $C_j$} = \sum_{i=1}^3 t_{ij} + r_{ij}. && \text{for $j=1, 2, 3, 4$}
		\end{align*}
		We'll also need a constraint to ensure that we can use the rail only when the fee is charged, 
		\begin{align*}
			r_{ij} \le M \times b_{ij}.
		\end{align*}
		
		Putting this all together, the integer program becomes \begin{tcolorbox}
			\vspace{-1em}
			\begin{align*}
				\min \quad & z = T \cdot t + R \cdot r + B \cdot b \\
					& \text{where the matrices $T$, $R$, $B$ are defined above} \\
				\st \quad & P_i = \sum_{j=1}^4 t_{ij} + r_{ij}, && \text{for $i=1, 2, 3$}\\
					& C_j = \sum_{i=1}^3 t_{ij} + r_{ij}. && \text{for $j=1, 2, 3, 4$} \\
					& r_{ij} \le M b_{ij}\\
					& M \text{ large} \\
					& b_{ij} \in  \{ 0, 1\} \\
					& t_{ij}, r_{ij} \ge 0 \\
					& t_{ij}, r_{ij} \in \mathbb{Z}
			\end{align*}	
		\end{tcolorbox}
%		\bolditem[2.6.15] Model, but do not solve. A firm combines Raw materials A, B, and C in the production of two products. The requirements (in lbs) for the manufacture of a unit of each product are as follows: \begin{center}
%			\begin{tabular}{lrrr}
%				\toprule
%				& \multicolumn{3}{c}{\textit{Raw Material}} \\
%				& A & B & C \\
%				\midrule
%				Product 1 & 4 & 12 & 8 \\
%				Product 2 & 7 & 9 & 10 \\
%				\bottomrule
%			\end{tabular}
%		\end{center}
%		The firm can sell all units produced at the fixed prices of \$5.55/unit for Product 1 and \$6.30/unit for Product 2. The firm purchases the necessary raw materials from outside sources, and the costs which must be considered in determining profits, vary. The firm can purchase: \begin{itemize}
%			\item up to 1 ton of Raw Material A at 20 cents/lb and any amount over 1 ton at 35 cents/lb;
%			\item up to 2 tons  of B at 10 cents/lb and any amount over at 20 cents/lb;
%			\item up to 1-1/2 tons of C at 15 cents/lb and any mount over at 25 cents/lb.
%		\end{itemize}
%		Determine a production schedule that maximizes net profit, assuming that there is an additional overhead cost to the firm of \$1/unit for each unit of Product 1 and Product 2.
%		
%		\solution
%		To start, we can choose the decision variables as \begin{align*}
%			\mathrm{Let} \quad x_1 & = \text{number of units of Product 1 to produce,} \\
%				x_2 & = \text{number of units of Product 2 to produce,} \\
%				x_a & = \text{pounds of Raw Material A to buy at the lower cost,} \\
%				x_b & = \text{pounds of Raw Material B to buy at the lower cost,} \\
%				x_c & = \text{pounds of Raw Material C to buy at the lower cost,} \\
%				x_a' & = \text{pounds of Raw Material A to buy at the higher cost,} \\
%				x_b' & = \text{pounds of Raw Material B to buy at the higher cost,} \\
%				x_c' & = \text{pounds of Raw Material C to buy at the higher cost.}
%		\end{align*}
%		(I think the last few decision variables aren't needed but I think it'll make this easier to do.)
%		
%		The constraints are given by how many pounds we're able to buy the raw materials at the lower cost and the constituents of each product, \[\begin{gathered}
%			x_a + x_a' = 4x_1 + 7 x_2 \\
%			x_b + x_b' = 12 x_1 + 9 x_2 \\
%			x_c + x_c' = 8 x_1 + 10 x_2 \\
%			0 \le x_a \le 2000 \\
%			0 \le x_b \le 4000 \\
%			0 \le x_c \le 3000 \\
%			x_1, x_2, x_a, x_b, x_c, x_a', x_b', x_c' \ge 0.
%		\end{gathered}\]
%	
%		The objective function is the net profit to maximize. The net profit is defined \begin{align*}
%			\mathrm{profit} \quad z& = \mathrm{revenue} - \mathrm{costs} \\
%			z & = 5.55 x_1 + 6.30 x_2 && \text{revenue}\\
%				& \quad - 1 x_1 - 1 x_2 && \text{overhead costs}\\
%				& \quad - 0.20 x_a - 0.10 x_b - 0.15 x_c && \text{material cost (low)} \\
%				& \quad - 0.35 x_a' - 0.20 x_b' - 0.25 x_c. && \text{material costs (high)}
%			\end{align*}
	\end{enumerate}
\end{document}