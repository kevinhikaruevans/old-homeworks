\documentclass{homework}

\title{Homework 9}
\author{Kevin Evans}
\studentid{11571810}
\date{November 8, 2021}
\setclass{Physics}{450}
\usepackage{amssymb}
%\usepackage{mathtools}
\usepackage{graphicx}
\usepackage{amsthm}
\usepackage{amsmath}
\usepackage{slashed}
\usepackage{boldline}
\usepackage{physics}
\usepackage[inter-unit-product =\cdot]{siunitx}

\usepackage[makeroom]{cancel}
\usepackage{booktabs}

\usepackage{times}
\usepackage{mhchem}
\usepackage{mathtools}

%\usepackage{calligra}
%\DeclareMathAlphabet{\mathcalligra}{T1}{calligra}{m}{n}
%\DeclareFontShape{T1}{calligra}{m}{n}{<->s*[2.2]callig15}{}
%\newcommand{\scriptr}{\mathcalligra{r}\,}
%\newcommand{\boldscriptr}{\pmb{\mathcalligra{r}}\,}
%\newcommand{\emf}{\mathcal{E}}

\newcommand{\aplus}{\hat{a}_+}
\newcommand{\aminus}{\hat{a}_-}

\begin{document}
	\maketitle
	\begin{enumerate}
		\item Assuming all the operators are Hermitian, \begin{align*}
%			\braket{ \psi }{\comm{A}{BC} \psi} & = \braket{ \psi }{(ABC - BCA) \psi} \\
		\comm{A}{BC} & = (ABC - BCA + BAC - BAC) \\
			& = \left[
				(AB-BA)C
				+ B(AC-CA)
			\right] \\
			& = B\comm{A}{C} + \comm{A}{B}C. \qed
		\end{align*}
	
		\item From the Hamiltonian relation from in class (October 27), for any non-time-dependent operator $Q$, \begin{align*}
			\sigma_H \sigma_Q & \ge \frac{\hbar}{2} \abs{ \dv{t}\expval{Q} }.
			\intertext{Then using the position operator $x$ for $Q$, }
			\sigma_H \sigma_x & \ge \frac{\hbar}{2}  \abs{\dv{\expval{x}}{t}} \\
				& \ge \frac{\hbar}{2m}  \abs{\expval{p}}. \qed
		\end{align*}
%		I could've used this with the commutator relation, but this way is 
		\item  \begin{enumerate}
			\item % 4.22
				From (4.10), \begin{align*}
					\comm{p_i}{r_j} = -i \hbar \delta_{ij},
				\end{align*}
				Then from (4.96),\begin{align*}
					\comm{L_z}{x} & = \comm{xp_y - yp_x}{x} \\
					& =\comm{xp_y}{x} - \comm{yp_x}{x} = -(-i\hbar y) = i \hbar y. \qed
				\end{align*}
				For the other dimensions,
				\begin{align*}
					\comm{L_z}{y} & = \comm{xp_y - yp_x}{y} = x \comm{p_y}{y} = - i \hbar x. \qed \\
					\comm{L_z}{z} & = 0 \text{, as there's no $p_z$ in the commutator.} \qed
				\end{align*}
				\begin{align*}
					\comm{L_z}{p_x} & = \comm{xp_y - y p_x}{p_x} = p_y\comm{x}{p_x} - p_x\comm{y}{p_x} = i \hbar p_y. \qed \\
					\comm{L_z}{p_y} & = p_y\comm{x}{p_y} - p_x \comm{y}{p_y} = - i \hbar p_x. \qed \\
					\comm{L_z}{p_z} & = 0, \text{ as there's no $z$'s.} \qed
				\end{align*}
			
			\item From (4.96), \begin{align*}
				\comm{L_z}{L_x} & = \comm{L_z}{yp_z - zp_y} = p_z\comm{L_z}{y} - z\comm{L_z}{p_y} \\
					& = -i \hbar x p_z + i z \hbar p_x \\
					& = i \hbar (z p_x - x p_z) = i \hbar L_y. \qed
			\end{align*}
		
			\item For the position commutator and by the identity in Problem 1, \begin{align*}
				\comm{L_z}{r^2} & = \comm{L_z}{x^2 + y^2 + z^2} \\
					& = \comm{L_z}{x}x + x\comm{L_z}{x} + \comm{L_z}{y}y + y\comm{L_z}{y} + 0 \\
					& = i 2\hbar y x - i 2\hbar  y x = 0.
				\intertext{Similarly for the momentum operator,}
				\comm{L_z}{p^2} & = \comm{L_z}{p_x^2 + p_y^2 + p_z^2} \\
					& = \comm{L_z}{p_x}p_x + p_x\comm{L_z}{p_x} + \comm{L_z}{p_y}p_y + p_y\comm{L_z}{p_y} + 0 \\
					& = i \hbar p_y p_x + i \hbar p_x p_y - i \hbar p_x p_y - i \hbar p_y p_x = 0.
			\end{align*}
		
			\item For the Hamiltonian $H=p^2/2m + V$, \begin{align*}
				\comm{L_x}{H} & = \frac{1}{2m} \comm{p^2}{L_x} + \comm{L_x}{V(r)} \\
					& = 0 + 0 \text{, by part (c).}
			\end{align*}
		\end{enumerate}
	
		\item Study Chapter 4.3.
	\end{enumerate}
\end{document}