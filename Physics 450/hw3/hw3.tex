\documentclass{homework}

\title{Homework 2}
\author{Kevin Evans}
\studentid{11571810}
\date{September 8, 2021}
\setclass{Physics}{450}
\usepackage{amssymb}
%\usepackage{mathtools}
\usepackage{graphicx}
\usepackage{amsthm}
\usepackage{amsmath}
\usepackage{slashed}
\usepackage{boldline}
\usepackage{physics}
\usepackage[inter-unit-product =\cdot]{siunitx}

\usepackage[makeroom]{cancel}
\usepackage{booktabs}

\usepackage{times}
\usepackage{mhchem}

%\usepackage{calligra}
%\DeclareMathAlphabet{\mathcalligra}{T1}{calligra}{m}{n}
%\DeclareFontShape{T1}{calligra}{m}{n}{<->s*[2.2]callig15}{}
%\newcommand{\scriptr}{\mathcalligra{r}\,}
%\newcommand{\boldscriptr}{\pmb{\mathcalligra{r}}\,}
%\newcommand{\emf}{\mathcal{E}}

\begin{document}
	\maketitle
	\begin{enumerate}
		\item Read Chapter 2.1, 2.4.
		
		\item From the time-dependent Schr\"odinger equation, \begin{align*}
			i \hbar \pdv{\Psi}{t} & = -\frac{\hbar^2}{2m} \pdv[2]{\Psi}{x} + V\Psi
			\intertext{If we assume a separable solution and plug it into the time-dependent SE,}
			\Psi(x, t) & = \psi(x) \phi(t) \\
			{i \hbar} \psi \pdv{\phi}{t} & = - \frac{\hbar^2}{2m} \phi \pdv[2]{\psi}{x}  + V \phi \psi
			\intertext{Dividing by $\Psi$, we can see the two sides must equal a constant $E$,}
			\frac{i \hbar}{\phi} \pdv{\phi}{t} & = - \frac{\hbar^2}{2m \psi} \pdv[2]{\psi}{x} + V = E \\
			\implies & - \frac{\hbar^2}{2m} \pdv[2]{\psi}{x} + V \psi = E \psi \\
			\implies & i \hbar \pdv{\phi}{t} = E \phi
		\end{align*}
		These equations are solvable by integrating \begin{align*}
			i \hbar \pdv{\phi}{t} & = E \phi \\
			\frac{i \hbar}{\phi} \dd{\phi} & = E  \dd{t} \\
			i \hbar \ln \phi & = E t + k_0 \\
			\phi(t)& = k_1 e^{ -i E t / \hbar }
		\end{align*}
		Assuming $V=0$, the spatial equation is solvable with \begin{align*}
			-\frac{\hbar^2}{2m} \pdv[2]{\psi}{x} & = E \psi
			\intertext{By inspection, this is a sinusoid,}
			\psi(x) & = e^{i \hbar x / \sqrt{2m E}} \\
				& = e^{ikx}, \quad \text{where } k = \frac{\hbar}{\sqrt{2mE}}
		\end{align*}
		Bringing this all together, the wavefunction with both parts is \begin{align*}
			\Psi(x, t) & = \psi(x) \phi(t) \\
				& = e^{i(kx - \omega t)}, & \text{where } k = \hbar / \sqrt{2mE} \\
				& & \omega = E / \hbar
		\end{align*}
	
		\item No it is not quantized, since there is an infinite spectrum of energies. If it were bounded in a square well, then the energies could be quantized.
		
		\item \begin{enumerate}
			\item From class, we derived \begin{align*}
				a(t) & = a_0 \sqrt{ 1 + \left(\frac{\hbar t}{m {a_0}^2}\right)^2 }.
				\intertext{For $m=\SI{1}{\g}$ and $a_0 = \SI{1}{\um}$, it doubles in width when}
				\sqrt{ 1 + \left(\frac{\hbar t}{m {a_0}^2}\right)^2 } & = 2 \\
				\frac{\hbar t}{m {a_0}^2} & = \sqrt{3} \\
				t & = \frac{ \sqrt{3} m {a_0}^2 }{\hbar} \\
					& \approx \SI{1.64e19}{\s} \approx 38 \times \text{age of the universe}.
			\end{align*}
		
			\item For $m=m_e$, \[ t \approx \SI{14.96}{\ns}. \]
				
				After one second, it's huge \begin{align*}
					a(t) & = a_0 \sqrt{ 1 + \left(\frac{\hbar t}{m_e {a_0}^2}\right)^2 } \\
						& \approx \SI{115.8}{\m}.
				\end{align*}
		\end{enumerate}
	
		\item Considering the Gaussian pdf 
			\[ \rho(x) = Ae^{-\lambda(x-a)^2},\]
			\begin{enumerate}
				\item It is normalized with \begin{align*}
					1 & = A \int_\mathbb{R} e^{-\lambda(x-a)^2} \dd{x} = A \sqrt{\frac{\pi}{\lambda}} \\
					\implies A & = \sqrt{\frac{\lambda}{\pi}}.
				\end{align*}
			
				\item The expectation of $x$ is \begin{align*}
					\expval{x} & = \int_\mathbb{R} x \rho(x) \dd{x} \\
						& = A \int_{-\infty}^\infty x e^{-\lambda(x-a)^2} \dd{x}.
					\intertext{Let $z=x-a$. Then we can write the integral as}
					\expval{x} & = A \int (z+a) e^{-\lambda z^2} \dd{z} \\
						& = A \left( \int z e^{-\lambda z^2} \dd{z} + a\int e^{-\lambda z} \dd{z}   \right)
					\intertext{Let $u = -\lambda z^2$, then $\dd{u} = -2\lambda z \dd{z}$,}
					\expval{x} & = A \left( -\frac{1}{2 \lambda} \int e^{u} \dd{u} + a\sqrt{\frac{\pi}{\lambda}}  \right) \\
						& = A \left(-\frac{1}{2 \lambda} \eval{ e^{-2 \lambda (x-a)^2} }_{z=-\infty}^\infty + a\sqrt{\frac{\pi}{\lambda}}\right) \\
						& = A a \sqrt{ \frac{\pi}{\lambda}} = a?
				\end{align*}
				
				For the expectation of $x^2$, we can first let $u=x-a$, then $x = u+a$, and \begin{align*}
					\expval{x^2} & = \int_{-\infty}^\infty x^2 \rho(x) \dd{x} \\
						& = A\int (u^2 + a^2 + 2au) e^{-\lambda u^2} \dd{x}
					\intertext{The last term is an odd function, so it evaluates to zero and we're left with}
					\expval{x^2} & = A\int u^2 e^{-\lambda u^2} \dd{u} + a^2A\int e^{-\lambda u^2} \dd{u}
					\intertext{Using some WolframAlpha to integrate}
					\expval{x^2} & =  A \left(
						\frac{\sqrt{\pi}}{2 \lambda^{3/2}} 
						+ a^2 \sqrt{ \frac{\pi}{\lambda} }
					\right) \\
						& = \frac{1}{2 \sqrt{\lambda}} + a^2
				\end{align*}
			
				\item The variance is given by \begin{align*}
					\sigma^2 & = \expval{x^2} - \expval{x}^2 \\
						& = a^2 - a + \frac{1}{2 \sqrt{\lambda}}
				\end{align*}
			\end{enumerate}
		
			\item Read 3.1 and 3.2.
	\end{enumerate}
\end{document}