\documentclass{homework}

\title{Homework 2}
\author{Kevin Evans}
\studentid{11571810}
\date{September 8, 2021}
\setclass{Physics}{450}
\usepackage{amssymb}
%\usepackage{mathtools}
\usepackage{graphicx}
\usepackage{amsthm}
\usepackage{amsmath}
\usepackage{slashed}
\usepackage{boldline}
\usepackage{physics}
\usepackage[inter-unit-product =\cdot]{siunitx}

\usepackage[makeroom]{cancel}
\usepackage{booktabs}

\usepackage{times}
\usepackage{mhchem}

%\usepackage{calligra}
%\DeclareMathAlphabet{\mathcalligra}{T1}{calligra}{m}{n}
%\DeclareFontShape{T1}{calligra}{m}{n}{<->s*[2.2]callig15}{}
%\newcommand{\scriptr}{\mathcalligra{r}\,}
%\newcommand{\boldscriptr}{\pmb{\mathcalligra{r}}\,}
%\newcommand{\emf}{\mathcal{E}}

\begin{document}
	\maketitle
	\begin{enumerate}
		\item $\checkmark$ Read through Chapter 1.
		
		\item \textbf{Proposition.} If $\psi_1(\bvec{r}, t)$ and $\psi_2(\bvec{r}, t)$ are solutions to the Schr\"odinger equation (SE), then $\alpha \psi_1 + \beta \psi_2$ is also a solution, where $\alpha, \beta \in \mathbb{C}$.
		\begin{proof}
			If we apply $\Psi = \alpha \psi_1 + \beta \psi_2$ to the SE, then \begin{align*}
				i \hbar \pdv{\Psi}{t} & = - \frac{\hbar^2}{2m} \laplacian{\Psi}+ V \Psi \\
				i \hbar \left(\alpha \pdv{\psi_1}{t} + \beta \pdv{\psi_2}{t}\right) & =  - \frac{\hbar^2}{2m} \left[
					\alpha (\laplacian + V){\psi_1}
					+ \beta (\laplacian + V) \psi_2
				\right]
			\intertext{We can then separate this to two equations in terms of $\psi_1$ and $\psi_2$,}
				i \hbar \alpha \pdv{\psi_1}{t} & = \alpha \left( - \frac{\hbar^2}{2m} \laplacian + V \right) \psi_1 \\
				i \hbar \beta \pdv{\psi_2}{t} & = \beta \left( - \frac{\hbar^2}{2m} \laplacian + V \right) \psi_2 \\
			\end{align*}
			Since these two equations are satisfied, we've shown the linear sum of the two is also a solution to the SE.
		\end{proof}
	
		\item For an arbitrary wavefunction $$\psi(\bvec{r}, t) = A(\bvec{r}, t) e^{i \chi(\bvec{r}, t)},$$ the quantum current density is given by \begin{align*}
			\bvec{j} & \equiv \frac{i\hbar}{2m} \left(\psi \grad{\psi^*} - \psi^* \grad{\psi}\right) \\
				& = \frac{i \hbar}{2m} \left[
					Ae^{i\chi} \grad(Ae^{-i\chi})
					- Ae^{-i\chi} \grad(Ae^{i\chi})
				\right] \\
				& = \frac{i \hbar}{2m} \bigg\{
					Ae^{i \chi} \left[
						\grad{(A)}
						e^{-i \chi}
						+ Ae^{- i\chi}
						\left(-i \grad \chi\right)
					\right] \\
				& \hspace{2.5em} - Ae^{-i \chi} \left[
				\grad{(A)}
				e^{i \chi}
				+ Ae^{ i\chi}
				\left(i \grad \chi\right)
				\right] 
				\bigg\} \\
				& = \frac{i \hbar}{2m} \left(A \grad{A} -i A^2 \grad{\chi} - A \grad{A} -i A^2 \grad{\chi}\right) \\
				& = \frac{\hbar A^2 \grad{\chi}}{m}
		\end{align*}
		
			
			
			\pagebreak
			
		\item For the wavefunction \begin{align*}
			\psi(x, t) & = Ae^{-\lambda \abs{x}} e^{-i \omega t}
			\intertext{It is normalized as}
			\int_\mathbb{R} \psi^* \psi \dd{x} & = A^2 \left(e^{i \omega t} e^{-i\omega t}\right) \int_\mathbb{R} e^{-2\lambda \abs{x}} \dd{x} = 1\\
				& = A^2 \left(
					\int_{-\infty}^0 e^{2 \lambda x} \dd{x} 
					+ \int_0^\infty e^{-2\lambda x} \dd{x}
				\right) 
			\intertext{Using WolframAlpha,}
				& = A^2 \left(\frac{1}{2\lambda } + \frac{1}{2\lambda}\right) \\
			A & =  \sqrt{\lambda}
			\intertext{\textbf{Note: } I'm not sure if we're solving for $A$ or creating a new normalization constant. If it's a new constant, we can set it to}
			\alpha & = \sqrt{\frac{\lambda}{A^2}}.
		\end{align*}
	
		\item For the wavefunction \begin{align*}
			\psi(x, t) & = Ae^{-a \left(
						mx^2 / \hbar + i t
					\right)} 
			\intertext{Before normalizing it, it's clear that the imaginary part will equal $1$ when $\psi^* \psi$ is taken and can be ignored, so}
			\int_\mathbb{R} \psi^* \psi \dd{x} & = A^2 \int_{-\infty}^\infty e^{-2a mx^2 / \hbar} \dd{x} = 1
			\intertext{This is a Gaussian integral and evaluates as}
			1	& = A^2  \sqrt{\frac{\pi}{2am}} \\
			A & = \left(\frac{2am}{\pi}\right)^{1/4} 
			\intertext{If we're using a new constant, it'll be}
			\alpha & = \frac{1}{A} \left(\frac{2 am}{\pi}\right)^{1/4}
		\end{align*}
	
	\pagebreak
		\item \begin{enumerate}
			\item The probability on the range $(a, b)$ is given by \begin{align*}
				P_{ab} & = \int_a^b \psi^* \psi \dd{x}.
				\intertext{Taking the time derivative,}
				\dv{P_{ab}}{t} & = \dv{t} \int_a^b \psi^* \psi \dd{x}.
				\intertext{Applying the chain rule within the integral and applying the SE [eq. (1.23)],}
				\dv{P_{ab}}{t} & = \int_a^b \dv{\psi^*}{t} \psi + \psi^* \dv{\psi}{t} \dd{x} \\
					& = \int_a^b \left(
						-\frac{i \hbar}{2m} \dv[2]{\psi^*}{x}
						+ \frac{i}{\hbar} V \psi^*
					\right) \psi
					+ \psi^* \left(
						\frac{i \hbar}{2m} \dv[2]{\psi}{x}
						- \frac{i}{\hbar} V \psi
					\right) \dd{x}.
			\end{align*}
			The potential terms cancel and we're left with
			\begin{align*}
				\dv{P_{ab}}{t} & = \frac{i \hbar}{2m}\int_a^b  -\dv[2]{\psi^*}{x} \psi + \psi^* \dv[2]{\psi}{x} \dd{x}
				\intertext{By using $J(x, t) = \frac{i\hbar}{2m} \left(\psi \dv{\psi^*}{x} - \psi^* \dv{\psi}{x}\right)$,}
				\dv{P_{ab}}{t} & = \frac{i \hbar}{2m} \int_a^b \dv{J}{x} \dd{x} = J(a, t) - J(b, t) && \qed
			\end{align*}
			The units of $J$ are energy/mass $\cdot$ time.
			
			\item From Problem 1.9, the wavefunction is given by \begin{align*}
				\Psi(x, t) & = Ae^{-a\left[(mx^2/\hbar) + it\right]}
				\intertext{The probability current is}
				J & \equiv \frac{i \hbar}{2m} \left(\Psi \pdv{\Psi^*}{x} - \Psi^* \pdv{\Psi}{x}\right) \\
				& = \frac{i \hbar}{2m} \bigg(
						A e^{-a\left[(mx^2/\hbar) + it\right]}
						\dv{x} A e^{-a\left[(mx^2/\hbar) - it\right]} \\
				& \qquad -	A e^{-a\left[(mx^2/\hbar) - it\right]}
						\dv{x} A e^{-a\left[(mx^2/\hbar) + it\right]}
					\bigg)
				\intertext{The complex phases $\pm it$ will cancel out and we're left with}
				J & = \frac{i \hbar A^2}{2m} \left(
					-(2amx/\hbar)e^{-2amx^2/\hbar}
						- \text{same thing?}
				\right) \\
				& = 0
			\end{align*}
		\end{enumerate}
	\end{enumerate}
\end{document}