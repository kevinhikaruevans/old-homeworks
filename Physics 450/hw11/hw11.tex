\documentclass{homework}

\title{Homework 11}
\author{Kevin Evans}
\studentid{11571810}
\date{November 29, 2021}
\setclass{Physics}{450}
\usepackage{amssymb}
%\usepackage{mathtools}
\usepackage{graphicx}
\usepackage{amsthm}
\usepackage{amsmath}
\usepackage{slashed}
\usepackage{boldline}
\usepackage{physics}
\usepackage[inter-unit-product =\cdot]{siunitx}

\usepackage[makeroom]{cancel}
\usepackage{booktabs}

\usepackage{times}
\usepackage{mhchem}
\usepackage{mathtools}

%\usepackage{calligra}
%\DeclareMathAlphabet{\mathcalligra}{T1}{calligra}{m}{n}
%\DeclareFontShape{T1}{calligra}{m}{n}{<->s*[2.2]callig15}{}
%\newcommand{\scriptr}{\mathcalligra{r}\,}
%\newcommand{\boldscriptr}{\pmb{\mathcalligra{r}}\,}
%\newcommand{\emf}{\mathcal{E}}

\newcommand{\aplus}{\hat{a}_+}
\newcommand{\aminus}{\hat{a}_-}

\begin{document}
	\maketitle
	\begin{enumerate}
		\item Study Chapter 4.2.
		\item Starting from (4.60), \begin{align*}
			u(\rho) & = \rho^{\ell+1} e^{-\rho } v(\rho) \\
			u'(\rho) & = (\ell+1)\rho^\ell e^{-\rho} v(\rho)
				- \rho^{\ell+1} e^{-\rho} v(\rho) 
				+ \rho^{\ell+1}e^{-\rho} v'(\rho) \\
					& = \rho^{\ell}e^{-\rho} \left[
						(\ell+1 - \rho)v(\rho) 
						+ \rho v'(\rho)
					\right] \\
			u''(\rho) & = \ell \rho^{\ell - 1} e^{-\rho}\left[
			(\ell+1 - \rho)v(\rho) 
			+ \rho v'(\rho)
			\right] - \rho^\ell e^{-\rho} \left[
			(\ell+1 - \rho)v(\rho) 
			+ \rho v'(\rho)
			\right] \\
			& \quad + \rho^\ell e^{-\rho} \left[
				(\ell+1) v'(\rho)
				- v(\rho)
				- \rho v'(\rho)
				+ \rho v''(\rho)
				+ v'(\rho)
			\right] \\
			& = \dots
		\end{align*}
	
		\item %4.12
			Using (4.76), for $R_{30}$, the coefficients are given by \begin{align*}
				c_1 & = \frac{2(0+0+1-3)}{1(0+0+2)} c_0 = -2 c_0 \\
				c_2 & = \frac{2(1+0+1-3)}{2(1+0+2)} (-2 c_0) = c_0 \\
				c_3 & = \frac{2(2+0+1-3)}{(\dots)} = 0.
				\intertext{So, $v(\rho)$ becomes}
				v(\rho) & = \left(1 -2\rho + \rho^2 \right)c_0 \\
				R_{30} & = \frac{1}{r} \rho \left( 1-\frac{2}{3a}r + \frac{1}{9a^2} r^2  \right) e^{-r/3a} .
				\intertext{Similarly for $R_{31}$,}
				c_1 & = \frac{2(0+1+1-3)}{2+2}c_0 = -\frac{1}{2} c_0 \\
				c_2 & = \frac{2(1+1+1-3)}{2(1+2+2)} (-1/2) c_0 = 0. \\
				v(\rho) & = (1 - \frac{1}{2} \rho) c_0 \\
				R_{31} & = \frac{r}{9a^2} \left(1-\frac{1}{6a}r\right)  e^{-r/3a}.
				\intertext{Lastly, for $R_{32}$,}
				c_1 & = \frac{2(2+1-3)}{(\dots)} = 0. \\
				v(\rho) = c_0 \\
				R_{32} & = \frac{r^2}{9a^3} e^{-r/3a}
			\end{align*}
		\item %4.15
			\begin{enumerate}
				\item The ground state of hydrogen has wavefunction \begin{align}
					\psi_{100}(r, \theta, \phi) & =  \frac{1}{\sqrt{\pi a^3}} e^{-r/a}. \tag{4.80}
				\end{align}
				
				For $\expval{r}$, \begin{align*}
					\expval{r} & = \frac{1}{\pi a^3} \int_0^\pi \sin \theta \dd{\theta} \int_0^{2\pi} \dd{\phi} \int_0^\infty r^3 e^{-2r/a} \dd{r} \\
						& = \frac{4}{a^3} \frac{3a^4}{8} \\
						& = \frac{3a}{2}.
				\end{align*}
				Similarly, for $\expval{r^2}$,  \begin{align*}
					\expval{r^2} & = \frac{1}{\pi a^3} \int_0^\pi \sin \theta \dd{\theta} \int_0^{2\pi} \dd{\phi} \int_0^\infty r^4 e^{-2r/a} \dd{r} \\
					& = \frac{4}{a^3} \frac{3a^5}{4} \\
					& = 3a^2.
				\end{align*}
			
				\item As $r^2 = x^2 + y^2 + z^2$, the electron will be in the $x$ direction $1/\sqrt{3}$ of the time and in the ``$x^2$'' direction $1/3$rd the time, so \begin{align*}
					\expval{x} & = \frac{3a}{2\sqrt{3}}\\
					\expval{x^2} & = 	a^2.
				\end{align*}
			
				\item From (4.89), for that state, the wavefunction is \begin{align*}
					\psi_{211}(r, \theta, \phi) & = \sqrt{ \left(\frac{1}{a}\right)^3 \frac{1}{4(3!)} }
						e^{-r/2a}
						\left( \frac{r}{a} \right)
						L_0^3(2r/na) Y_1^1(\theta, \phi) \\
							& = -\sqrt{ \frac{3}{192a^5} } r e^{-r/2a} \sin\theta e^{i \phi} \\
							& = -\frac{1}{8a^{5/2} \sqrt{\pi}} r e^{-r/2a} \sin \theta e^{i \phi}.
				\end{align*}
				The expectation $\expval{x^2}$ is then \begin{align*}
					\braket{\psi_{211}}{x^2 \psi_{211}} & = \braket{\psi}{r^2 \sin^2\theta \cos^2\phi \psi} \\
						& =  \frac{1}{64 \pi a^5} \int_0^\pi \sin^5 \theta \dd{\theta}
							\int_0^{2\pi} \cos^2 \phi \dd{\phi}
							\int_0^\infty r^6 e^{-r/a} \dd{r} \\
						& = \frac{1}{64 \pi a^5} \left(\frac{16}{15}\right) \left(\pi\right) \left(720a^7\right) \\
						& = 12a^2.
				\end{align*}
			\end{enumerate}
			
		\item %4.16
			The probability density is given by \begin{align*}
				\rho(r) & = \abs{\Psi}^2 = \frac{1}{{\pi a^3}} e^{-2r/a}.
				\intertext{The most probable point is where the probability is maximized, i.e. the $r$ where }
				\dv{P}{r} & = \dv{4 \pi r^2 \rho(r)}{r} = 0. \\
				\implies \dv{r} \left[r^2 e^{-2r/a}\right] & = e^{-2r/a} \left(2r - 2r^2/a\right) = 0\\
				2r - 2r^2/a & = 0. \\
				\Aboxed{ r & = a. } 
			\end{align*}
		\item %4.19
			The quantities are just scaled by $Z$ or $Z^2$, so: \begin{align*}
				E_n(Z) & = \left(-\SI{13.6}{\eV}\right)\frac{Z^2}{n^2} \\
				E_1(Z) & = \left(\SI{-13.6}{\eV}\right) Z^2 \\
				a_0(Z) & = \left(\SI{0.529e-10}{\m}\right) \frac{1}{Z} \\
				\mathcal{R}(Z) & = \left(\SI{1.097e7}{\per\m}\right) Z^2.
			\end{align*}
			The Lyman series is roughly \begin{align*}
				E_{2\to 1}(2) & = \left(\SI{13.6}{\eV}\right) \frac{4}{4-1} \approx \SI{18}{\eV} \text{\quad (visible or uv?)} \\
				E_{2 \to 1}(3) & \approx \SI{40}{\eV}. \text{\quad (uv)}
			\end{align*}
	\end{enumerate}
\end{document}