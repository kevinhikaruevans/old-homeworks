\documentclass{homework}

\title{Homework 7}
\author{Kevin Evans}
\studentid{11571810}
\date{October 18, 2021}
\setclass{Physics}{450}
\usepackage{amssymb}
%\usepackage{mathtools}
\usepackage{graphicx}
\usepackage{amsthm}
\usepackage{amsmath}
\usepackage{slashed}
\usepackage{boldline}
\usepackage{physics}
\usepackage[inter-unit-product =\cdot]{siunitx}

\usepackage[makeroom]{cancel}
\usepackage{booktabs}

\usepackage{times}
\usepackage{mhchem}
\usepackage{mathtools}

%\usepackage{calligra}
%\DeclareMathAlphabet{\mathcalligra}{T1}{calligra}{m}{n}
%\DeclareFontShape{T1}{calligra}{m}{n}{<->s*[2.2]callig15}{}
%\newcommand{\scriptr}{\mathcalligra{r}\,}
%\newcommand{\boldscriptr}{\pmb{\mathcalligra{r}}\,}
%\newcommand{\emf}{\mathcal{E}}

\begin{document}
	\maketitle
	\begin{enumerate}
		\item \begin{enumerate}
			\item The wavefunction for $x < 0$ can be described as a free particle with a left (incident) and right (reflected) direction, \begin{align*}
				\Psi_I(x) & = A e^{ikx} + B e^{-ikx}.
				\intertext{At $x>0$, it must be a real exponential as $E < V$. It should only be a negative exponential, as a positive one would explode at infinity,}
				\Psi_{II}(x) & = C e^{-\ell x}, \\
				\text{where} \quad k & = \frac{\sqrt{2mE}}{\hbar} \\
					\ell & = \frac{\sqrt{2m(V_0-E)}}{\hbar}.
			\end{align*}
			From the continuity at $x=0$, \begin{align*}
				A + B & = C \\
				ik(A - B) & = -\ell C.
			\end{align*}
			Dividing these, the reflection coefficient can be found as \begin{align*}
				ik (A-B) & = - \ell (A+B) \\
				A(ik + \ell) & = (ik - \ell) B \\
				R & \equiv \frac{\abs{B}^2}{\abs{A}^2} \\
				& = \frac{\abs{ik+\ell}^2}{\abs{ik-\ell}^2} = \frac{-k^2 + \ell^2}{-k^2 + \ell^2} = 1.
			\end{align*}
			The wave is completely reflected at the barrier.
			
			\item For $E>V_0$, the wavefunction over the barrier can take the form of a free particle moving right, \begin{align*}
				\Psi_{II}(x) & = Ce^{i\ell x} \\
				\text{where} \quad \ell & = \frac{\sqrt{2m(E-V_0)}}{\hbar}.
				\intertext{For continuity at $x=0$ in $\Psi(x)$ and $\Psi'(x)$, }
				A + B &= C \\
				ik(A-B) &= i\ell C 
				\intertext{Dividing these again, we see that}
				ik(A-B) & = i\ell(A+B) \\
				i(k - \ell)A & = i(k + \ell)B \\
				R & = \frac{ \abs{B}^2 }{\abs{A}^2} = \frac{ \abs{k-\ell}^2 }{\abs{k+\ell}^2} < 1.
			\end{align*}
		
			\item This problem says to use (2.99) as a hint, but I'm guessing it's actually meaning to use (2.98)? In any case, we know that: \begin{align*}
				v_I & = \sqrt{E/2m} \\
				v_{II} & = \sqrt{(E-V_0)/2m}.
				\intertext{I have no idea where to go from here, but to match the form of (2.175), we're going to need to take the ratio}
				\frac{ v_{II} }{v_I} & = \sqrt{ \frac{E - V_0 }{E} }
				\intertext{Then we can just tack it onto $T$?}
				T & = \sqrt{ \frac{E-V_0}{E} } \frac{\abs{F}^2}{\abs{A}^2}
				\intertext{I don't understand this problem and I hope solutions get posted...}
			\end{align*}
			
			\item
				From part (a) and (b), \begin{align*}
					E & = \frac{\hbar^2 k^2}{2m}\\
					E-V_0 & = \frac{\ell^2 \hbar^2}{2m} \\
					\implies \sqrt{\frac{E-V_0}{E}} & = \frac{\ell}{k}
				\end{align*}
			
				For $E>V_0$ and starting from the continuity of part (b) and the equation from (c), \begin{align*}
				A + B & = C \\
				ik(A-B) & = -i\ell C \implies A-B  = -\frac{\ell}{k} C\\
				\implies 2A & = \left(1 - \frac{\ell}{k}\right) C \\
				T & = \sqrt{ \frac{E-V_0}{E} } \frac{\abs{C}^2}{\abs{A}^2} \\
					& = \frac{4\ell}{k\abs{1-\ell/k}^2} = \frac{4k\ell}{(k-\ell)^2}.
			\end{align*}
			Checking $T+R=1$, \begin{align*}
					T+R & = \frac{4k\ell}{(k-\ell)^2} + \frac{(k-\ell)^2}{(k+\ell)^2} \\
						& = \text{not 1?}
					\intertext{There's an algebraic mistake and I think I should've used $(k+\ell)^2$ in the denominator of the first term.}
						& = \frac{4k\ell}{(k+\ell)^2} + \frac{(k-\ell)^2}{(k+\ell)^2} \\
						& = \frac{4k\ell + (k-\ell)^2}{(k+\ell)^2} = 1. \qed
			\end{align*}
		\end{enumerate}
	\end{enumerate}
\end{document}