\documentclass{homework}

\title{Homework 12}
\author{Kevin Evans}
\studentid{11571810}
\date{December 6, 2021}
\setclass{Physics}{450}
\usepackage{amssymb}
%\usepackage{mathtools}
\usepackage{graphicx}
\usepackage{amsthm}
\usepackage{amsmath}
\usepackage{slashed}
\usepackage{boldline}
\usepackage{physics}
\usepackage[inter-unit-product =\cdot]{siunitx}

\usepackage[makeroom]{cancel}
\usepackage{booktabs}

\usepackage{times}
\usepackage{mhchem}
\usepackage{mathtools}

%\usepackage{calligra}
%\DeclareMathAlphabet{\mathcalligra}{T1}{calligra}{m}{n}
%\DeclareFontShape{T1}{calligra}{m}{n}{<->s*[2.2]callig15}{}
%\newcommand{\scriptr}{\mathcalligra{r}\,}
%\newcommand{\boldscriptr}{\pmb{\mathcalligra{r}}\,}
%\newcommand{\emf}{\mathcal{E}}

\newcommand{\aplus}{\hat{a}_+}
\newcommand{\aminus}{\hat{a}_-}

\begin{document}
	\maketitle
	\begin{enumerate}
		\item Study Chapter 4.4.
		\item % 4.30
			\begin{enumerate}
				\item 			To normalize, we should satisfy \begin{align*}
					1 & = \chi^\dagger \chi \\
					& = A^2 \begin{pmatrix}
						-3i \\ 4
					\end{pmatrix} \begin{pmatrix}
						3i & 4
					\end{pmatrix} \\
					& = A^2 \left[
					-3i(3i)
					+ 4(4)
					\right] = 25 A^2\\
					\Aboxed{ A & = 1/5. }
				\end{align*}
			
				\item For $S_x = \frac{\hbar}{2} \sigma_x$, its expectation is \begin{align*}
					\expval{S_x} & = \chi^\dagger S_x \chi = \frac{\hbar A^2}{2} \begin{pmatrix}
						-3i & 4
					\end{pmatrix} \begin{pmatrix}
					0 & 1 \\
					1 & 0 
				\end{pmatrix} \begin{pmatrix}
				3i \\ 4
			\end{pmatrix}  \\
				& = \frac{\hbar}{50} 
					\begin{pmatrix}
						-3i & 4
					\end{pmatrix} \begin{pmatrix}
					4 \\ 3i
				\end{pmatrix} = \boxed{0.}
				\end{align*}
				Similarly for $S_y = \frac{\hbar}{2} \sigma_y$ and $S_z = \frac{\hbar}{2} \sigma_z$, 
				\begin{align*}
					\expval{S_y} & = \frac{\hbar}{50} \begin{pmatrix}
						-3i & 4
					\end{pmatrix} \begin{pmatrix}
					0 & -i \\
					i & 0
				\end{pmatrix} \begin{pmatrix}
					3i \\ 4
				\end{pmatrix} \\
				& = \frac{\hbar}{50} \begin{pmatrix}
					-3i & 4
				\end{pmatrix} \begin{pmatrix}
					-4i \\ -3
				\end{pmatrix} \\
				& = \frac{\hbar}{50} \left(12i^2 - 12\right) = -\frac{24\hbar}{50} \\
				& = \boxed{-\frac{12\hbar}{25}.}  \\
				\expval{S_z} & = \frac{\hbar}{50} \begin{pmatrix}
						-3i & 4
					\end{pmatrix}
					\begin{pmatrix}
						1 & 0 \\
						0 & -1
					\end{pmatrix} \begin{pmatrix}
					3i \\ 4
				\end{pmatrix} \\
				& = \frac{\hbar}{50} \begin{pmatrix}
					-3i & 4
				\end{pmatrix} \begin{pmatrix}
					3i \\ -4
				\end{pmatrix} \\
				& = \frac{\hbar}{50} \left(
					-9i^2 - 16
				\right) \\
				& = \boxed{ -\frac{7}{50}\hbar }
%				& = \frac{\hbar}{50} \begin{pmatrix}
%					-3i & 4
%				\end{pmatrix} \begin{pmatrix}
%					4 \\
%					3i
%				\end{pmatrix} \\
%					\expval{S_z} & = \frac{\hbar}{50} \begin{pmatrix}
%						-3i & 4
%					\end{pmatrix} \begin{pmatrix}
%								1 & 0 \\
%								0 & -1
%							\end{pmatrix} \begin{pmatrix}
%							4 \\ 3i
%						\end{pmatrix} \\
%					& = \frac{\hbar}{50} \begin{pmatrix}
%						-3i & 4 
%					\end{pmatrix} \begin{pmatrix}
%					4 \\ -3i
%				\end{pmatrix} \\
%					& = \boxed{-\frac{24\hbar}{50} i.}
				\end{align*}
			
			\item To find the uncertainties, we should first find the expectation of the squared operators. Using WolframAlpha, the square of each Pauli matrix is the identity, \begin{align*}
				{S_x}^2 & = {S_y}^2 = {S_z}^2 = \frac{\hbar}{2} \:\mathbb{I}_2.
				\intertext{So, the expectations are also all the same,}
				\expval{{S_x}^2} & = \expval{{S_y}^2} = \expval{{S_z}^2} \\
					& = \frac{\hbar^2}{4} \quad \text{by normalization}
				\intertext{The uncertainties are then}
					\sigma_{S_x} & = \expval{{S_x}^2} - \expval{{S_x}}^2 \\
						& = \frac{\hbar^2}{4} - 0 = \hbar^2/2. \\
					\sigma_{S_y} & = \frac{\hbar^2}{4} - \frac{144\hbar^2}{25^2} = \frac{49}{2500} \hbar^2. \\
					\sigma_{S_z} & = \frac{\hbar^2}{4} - \frac{49}{2500}\hbar^2 = \frac{1201}{2500} \hbar^2.
				\end{align*}
			
			\item For the three permutations, \begin{align*}
				\sigma_x \sigma_y & = \frac{7 \sqrt{2}}{100}\hbar^2 \ge \frac{7}{50}\hbar / 2 && \checkmark \\
				\sigma_y \sigma_z & = \frac{58849}{6250000}\hbar^2 \ge  0 && \checkmark \\
				\sigma_x \sigma_z & = \frac{49}{5000} \hbar^2 \ge 12/25\hbar &&  ?
			\end{align*}
			\end{enumerate}
		\item % 4.31
			For a generalized spinor $\chi = \begin{pmatrix}
				a \\ b
			\end{pmatrix}$, all those expectations are \begin{align*}
				\expval{S_x} & = \frac{\hbar}{2} \begin{pmatrix}
					a^* & b^*
				\end{pmatrix} \begin{pmatrix}
					0 & 1 \\
					1 & 0
				\end{pmatrix} \begin{pmatrix}
				a \\ b
			\end{pmatrix} \\
			& = \frac{\hbar}{2} \left(a^* b + b^* a\right). \\
			\expval{S_y} & = \frac{\hbar}{2} \begin{pmatrix}
					a^* & b^*
				\end{pmatrix} \begin{pmatrix}
					0 & -i \\
					i & 0 
				\end{pmatrix} \begin{pmatrix}
						a \\ b
				\end{pmatrix} \\
			& = \frac{\hbar}{2} \left(-a^* b + b^* a\right). \\
		\expval{S_z} & = \frac{\hbar}{2} \begin{pmatrix}
			a^* & b^*
		\end{pmatrix} \begin{pmatrix}
				1 & 0 \\
				0 & -1 
			\end{pmatrix} \begin{pmatrix}
			a \\ b
		\end{pmatrix} \\
		& = \frac{\hbar}{2} \left(
			a^*a - b^* b
		\right).
		\end{align*}
			Then as \begin{align*}
				{S_x}^2 & = {S_y}^2 = {S_z}^2 = \frac{\hbar^2}{4} \mathbb{I}_2, \\
				\implies \expval{S_x^2} & = \expval{S_y^2} = \expval{S_z^2} \\
					& = \frac{\hbar^2}{4} \left(a^*a + b^*b\right) = \frac{\hbar}{4}. \quad \text{(normalized)}
				\intertext{So, the sum of these is}
				\expval{S_x^2} + \expval{S_y^2} + \expval{S_z^2} & = \frac{3}{4}\hbar. 
				\intertext{Using the eigenvalue definition of $\expval{S^2}$, these are equal,}
				\hbar^2 s(s+1) & = \hbar^2 \frac{1}{2} \frac{3}{2} = \frac{3}{4} \hbar.
			\end{align*}
		\item % 4.32
			\begin{enumerate}
				\item For $S_y$, \begin{align*}
					S_y & = \frac{\hbar}{2} \begin{pmatrix}
						0 & -i \\
						i & 0
					\end{pmatrix}
					\intertext{To find the eigenvalues, we need to first find when the determinate is zero,}
					\begin{vmatrix}
						-\lambda & -\hbar/2 \\
						i \hbar/2 & -\lambda
					\end{vmatrix} & = 0. \\
					\lambda^2 -\hbar^2/4 & = 0 \\
					\lambda & = \pm \hbar / 2.
					\intertext{Plugging this into the eigenvalue problem to find the associated eigenvectors,}
					\frac{\hbar}{2} \begin{pmatrix}
						0 & -i \\
						i & 0
					\end{pmatrix} \begin{pmatrix}
						\alpha \\ \beta
					\end{pmatrix} & = \pm \frac{\hbar}{2} \begin{pmatrix}
						\alpha \\ \beta
					\end{pmatrix} \\
					\begin{pmatrix}
						-i \beta \\ i\alpha
					\end{pmatrix} & = \pm \begin{pmatrix}
							\alpha \\ \beta
					\end{pmatrix}. \\
					\implies -i \beta & = \pm \alpha \\
					i\alpha & = \pm \beta.
					\intertext{From inspection, the eigenvectors are }
					\chi_\pm^{(y)} & = \frac{1}{\sqrt{2}}\begin{pmatrix}
						1 \\
						\pm i
					\end{pmatrix}.
				\end{align*}
			
				\item If we measure $S_y$ on a generalized spinor $\chi$, we will get either $\pm \hbar/2$. The corresponding probabilities are \begin{align*}
					P^\pm & = \braket{\chi^\pm}{\chi}^2 \\
						& = \frac{1}{2} \abs{ \begin{pmatrix}
							1 & \mp i
						\end{pmatrix} \begin{pmatrix}
							a \\ b
						\end{pmatrix}}^2 \\
						& = \frac{1}{2} \abs{ a \mp ib }^2.
					\intertext{These add up to 1 (on WolframAlpha).}
				\end{align*}
			
				\item By Problem 3, we would get the expectation, $\hbar/4$ always.
			\end{enumerate}
	\end{enumerate}
\end{document}