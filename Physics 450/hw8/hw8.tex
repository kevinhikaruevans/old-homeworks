\documentclass{homework}

\title{Homework 8}
\author{Kevin Evans}
\studentid{11571810}
\date{October 25, 2021}
\setclass{Physics}{450}
\usepackage{amssymb}
%\usepackage{mathtools}
\usepackage{graphicx}
\usepackage{amsthm}
\usepackage{amsmath}
\usepackage{slashed}
\usepackage{boldline}
\usepackage{physics}
\usepackage[inter-unit-product =\cdot]{siunitx}

\usepackage[makeroom]{cancel}
\usepackage{booktabs}

\usepackage{times}
\usepackage{mhchem}
\usepackage{mathtools}

%\usepackage{calligra}
%\DeclareMathAlphabet{\mathcalligra}{T1}{calligra}{m}{n}
%\DeclareFontShape{T1}{calligra}{m}{n}{<->s*[2.2]callig15}{}
%\newcommand{\scriptr}{\mathcalligra{r}\,}
%\newcommand{\boldscriptr}{\pmb{\mathcalligra{r}}\,}
%\newcommand{\emf}{\mathcal{E}}

\newcommand{\aplus}{\hat{a}_+}
\newcommand{\aminus}{\hat{a}_-}

\begin{document}
	\maketitle
	\begin{enumerate}
		\item Starting from (2.58) and following the proof on p. 42, \begin{align*}
			\hat{H} & = \hbar \omega \left(\hat{a}_\pm \hat{a}_\mp \pm \frac{1}{2}\right), \\
			\hat{H}(\hat{a}_- \psi) & = \hbar \omega \left(\aminus \aplus - \frac{1}{2}  \right) \aminus \psi \\
				& = \hbar \omega \aminus \left(\aplus \aminus - \frac{1}{2} \right)\psi
			\intertext{As $\comm{\aminus}{\aplus} = \aminus \aplus - \aplus \aminus = 1 \implies \aplus \aminus = \aminus \aplus - 1$,}
				& = \aminus \left[
					\hbar \omega \left(\aminus \aplus - 1 - \frac{1}{2}\right) \psi
				\right] \\
				& = \aminus \left[
					\underbrace{\hbar \omega \left(\aminus \aplus  - \frac{1}{2}\right)}_{\hat{H}}- \hbar \omega 
				\right]\psi  \\
				\hat{H}(\aminus \psi) & = \aminus \left(E - \hbar \omega\right) \psi \qed
			\intertext{Therefore $E-\hbar \omega$ is a solution to the Hamiltonian of the lowering operator on a wavefunction.}
		\end{align*}
		On the ground state, the energy is $E_0 = \hbar \omega / 2$ and lowering it results in a negative energy $$E_{-1} = -\hbar \omega / 2.$$
		\item From the analytical solution (2.86), \begin{align*}
			\psi(x) & = \left(\frac{m \omega}{\pi \hbar}\right)^{1/4}\frac{1}{\sqrt{2^n n!}} H_n (\xi) e^{-\xi^2 / 2} \\
				& = \left(\frac{m \omega}{\pi \hbar}\right)^{1/4} e^{- \xi^2 / 2} \\
				& = \left(\frac{m \omega}{\pi \hbar}\right)^{1/4} e^{- m \omega x^2 / 2 \hbar}.
		\end{align*}
		Then applying the raising operator $\hat{a}_+$, \begin{align*}
			\hat{a}_+ \psi(x) & = \frac{1}{\sqrt{2 \hbar m \omega}} \left(- i \hat{p} + m \omega x\right) \psi(x) \\
				& =  \frac{1}{\sqrt{2 \hbar m \omega}} \left(
						\frac{m \omega}{\pi \hbar}
					\right)^{1/4} 
					\left(
						m \omega x e^{-m \omega x^2 / 2 \hbar} 
						+ m \omega x e^{-m \omega x^2 / 2 \hbar}
					\right) \\
				& = \left(
				\frac{m \omega}{\pi \hbar}
				\right)^{1/4} \frac{1}{\sqrt{2}} \left(2
					\sqrt{ \frac{m \omega}{\hbar} } x
				\right) e^{-m \omega x^2 / 2 \hbar}.
		\end{align*}
		\item Study Chapter 3.5.
	\end{enumerate}
\end{document}