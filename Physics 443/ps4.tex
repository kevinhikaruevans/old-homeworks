\documentclass{homework}

\title{Problem Set 4}
\author{Kevin Evans}
\studentid{11571810}
\date{September 28, 2020}
\setclass{Physics}{443}
\usepackage{amssymb}
\usepackage{mathtools}

\usepackage{amsthm}
\usepackage{amsmath}
\usepackage{slashed}
\usepackage{relsize}
\usepackage{threeparttable}
\usepackage{float}
\usepackage{booktabs}
\usepackage{boldline}
\usepackage{changepage}
\usepackage{physics}
\usepackage[inter-unit-product =\cdot]{siunitx}
\usepackage{setspace}

\usepackage[makeroom]{cancel}
%\usepackage{pgfplots}

\usepackage{enumitem}
\usepackage{times}

\usepackage{tcolorbox}

\begin{document}
	\maketitle
	\begin{enumerate}
		\item From the frequency dependent index of refraction, \begin{align*}
			n^2 & = 1 + \frac{Ne^2}{m\epsilon_0} \sum_j \left(\frac{f_j}{\omega_j^2 - \omega^2}\right)
			\intertext{We can apply the relation $\omega = 2 \pi c / \lambda$ and rewrite $n$ as}
			n^2 & = 1 + \frac{Ne^2}{m \epsilon_0} \sum_j \frac{f_j}{4 \pi^2 c^2 / \lambda_j^2 - 4 \pi^2 c^2 / \lambda^2} \\
				& = 1 + \frac{Ne^2}{m \epsilon_0} \sum_j \frac{f_j \lambda_j^2 \lambda^2 / 4 \pi^2 c^2}{\lambda^2 - \lambda_j^2}
			\intertext{Condensing the coefficient of each term in the summation as $A_j = Ne^2 f_j \lambda_j^2 / 4 \pi^2 c^2 m \epsilon_0$,}
			n^2 & = 1 + \sum_j \frac{A_j \lambda^2}{\lambda^2 - \lambda_j^2} \qed
		\end{align*}
	
		\item From Problem 4.4 of Hecht,\begin{enumerate}
			\item The terms can roughly be described as \begin{align*}
				m_e \ddot{x} & : \text{The force experienced by an electron} \\
				m_e \gamma \dot{x} & : \text{The resistive/friction term} \\
				m_e \omega_0^2 x & : \text{The Hooke's law/spring restoring force term} \\
				q_e E(t) & : \text{A driving force due to the electron charge within the electric field}
			\end{align*}
		
			\item With a solution of form $x = x_0 e^{i(\omega t - \alpha)}$, \begin{align*}
				\dot{x} & = i x_0 \omega e^{i(\omega t - \alpha)} = i \omega x \\
				\ddot{x} & = -x_0  \omega^2 e^{i(\omega t - \alpha)} = -\omega^2 x
				\intertext{Applying this to the driven and damped oscillator DE,}
				m_e \ddot{x} + m_e \gamma \dot{x} + m_e \omega_0^2 x & = q_e E(t) \\
				- m_e \omega^2 x + m_e \gamma i \omega x + m_e \omega_0^2 & = q_e E(t) \\
				m_e x \left(\omega_0^2 - \omega^2 + i \omega \gamma\right) & = q_e E(t) \\
				x_0 & = \frac{q_e E(t)}{m_e} \left[ \frac{1}{\omega_0^2 - \omega^2 + i \omega \gamma}\right]  \frac{1}{e^{i \omega t} e^{-i \omega t}} \\
				& = \frac{q_e E_0}{m_e} \frac{e^{i \alpha}}{\omega_0^2 - \omega^2 + i \omega \gamma}
				\intertext{Since $x_0 \in \mathbb{R}$, the modulus will be taken (and as $\abs{e^{i \alpha}} = 1$)}
				\abs{x_0} & = \frac{q_e E_0}{m_e} \frac{1}{\left[ \left(\omega_0^2 - \omega^2\right)^2 + \omega^2 \gamma^2\right]^{1/2}}
			\end{align*}
		
			\item From (b), we can use $x_0$ and find the phase angle by using the real and imaginary components, \begin{align*}
				x_0 & = \frac{q_e E_0}{m_e} \frac{e^{i\alpha}}{\omega_0^2 - \omega^2 + i \omega \gamma} \\
				\Re{x_0} & = \frac{ q_e E_0 }{m_e} \frac{\cos \alpha }{\omega_0^2 - \omega^2} \\
				\Im{x_0} & = \frac{q_e E_0}{m_e} \frac{\sin\alpha}{\omega \gamma} \\
				\tan \alpha & = \frac{\sin \alpha}{\cos \alpha}  = \frac{\omega \gamma }{\omega_0^2 - \omega^2} \\
				\alpha & = \arctan(\frac{\omega \gamma}{\omega_0^2 - \omega^2})
			\end{align*}
		\end{enumerate}
	
		\item From the trial solution derived in-class, the complex wavenumber was shown as \begin{align*}
			k^2 & = \frac{\omega^2}{c^2} - \frac{i \omega \mu_0 \sigma_0}{1 + i \omega \tau}
			\intertext{Then from the relations $k^2 = n^2 \omega^2 / c^2$ and $\sigma_0 = \omega_p^2 \tau / \mu_0 c^2$,}
			n^2 & = \frac{c^2}{\omega^2} \left[
				\frac{\omega^2}{c^2} - \frac{i \omega \mu_0 \sigma_0}{1 + i \omega \tau}
			\right] \\
				& = 1 - \frac{i \omega_p^2 \tau}{\omega + i \omega^2 \tau}
			\intertext{For $\omega_p = \SI{10e15}{\radian\per\s}$ and $\tau = \SI{10e-13}{s}$,}
				n(\omega_p) & = \sqrt{1 - \frac{i \left(10^{15}\right)^2 \left(10^{-13}\right)}{10^{15} + i \left(10^{15}\right)^2 10^{-13}} } \approx 0.071 - i0.070 \\
				n(2\omega_p) & \approx 0.866 - i0.000722 \\
				n(\omega_p / 2)  & \approx 0.023 - i1.732
		\end{align*}
	
		\item The skin depth is given as $\delta = \frac{1}{\alpha} = \sqrt{2 / \omega \mu_0 \sigma_0}$. \begin{enumerate}
			\item For $\lambda = \SI{600}{\nano\meter}$, \begin{align*}
				d & = \sqrt{ \frac{2}{\omega \mu_0 \sigma_0} } = \sqrt{ \frac{\lambda}{\pi c \mu_0 \sigma_0}} \\
					& = \sqrt{ \frac{\SI{600}{\nano\meter}}{\pi c \mu_0 \left(\SI{6e7}{\siemens\per\m}\right)}  } \\
					& \approx \SI{2.9}{\nm}
			\end{align*}
			\item For $\lambda = \SI{0.6}{\centi\meter}$, \begin{align*}
				d & =  \sqrt{ \frac{\SI{0.6}{\centi\meter}}{\pi c \mu_0 \left(\SI{6e7}{\siemens\per\m}\right)}  } \\
					& \approx \SI{290}{\nm}
			\end{align*}
		\end{enumerate}
	\end{enumerate}
\end{document}