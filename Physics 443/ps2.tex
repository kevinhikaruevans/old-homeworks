\documentclass{homework}

\title{Problem Set 2}
\author{Kevin Evans}
\studentid{11571810}
\date{September 9, 2020}
\setclass{Physics}{443}
\usepackage{amssymb}
\usepackage{mathtools}

\usepackage{amsthm}
\usepackage{amsmath}
\usepackage{slashed}
\usepackage{relsize}
\usepackage{threeparttable}
\usepackage{float}
\usepackage{booktabs}
\usepackage{boldline}
\usepackage{changepage}
\usepackage{physics}
\usepackage[inter-unit-product =\cdot]{siunitx}
\usepackage{setspace}

\usepackage[makeroom]{cancel}
%\usepackage{pgfplots}

\usepackage{enumitem}
\usepackage{times}


\begin{document}
	\maketitle
	
	\begin{enumerate}
		\item \begin{enumerate}
			\item Linear polarization, $63^\circ$ from the $x$ axis.
			\item Right elliptical polarization.
			\item Left circular polarization.
		\end{enumerate}
	
		\item \begin{enumerate}
			\item If we add a $\pi/2$ phase shift, \begin{align*}
				\left[ \begin{matrix}
					-2 \\ 1
				\end{matrix} \right]
			\end{align*}
			It's still linearly polarized, just out of phase.
			
			\item Using a left elliptical polarization, these matrices are orthogonal \begin{align*}
				\left[
				\begin{matrix}
					2 \\ i				
				\end{matrix}
				\right] \text{ or } \left[
					\begin{matrix}
						-2i \\ 1
					\end{matrix}
				\right]
			\end{align*}
			(The relative difference is the same, right?)
			
			\item Using a right circular polarization, \begin{align*}
				\left[
					\begin{matrix}
						1 \\ -i
					\end{matrix}
				\right]
			\end{align*}
		\end{enumerate}
	
		\item \begin{enumerate}
			\item Reusing the diagram and conventions from in class, where \begin{align*}
				n_1 & = \text{higher/slower index ($y$)} \\
				n_2 & = \text{lower/faster index ($x$)}
\intertext{			Then piggybacking from the quarter-wave derivation,}
				\Delta \phi & = k_y d - k_x d = \frac{2\pi d}{\lambda} \left(n_1 - n_2\right) \\
				\pi & = \frac{2 \pi d}{\lambda} \left(n_1 - n_2\right) \\
				d (n_1 - n_2) & = \frac{\lambda}{2} \\
					& = \frac{\lambda}{2} + m \lambda \\
					& \hspace{1.5em} \text{(where $m \in \mathbb{Z}$, for higher orders)}
			\end{align*}
		
			\item Finding the Eigenvalues $\lambda$, \begin{align*}
				\left[ \begin{matrix}
					1 & 0 \\
					0 & -1
				\end{matrix}\right] \bvec{M} & = \lambda \bvec{M} \\
				\abs{ \begin{matrix}
					1 - \lambda & 0 \\
					0 & -1 - \lambda
				\end{matrix} } & = 0 \\
				\left(1 - \lambda\right)\left(-1 - \lambda\right) & = 0 \\
				\lambda & = \pm 1
				\intertext{Substituting the found $\lambda$ values into the original equation, the Eigenvectors can be found. }
				\left[ \begin{matrix}
					1 - \lambda & 0 \\
					0 & -1 - \lambda
				\end{matrix}\right] \left[
					\begin{matrix}
						A \\ B
					\end{matrix}
				\right] = 0 \\
				\left(1 - \lambda\right) A = 0 \\
				\left(-1 - \lambda\right) B = 0
			\end{align*}
				For $\lambda = 1$, $B=0$ and the relative Jones vector is
				$\left[
					\begin{matrix}
						1 \\
						0
					\end{matrix}
				\right]$.
				
				Similarly for $\lambda = -1$, the Jones vector is 
				$\left[
					\begin{matrix}
						0 \\
						1
					\end{matrix}
				\right]$.
			\item A linearly polarized wave at $45^\circ$ can be represented with a Jones vector $\left[\begin{matrix}
			1 \\ 1
			\end{matrix}\right]$, and using the Jones matrix from (b), \begin{align*}
				\left[\begin{matrix}
					1 & 0 \\
					0 & -1
				\end{matrix}\right] \left[
				\begin{matrix}
					1 \\ 1
				\end{matrix}
			\right] & = \left[
				\begin{matrix}
					1 \\
					-1
				\end{matrix}
			\right]
			\end{align*}
		The $x$-component remains the same, but now the $y$-component is ``flipped.'' The wave is now phase shifted by a half-wave and is linearly polarized at $225^\circ$.
		\end{enumerate}
	\end{enumerate}
\end{document}