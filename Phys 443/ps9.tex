\documentclass{homework}

\title{Problem Set 9}
\author{Kevin Evans}
\studentid{11571810}
\date{November 18, 2020}
\setclass{Physics}{443}
\usepackage{amssymb}
\usepackage{mathtools}

\usepackage{amsthm}
\usepackage{amsmath}
\usepackage{slashed}
\usepackage{relsize}
\usepackage{threeparttable}
\usepackage{float}
\usepackage{booktabs}
\usepackage{boldline}
\usepackage{changepage}
\usepackage{physics}
\usepackage[inter-unit-product =\cdot]{siunitx}
\usepackage{setspace}

\usepackage[makeroom]{cancel}
%\usepackage{pgfplots}

\usepackage{enumitem}
\usepackage{times}

\usepackage{tcolorbox}
\usepackage{mathrsfs}

\begin{document}
	\maketitle
	\begin{enumerate}
		\item The angle between the two stars is \begin{align*}
			\theta & = \tan[-1](\frac{\SI{50e6}{\km}}{\SI{10}{ly}}) \\
				& = \SI{5.285e-7}{\radian}
			\intertext{From the Rayleigh criterion,}
			\sin(\theta) & = 1.22 \lambda / D \\
			\sin(\num{5.285e-7}) & = 1.22 \left(\SI{500}{\nm}\right) / (2r) \\
			r & = \SI{66.1}{\m}
		\end{align*}
		\item The forth secondary maxima is given by the grating equation for $N=2$ \begin{align*}
			\gamma & = \frac{9 \pi}{4}
			\intertext{For the forth secondary maxima and a zero of the envelope to coincide, $\beta = 1 \pi$, and the ratio is then}
			\frac{b}{h} & = \frac{\beta}{\gamma} = 4 / 9
		\end{align*}
		\item For the sodium doublet, \begin{align*}
			\Delta \lambda & = \SI{589.592}{\nm} - \SI{588.995}{\nm} \\
				& = \SI{0.597}{\nm} \\
			\mathrm{RP} & = nN = \frac{\lambda}{\Delta \lambda} \\
				N & = \frac{\SI{589.2935}{\nm}}{\SI{0.597}{\nm}} \\
				& = 987 \text{ lines }
		\end{align*}
	\end{enumerate}
\end{document}