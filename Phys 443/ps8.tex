\documentclass{homework}

\title{Problem Set 8}
\author{Kevin Evans}
\studentid{11571810}
\date{November 9, 2020}
\setclass{Physics}{443}
\usepackage{amssymb}
\usepackage{mathtools}

\usepackage{amsthm}
\usepackage{amsmath}
\usepackage{slashed}
\usepackage{relsize}
\usepackage{threeparttable}
\usepackage{float}
\usepackage{booktabs}
\usepackage{boldline}
\usepackage{changepage}
\usepackage{physics}
\usepackage[inter-unit-product =\cdot]{siunitx}
\usepackage{setspace}

\usepackage[makeroom]{cancel}
%\usepackage{pgfplots}

\usepackage{enumitem}
\usepackage{times}

\usepackage{tcolorbox}
\usepackage{mathrsfs}

\begin{document}
	\maketitle
	\begin{enumerate}
		\item For $\lambda = \SI{600}{\nm}$ and $\Delta \lambda = \SI{10}{\nm}$, the coherence length and time is given by \begin{align*}
			\Delta \ell_c & = \frac{\lambda^2}{\Delta \lambda} = \SI{36}{\um} \\
			\Delta t_c & = \Delta \ell_c / c = \SI{120}{\fs}
		\end{align*}
	
		\item Using the notes in-class and since we're dealing with a small angle, \begin{align*}
			\ell_t & = \lambda / \theta_s \\
				& = \frac{\SI{600}{\nm}}{\left(\SI{0.5}{\deg} \times \pi / \SI{180}{\deg}\right)} \\
				& = \SI{68.8}{\um}
		\end{align*}
	
		\item In this problem, we're trying to find the separation $s$, which can be found treating them as two incoherent sources, 
		\begin{align*}
			\ell_t & = \frac{\lambda}{\theta_s} \approx \frac{\lambda r}{s} \\
			s & = \frac{\lambda r}{\ell_t} = \frac{\SI{589}{\nm} \times \left(\SI{2}{\m} \times \SI{1e9}{\nm/\m}\right)}{\SI{1}{\mm} \times \SI{1e6}{\mm/\nm}} \\
				& = \SI{1.2}{\mm}
		\end{align*}
	
		\item The envelope square wave has a frequency of \SI{40}{\MHz}, so for each period, it's able to pass \SI{12.5}{\ns} of light. This corresponds to a coherence length of \SI{3.75}{\m}. From the notes in class, \begin{align*}
			\ell_c & = \frac{\lambda^2}{\Delta \lambda} \\
			\Delta \lambda & = \frac{\lambda^2}{\ell_c} = \frac{\left(\SI{488}{\nm}\right)^2}{\SI{3.75}{\m}} \\
				& = \SI{6.35e-5}{\nm}
		\end{align*}
		
		\item For the frequency spectrum $$I(\omega) = \frac{A}{(\omega - \omega_0)^2 + b^2}$$ the width $\Delta \omega$ is found by the FWHM. The maximum is found at $\omega = \omega_0$ as $I_\mathrm{max} = A/b^2$. At half-max, $\omega=\omega_0 \pm b$, so the full width is $\Delta \omega = 2b$, or $\Delta \nu = b / \pi$. 
		
		From the relation from in-class, \begin{align*}
			\expval{\tau_0} & = \frac{1}{\Delta \nu} = \frac{\pi}{b}
		\end{align*}
	\end{enumerate}
\end{document}