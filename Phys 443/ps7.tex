\documentclass{homework}

\title{Problem Set 7}
\author{Kevin Evans}
\studentid{11571810}
\date{October 26, 2020}
\setclass{Physics}{443}
\usepackage{amssymb}
\usepackage{mathtools}

\usepackage{amsthm}
\usepackage{amsmath}
\usepackage{slashed}
\usepackage{relsize}
\usepackage{threeparttable}
\usepackage{float}
\usepackage{booktabs}
\usepackage{boldline}
\usepackage{changepage}
\usepackage{physics}
\usepackage[inter-unit-product =\cdot]{siunitx}
\usepackage{setspace}

\usepackage[makeroom]{cancel}
%\usepackage{pgfplots}

\usepackage{enumitem}
\usepackage{times}

\usepackage{tcolorbox}
\usepackage{mathrsfs}

\begin{document}
	\maketitle
	\begin{enumerate}
		\item As the function is odd, we can assume $A_m = 0 \; \forall \; m$ and only find $B_m$, \begin{align*}
			B_m & = \frac{2}{T} \left[ \int_0^{T/2} \sin(m \omega t) \dd{t} - \int_{T/2}^T \sin(m \omega t) \dd{t}\right] \\
				& = -\frac{2}{m\omega T} \left[ \eval{ \cos(m \omega t) }_0^{T/2} -  \eval{\cos(m \omega t) }_{T/2}^{T} \right] \\
				& = -\frac{1}{m \pi} \left[\cos(m \pi) - 1 - 1 + \cos(m \pi) \right] \\
				& = \frac{2}{m \pi} \left(1 - \cos m \pi\right)
			\intertext{As this is only non-zero for odd $m$, the Fourier series is}
			f(t) & = \frac{4}{\pi} \sum_{m = 1, 3, 5\dots} \frac{1}{m} \sin( m \omega t)
		\end{align*}
	
		\item For the function \[f(x) = e^{-ax}\]
	
		We can apply the Fourier transform for $x > 0$ and find \begin{align*}
			F(k) & = \int_0^\infty e^{-ax} e^{-ikx} \dd{x} \\
				& = \int_0^\infty e^{(-a -ik)x} \dd{x} \\
				& = \frac{1}{a + ik}
		\end{align*}
	
		\item If $\mathscr{F} \{g(x)\} = G(k)$, then $\mathscr{F} \{ g(x-a) \} = G(k)e^{-ika}$
		\begin{align*}
			\textit{Let } u & = x-a \\
			\textit{Then } x & = u + a \\
				\dd{x} & = \dd{u} \\
			\mathscr{F}\{g(u)\} & = \int_\mathbb{R} g(u) e^{-ik(u+a)} \dd{x} \\
				& = \int_\mathbb{R} g(u) e^{-ik(u+a)} \dd{x} \\
				& = e^{-ika} \underbrace{ \int_\mathbb{R} g(u) e^{-iku} \dd{u}}_{G(k)} \\ 
				& = G(k) e^{-ika}&&  \qed
		\end{align*}
	
		\item From the in-class derivation of the Fourier transform of a Gaussian cosine, the FWHM in the time domain is \begin{align*}
			\Delta t & = 2 t_\mathrm{1/2} = 2 \sqrt{\alpha \ln 2} \\
				& = 2 \sqrt{ \tau^2 \ln 2} = \SI{33.3}{\fs}
		\end{align*}
	
		\begin{enumerate}
			\item In the frequency spectrum, if we assume a product of $8 \ln 2$, \begin{align*}
				\Delta \nu \Delta t & = 8 \ln 2 / (2 \pi) \\
				\Aboxed{ \Delta \nu & = \SI{ 26.5 }{\THz} }
			\end{align*}
		
			\item For the (absolute) wavelength, \begin{align*}
				\Delta \lambda & = \frac{\lambda^2 \Delta \nu }{c} \\
					& = \frac{\left(\SI{800}{\nm}\right)^2 \left(\SI{26.5}{\THz}\right)}{c} \\
				\Aboxed{ \Delta \lambda & =  \SI{56.5}{\nm} }
			\end{align*}
		\end{enumerate}
	\end{enumerate}
\end{document}