\documentclass{homework}

\title{Homework 7}
\author{Kevin Evans}
\studentid{11571810}
\date{March 24, 2021}
\setclass{Physics}{342}
\usepackage{amssymb}
\usepackage{mathtools}

\usepackage{amsthm}
\usepackage{amsmath}
\usepackage{slashed}
\usepackage{relsize}
\usepackage{threeparttable}
\usepackage{float}
\usepackage{booktabs}
\usepackage{boldline}
\usepackage{changepage}
\usepackage{physics}
\usepackage[inter-unit-product =\cdot]{siunitx}
\usepackage{setspace}

\usepackage[makeroom]{cancel}
%\usepackage{pgfplots}

\usepackage{enumitem}
\usepackage{times}
\usepackage{mhchem}

\usepackage{calligra}
\DeclareMathAlphabet{\mathcalligra}{T1}{calligra}{m}{n}
\DeclareFontShape{T1}{calligra}{m}{n}{<->s*[2.2]callig15}{}
\newcommand{\scriptr}{\mathcalligra{r}\,}
\newcommand{\boldscriptr}{\pmb{\mathcalligra{r}}\,}
\newcommand{\emf}{\mathcal{E}}

\begin{document}
	\maketitle
	\begin{enumerate}
		\item \begin{enumerate}
			\item For the electric field, \begin{align*}
				\bvec{E} & = -\pdv{\bvec{A}}{t} = \begin{cases}
					-\frac{\mu_0 k s}{2} \uvec{\phi} & s \le R \\
					-\frac{\mu_0 k R^2}{2s} \uvec{\phi} & s \ge R
				\end{cases}
				\end{align*}
				For the magnetic field, \begin{align*}
					\bvec{B} & = \curl{\bvec{A}}  = \begin{cases}
						 {\mu_0 k t} \uvec{z} & s \le R \\
						 0 & s \ge R
					\end{cases}
				\end{align*}
			
			\item The charge distribution can be found using \begin{align*}
				-\frac{\rho}{\epsilon_0} & = \cancel{ \laplacian{V} } + \dv{t} \div{\bvec{A}} \\
				\rho & = 0 \text{ as there's no $\phi$ dependence on the $\phi$ component}
			\end{align*}
		
			From Ampere's law, \begin{align*}
				\mu_0 \bvec{J} & = \curl{\bvec{B}} - \mu_0 \epsilon_0 \pdv{\bvec{E}}{t} \\
					& = 0?
			\end{align*}
			These can't both be zero, so I'm guessing there's an arithmetic mistake somewhere?
%			TODO: use Maxwell's here
%			Using the Lorenz gauge, \begin{align*}
%				-\mu_0 \bvec{J} & = \laplacian{\bvec{A}} \\
%					\bvec{J} & = -\mu_0 (-{ A_\phi } /{s^2}) \\
%						& = \begin{cases}
%							\frac{ {\mu_0}^2 k t }{2s} \uvec{\phi} & s \le R \\
%							\frac{\mu_0 k t R^2}{2s^3} & s \ge R
%						\end{cases} 
%			\end{align*}
%			The only non-zero component of the charge distribution is\begin{align*}
%			\mu_0 \bvec{J} & = 	\grad{\mu_0 \epsilon_0 \dv}
%			\end{align*}

		\end{enumerate}
	
		\item \begin{enumerate}
			\item The magnetic field is \begin{align*}
				\bvec{B} & =\curl{\bvec{A}} = 0
				\intertext{The electric field is}
				\bvec{E} & = - \grad V - \pdv{\bvec{A}}{t} \\
					& = \frac{q}{4 \pi \epsilon_0 r^2} \uvec{r}
				\intertext{By Gauss' law,}
				\div{\bvec{E}} & = \frac{\rho}{\epsilon_0} \\
				\rho & = \frac{q}{4 \pi \epsilon_0} (4 \pi \delta) \epsilon_0\\
					& = q \delta 
				\intertext{The current can be found using Ampere's law,}
				\mu_0 \bvec{J} & =  \curl{\bvec{B}} - \mu_0 \epsilon_0 \pdv{\bvec{E}}{t} \\
					& = 0
			\end{align*}
			
			\item Using $\lambda = -(1/4\pi \epsilon_0) (qt / r)$, \begin{align*}
				\bvec{A}' & = \bvec{A} + \grad{\lambda} \\
					& = -\frac{qt}{4 \pi \epsilon_0} \left(\frac{1}{r^2} - \frac{1}{r^2}\right) = 0 \\
				V' & = V - \pdv{\lambda}{t} \\
					& = \frac{q}{4 \pi \epsilon_0r}
			\end{align*}
		\end{enumerate}
	
		\item \begin{enumerate}
			\item[(i)] Lorenz gauge due to the $t$ dependence.
			\item[(ii)] Could be both?
			\item[(iii)] Neither? This can't be the Coulomb gauge as $\div{\bvec{A}} \ne 0$ and it can't be in the Lorenz gauge as it can't satisfy (10.12).
		\end{enumerate}
	
		\item We can divide this problem to two parts: the semicircle and the ends. For the semicircle, the distance $\scriptr$ is constant and we can just integrate along the swept angle, \begin{align*}
			\bvec{A} & = \frac{\mu_0}{4 \pi} \int \frac{I(t_r)}{\scriptr} \dd{\vec{\ell}'} \\
				& = \frac{\mu_0 I_0}{4 \pi} \int_{-\pi/2}^{\pi/2} \left(\cos \phi \uvec{x} + \sin \phi \uvec{y}\right) \dd{\phi} \\
				& = \frac{\mu_0 I_0}{2 \pi} \uvec{x}
		\end{align*}
		For the two lines, the point only ``sees'' up to $ct$ and is multiplied twice as the lines are symmetric, so the vector field is \begin{align*}
			\bvec{A} & = 2 \times \frac{\mu_0 I_0}{4 \pi} \uvec{x} \int_R^{ct} x^{-1} \dd{x} \\
				& = \frac{\mu_0 I_0}{2 \pi} \ln(ct / R) \uvec{x}
		\end{align*}
		The total vector field is then \begin{align*}
			\bvec{A}(t) & = \begin{cases}
				0 & t \le 0 \\
				\frac{\mu_0 I_0}{2 \pi} \left(\ln(ct/R)  + \frac{1}{R} \right) \uvec{x} & t > 0?
			\end{cases}
			\intertext{(I'm not sure if it's only non-zero after $t=0$ or if it would be after $t=R/c$.)}
		\end{align*}
	
		\item Starting from (10.36), $\dot{\bvec{J}} = 0$ and \begin{align*}
			\bvec{E}(\bvec{r}, t) & = \frac{1}{4 \pi \epsilon_0} \int \left[
				\frac{\rho(\bvec{r'}, t)}{\scriptr^2} \uvec{\boldscriptr}
				+ \frac{\dot{\rho}(\bvec{r'}, t_r)}{c \scriptr} \uvec{\boldscriptr} 
			\right] \dd{\tau'}\\
			 & = \frac{1}{4 \pi \epsilon_0} \int \left[
			\frac{c\rho(\bvec{r'}, t)}{c\scriptr^2} \uvec{\boldscriptr}
			+ \frac{\dot{\rho}(\bvec{r'}, t_r) \scriptr}{c \scriptr^2} \uvec{\boldscriptr} 
			\right] \dd{\tau'}\\
			 & = \frac{1}{4 \pi \epsilon_0} \int \left[
\frac{c\rho(\bvec{r'}, t) + \dot{\rho}(\bvec{r'}, t_r) \scriptr}{c\scriptr^2}
\right] \uvec{\boldscriptr}  \dd{\tau'}\\
			 & = \frac{1}{4 \pi \epsilon_0} \int \left[
\frac{\rho(\bvec{r'}, t) + (t_r - t)\dot{\rho}(\bvec{r'}, t_r) \scriptr}{\scriptr^2}
\right] \uvec{\boldscriptr}  \dd{\tau'}\\
			 & = \frac{1}{4 \pi \epsilon_0} \int \left[
\frac{\rho(\bvec{r'}, t) + (t - t_r)\dot{\rho}(\bvec{r'}, t_r) \scriptr}{\scriptr^2}
\right] \uvec{\boldscriptr}  \dd{\tau'}
\intertext{By the $\rho$ given in the problem,}
				 \bvec{E}(\bvec{r}, t)& = \frac{1}{4 \pi \epsilon_0} \int \left[
	\frac{\rho(\bvec{r}', t)}{\scriptr^2}
	-\frac{ t_r{\dot{\rho}(\bvec{r}', t_r)} }{\scriptr^2}
	\right] \uvec{\boldscriptr}  \dd{\tau'}\\
		\end{align*}
	Not really sure what to do with that extra $t_r$ term here...
	\end{enumerate}
\end{document}