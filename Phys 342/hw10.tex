\documentclass{homework}

\title{Homework 10}
\author{Kevin Evans}
\studentid{11571810}
\date{April 21, 2021}
\setclass{Physics}{342}
\usepackage{amssymb}
\usepackage{mathtools}

\usepackage{amsthm}
\usepackage{amsmath}
\usepackage{slashed}
\usepackage{relsize}
\usepackage{threeparttable}
\usepackage{float}
\usepackage{booktabs}
\usepackage{boldline}
\usepackage{changepage}
\usepackage{physics}
\usepackage[inter-unit-product =\cdot]{siunitx}
\usepackage{setspace}

\usepackage[makeroom]{cancel}
%\usepackage{pgfplots}

\usepackage{enumitem}
\usepackage{times}
\usepackage{mhchem}

\usepackage{calligra}
\DeclareMathAlphabet{\mathcalligra}{T1}{calligra}{m}{n}
\DeclareFontShape{T1}{calligra}{m}{n}{<->s*[2.2]callig15}{}
\newcommand{\scriptr}{\mathcalligra{r}\,}
\newcommand{\boldscriptr}{\pmb{\mathcalligra{r}}\,}
\newcommand{\emf}{\mathcal{E}}

\begin{document}
	\maketitle
	\begin{enumerate}
		\item \begin{enumerate}
			\item The total energy lost due to bremsstrahlung (assuming $\gamma=1$) is \begin{align*}
				E_\mathrm{brem} & = \int_0^\infty P \dd{t} \\
				& = \frac{\mu_0 q^2 a^2}{6 \pi c} \int_0^{v_0 / a}  \dd{t} \\
				& = \frac{\mu_0 e^2 a v_0}{6 \pi c}
			\end{align*}
			As a fraction of the initial kinetic energy, \begin{align*}
				\frac{ E_\mathrm{brem} }{E_\mathrm{KE}} & = \frac{2 \mu_0 e^2 a}{6 \pi c m_e v_0^2}
			\end{align*}
		
			\item From the classical equations of motion, \begin{align*}
				v_0^2 & = 2 a x \\
				\left(10^5 \quad \si{\m\per\s}\right)^2 & = 2 \left(\SI{3.0}{\nm}\right) a \\
				a & = \SI{1.67e19}{\m\per\s\squared} \\
				\frac{E_\mathrm{brem}}{E_\mathrm{KE}} & = \num{2.2e-15}
			\end{align*}
			The loss is quite small and can be ignored.
		\end{enumerate}
	
		\item Since we're doing everything classically, \begin{align*}
			E_\mathrm{Coulomb} & = E_\mathrm{KE} \\
			\frac{1}{4 \pi \epsilon_0} \frac{e^2}{r_0} & = \frac{1}{2} m_e v^2 \\
			v & = \sqrt{\frac{e^2}{2 \pi \epsilon_0 m_e r_0}} \\
				& \approx \SI{3.1e6}{\m\per\s} \approx 0.011c
		\end{align*}
		From the Larmor formula, the power radiated is \begin{align*}
			P & = \frac{\mu_0 q^2 a^2}{6 \pi c}
			\intertext{As $P = \dv{E}{t}$, }
			P & = \dv{t} \frac{1}{4 \pi \epsilon_0} \frac{e^2}{r} = - \frac{1}{4 \pi \epsilon_0} \frac{e^2}{r^2} \dv{r}{t} = \frac{\mu_0 q^2 a^2}{6 \pi c}  \\
			-\int \frac{1}{4 \pi \epsilon_0} \frac{e^2}{r^2} \dd{r} & = \int \frac{\mu_0 q^2 (v^2 / r)^2}{6 \pi c} \dd{t} \\
		\end{align*}
	
		\item \begin{enumerate}
			\item The damping factor $\gamma$ is given by (11.84). For some visible light, $\omega = 10^{15}$ rad/s, \begin{align*}
				\gamma &  = \omega^2 \tau \\
					& = \left(\SI{6e-24}{\s}\right) \left(10^{15}\right)^2 \\
					& = \num{6e6} \\
				\gamma & \ll \omega_0
			\end{align*}
		
			\item I'm not really sure what to do here, but from the discussion in-class, \begin{align*}
				F_\mathrm{spring} & = F_\mathrm{Coulomb} \\
				m \omega^2 x & = \frac{q^2}{4 \pi \epsilon_0 x} \\
				\omega & = \sqrt{\frac{q^2}{4 \pi \epsilon_0 x^2 m}}
			\end{align*}
		\end{enumerate}
	
		\item \begin{enumerate}
			\item On each end of the dumbbell, there's $q/2$ charge. From the Abraham-Lorentz formula, \begin{align*}
				F_\mathrm{rad} & = \frac{\mu_0 q^2}{24 \pi c} \dot{a}
			\end{align*}
			Adding this to the interaction term results in the expected $F_\mathrm{rad}$, \begin{align*}
				F_\mathrm{rad}& = 2 \times \frac{\mu_0 q^2}{24 \pi c} \dot{a} + \frac{\mu_0 q^2 \dot{a}}{12 \pi c} = \frac{\mu_0 q^2 \dot{a}}{6 \pi c}
			\end{align*}
		
			\item 
		\end{enumerate}
	
		\item For one charge, the average intensity is \begin{align*}
			I & = \frac{\mu_0 \ddot{p}^2}{6 \pi c} \frac{\sin[2](\theta)}{r^2}
			\intertext{The total power over both the charge and its image is then}
			P & = \int I \dd{a} \\
				& = 2 \int_0^{2 \pi} \int_0^\pi \frac{\mu_0 \ddot{p}^2}{6 \pi c} \sin[3](\theta) \dd{\theta} \dd{\phi} \\
				& = \frac{8 \mu_0 \ddot{p}^2}{9 c} \\
				& = \frac{8 \mu_0 q \ddot{z}^2}{9 c}?
		\end{align*}
	\end{enumerate}
\end{document}