\documentclass{homework}

\title{Homework 9}
\author{Kevin Evans}
\studentid{11571810}
\date{November 4, 2020}
\setclass{Physics}{341}
\usepackage{amssymb}
\usepackage{mathtools}

\usepackage{amsthm}
\usepackage{amsmath}
\usepackage{slashed}
\usepackage{relsize}
\usepackage{threeparttable}
\usepackage{float}
\usepackage{booktabs}
\usepackage{boldline}
\usepackage{changepage}
\usepackage{physics}
\usepackage[inter-unit-product =\cdot]{siunitx}
\usepackage{setspace}

\usepackage[makeroom]{cancel}
%\usepackage{pgfplots}

\usepackage{enumitem}
\usepackage{times}

\usepackage{calligra}
\DeclareMathAlphabet{\mathcalligra}{T1}{calligra}{m}{n}
\DeclareFontShape{T1}{calligra}{m}{n}{<->s*[2.2]callig15}{}
\newcommand{\scriptr}{\mathcalligra{r}\,}
\newcommand{\boldscriptr}{\pmb{\mathcalligra{r}}\,}

\begin{document}
	\maketitle
	\begin{enumerate}
		\item \begin{enumerate}
			\item The bound surface charge density is $\mp k$ at $a$ and $b$. The bound volume charge is taken from the divergence, \begin{align*}
				\rho_b & = -\div{\bvec{P}} = -k / s
			\end{align*}
			Within the dielectric, the total bound charge at radius $s$ is \begin{align*}
				Q_b & = -k(2 \pi a l) + \int_V \rho_b \dd{\tau} \\
					& = -2\pi k a l + 2 \pi l \int_a^s -k \dd{s} \\
					& = -2 \pi k l \left(a + s - a\right) \\
					& = -2 \pi k l s
				\intertext{From Gauss's law,}
				E (2 \pi s l) & = -2 \pi l k s / \epsilon_0 \\
				\bvec{E}	& = - \frac{k}{\epsilon_0} \uvec{s}
			\end{align*}
		
			\item Since there is no free charge, $\bvec{D} = 0$. Then, \begin{align*}
				\bvec{E} & = -\frac{1}{\epsilon_0} \bvec{P} \\
					& = -\frac{k}{\epsilon_0} \uvec{s}
			\end{align*}
%			\item From the polarization $k \uvec{s}$, the bound surface charge is $\sigma = \pm k$ at $a$ and $b$. From Gauss's law, the electric field within the dielectric is \begin{align*}
%				\bvec{E} & = -\frac{ka}{\epsilon_0 s} \uvec{s}
%			\end{align*}
%			\item Since there is no free charge, the displacement field is zero and the electric field (within the dielectric) can be written as \begin{align*}
%				\bvec{E}_\mathrm{in} & = -\frac{1}{\epsilon_0} \bvec{P} = -\frac{k}{\epsilon_0} \uvec{s}
%			\end{align*}
%			Outside of the dielectric ($r < a$ or $r > b$), there is no electric field, $\bvec{E}_\mathrm{out} = 0$.
		\end{enumerate}
	
		\item \begin{enumerate}
			\item In each dielectric, the displacement field points downward and the only free charge is from the plates,
				\[ \bvec{D} = -\sigma \uvec{z} \]
			
			\item Since it's a linear dielectric, \begin{align*}
				\bvec{E}_1 & = \frac{1}{\epsilon} \bvec{D} = -\frac{\sigma}{2\epsilon_0} \uvec{z} \\
				\bvec{E}_2 & = -\frac{\sigma}{3\epsilon_0} \uvec{z}
			\end{align*}
		
			\item From $\bvec{D} = \epsilon_0 \bvec{E} + \bvec{P}$, \begin{align*}
				\bvec{P}_1 & = \bvec{D} - \epsilon_0 \bvec{E}_1 \\
					& = \left( -\sigma + \frac{\sigma}{2}\right) \uvec{z} \\
					& = -\frac{\sigma}{2} \uvec{z} \\
				\bvec{P}_2 & = \left(-\sigma +  \sigma / 3\right) \uvec{z} \\
					& = -\frac{2\sigma}{3} \uvec{z}
			\end{align*}
		
			\item The potential difference can be found by integrating the electric fields along a path between the two plates, \begin{align*}
				V & = \int_{0 \to a} \bvec{E} \cdot \dd{\bvec{l}} \\
					& = \int_{0}^{3a / 4} \left(-\sigma / 3 \epsilon_0\right) \dd{z} + \int_{3a/4}^a \left(-\sigma / 2 \epsilon_0\right) \dd{z} \\
					& = \frac{-\sigma}{\epsilon_0} \left[ \frac{1}{3} \left(3a/4\right)
						+ \frac{1}{2} \left(a - 3a / 4\right)
					 \right] \\
					& = -\frac{3\sigma a}{8 \epsilon_0}
			\end{align*}
		
			\item The bound charges can be found from the polarization. In each dielectric, the upper bound charges will be negative, and the lower will have positive, \begin{align*}
				\sigma_b & \equiv \bvec{P} \cdot \uvec{n} \\
				\sigma_{b1} & = \pm \frac{\sigma}{2} \\
				\sigma_{b2} & = \pm \frac{2 \sigma}{3}
			\intertext{For the volume charges, they are all zero as the polarization is uniform throughout the dielectrics}
				\rho_b & \equiv -\div{\bvec{P}} = 0
			\end{align*}
		
			\item On each surface, the total surface charge density is $\sigma_\mathrm{tot} = \sigma_b + \sigma_f$. From Gauss's law, for the dielectric layers, the electric fields are \begin{align*}
				\int \bvec{E}_1 \cdot \dd{\bvec{a}} & = \frac{1}{\epsilon_0} \int \sigma_\mathrm{tot} \dd{a} \\
				\bvec{E}_1 & = \frac{1}{\epsilon_0} \left[ \sigma - \frac{\sigma}{2}\right] (-\uvec{z}) \\
					& = -\frac{\sigma}{2 \epsilon_0} \uvec{z} \\
				\bvec{E}_2 & = \frac{1}{\epsilon_0} \left[ -\sigma + \frac{2\sigma}{3}  \right] (-\uvec{z}) = \frac{\sigma}{3 \epsilon_0} \uvec{z} ?
			\end{align*}
		\end{enumerate}

		\item From Gauss's law, within the sphere \begin{align*}
			D (4 \pi r^2) & = \int_0^r \rho_f \dd{\tau} \\
				& =  4 \pi k \int_0^r  r^3 \dd{r} \\
				& = \pi k r^4 \\
			\bvec{D} & = \frac{ kr^2 }{4} \uvec{r}
			\intertext{The electric field then follows}
			\bvec{E} & = \begin{cases*}
				\frac{kr^2}{4 \epsilon_r \epsilon_0} \uvec{r} & $r < R$ \\
				\frac{kR^4}{4 \epsilon_0 r^2} \uvec{r} & $r > R$
			\end{cases*}
			\intertext{And the potential at the center from infinity,}
			V & = -\int_\infty^0 \bvec{E} \cdot \dd{\bvec{l}} = - \left[\frac{kR^4}{4 \epsilon_0}\int_\infty^R r^{-2} \dd{r}
				+ \frac{k}{4 \epsilon_r \epsilon_0} \int_R^0 r^2 \dd{r}
			 \right] \\
			 	& = -\left[ -\frac{kR^4}{4 \epsilon_0} \left(R^{-1} - 0\right)
			 		+ \frac{k}{12 \epsilon_r \epsilon_0} \left( 0 - R^3  \right)
			 	\right] \\
			 	& = \frac{k R^3}{4\epsilon_0} \left(1 + \frac{1}{3 \epsilon_r}\right)
		\end{align*}
	
		\item From the in-class discussion, if we let $\bvec{E} = E_0 \uvec{z}$ and apply the boundary conditions: \begin{align*}
			V(r \le a) & = 0 \\
			V_\mathrm{in}(r = b) & = V_\mathrm{out}(r=b) \\
			V_\mathrm{out}(r \to \infty) & = -E_0 r \cos \theta \\
			\epsilon_0 \pdv{V_\mathrm{out}}{r} - \epsilon \pdv{V_\mathrm{in}}{r} & = 0
			\intertext{In the dielectric layer, we can assume the general form of the potential of}
			V_\mathrm{in}(r, \theta) & = \sum_l \left[ A_\mathrm{in, l} r^l + \frac{ B_\mathrm{in, l}}{r^{l+1}} \right] P_l(\cos \theta)
			\intertext{At $r = a$, this potential must be zero (from the first BC) and it's found that}
			B_\mathrm{in, l} & = A_\mathrm{in, l} a^{2l + 1}
			\intertext{Outside of the sphere, we can assume the potential to have the form}
			V_\mathrm{out}(r \to \infty, \theta) & = -E_0 r \cos \theta = (-E_0) r P_1(\cos \theta)
			\intertext{Only the $l=1$ term remains for the $A_l$ values (but the $B_l$ values can still be non-zero?)}
			V_\mathrm{out}(r) & = -E_0 r \cos \theta + \sum_l \frac{B_\mathrm{out,l}}{r^{l+1}} P_l(\cos \theta)
			\intertext{For the BC at $r=b$, the potentials are continuous and we can find the relation between $B_\mathrm{out}$ and $A_\mathrm{in}$}
			V_\mathrm{in}(b, \theta) & = V_\mathrm{out}(b, \theta) \\
			\sum_l \left[A_\mathrm{in} b^l + \frac{A_\mathrm{in} a^{2l + 1}}{b^{l+1}}\right]  P_l(\cos \theta) & = -E_0 b \cos \theta + \sum_l \frac{B_\mathrm{out}}{b^{l+1}} P_l (\cos \theta) \\
			A_\mathrm{in} b + \frac{A_\mathrm{in} a^3}{b^2} & = -E_0 b \cos \theta  + \frac{ B_\mathrm{out} }{b^2} \\
			B_\mathrm{out} & = A_\mathrm{in} \left(b^3 + a^3\right) + E_0 b^3
		\end{align*}
		From the last BC, we can expect the normal derivatives of the potential to be discontinuous by the free charge (evaluated at $r=b$). But since there's no free charge, we can equate \begin{align*}
			\epsilon_0 \pdv{V_\mathrm{out}}{r} & = \epsilon \pdv{V_\mathrm{in}}{r} \\
			\epsilon_0 \cos \theta \left[ -E_0 - 2 B_\mathrm{out} r^{-3}   \right]_{b}  & = \epsilon \cos \theta \left[ A_\mathrm{in} - 2 A_\mathrm{in} a^{3} r^{-3} \right] \\
			\epsilon_0  \left[ -E_0 - 2 \left(A_\mathrm{in} \left(b^3 + a^3\right) + E_0 b^3\right) r^{-3}   \right]_{b}  & = \epsilon  \left[ A_\mathrm{in}  - 2 A_\mathrm{in} a^{3} r^{-3} \right]_b \\
			A_\mathrm{in} \left[\epsilon - 2\epsilon a^3 b^{-3} + 2 \epsilon_0 (b^3 + a^3) b^{-3} \right] & = -3\epsilon_0 E_0 \\
			A_\mathrm{in} \epsilon_0 \left[ \epsilon_r + \frac{2a^3}{b^3} (1 - \epsilon_r) \right] & = -3 \epsilon_0 E_0 \\
			A_\mathrm{in} & = -\frac{ 3 E_0 b^3 }{\epsilon_r b^3 + 2a^3 (1-\epsilon_r) }
		\end{align*}
		Putting this all together, the potential (with only the $l=1$ term) and electric field inside will be \begin{align*}
			V_\mathrm{in} & =  -\frac{ 3 E_0 b^3 }{\epsilon_r b^3 + 2a^3 (1-\epsilon_r) } \left[ r + \frac{ a^{3} }{r^2} \right] \cos \theta \\
			\bvec{E} & = -\grad{V} \\
				& = \frac{ 3 E_0 b^3 }{\epsilon_r b^3 + 2a^3 (1-\epsilon_r) } \left[
					\left(1 - 2\frac{ a^3 }{r^3} \right) \cos\theta \uvec{r} 
					- \left(1 + \frac{ a^3 }{r^3} \right) \sin \theta \uvec{\theta}
				\right]
		\end{align*}
		
		\item From the charged sphere, its electric field is only non-zero outside of $a$, and it's \[ \bvec{E} = \frac{Q}{4 \pi \epsilon_0 r^2} \uvec{r} \]
		Through the dielectric and outside of it, the displacement is \begin{align*}
			\bvec{D} & = \epsilon \bvec{E} \\
				& = \begin{cases*}
					\frac{ (1 + \chi_e) Q }{4 \pi  r^2} \uvec{r} & $a < r < b$\\
					\epsilon_0 \bvec{E} & $r > b$
				\end{cases*}
		\end{align*}
		Integrating over all space, \begin{align*}
			W & = \frac{1}{2} \int \bvec{D} \cdot \bvec{E} \dd{\tau} \\
				& = \frac{1}{2} \left[ \frac{(1 + \chi_e)Q^2}{16 \pi^2 \epsilon_0} (4 \pi) \int_a^b r^{-4} \dd{r} + \frac{Q^2}{16 \pi^2 \epsilon_0} (4 \pi) \int_b^\infty r^{-4} \dd{r} \right] \\
				& = \frac{1}{2} \left(\frac{Q^2}{4 \pi \epsilon_0}\right) \left(\frac{1}{3}\right) \left[
				(1 + \chi_e) \left(\frac{1}{a^3} - \frac{1}{b^3} \right)
				+ \frac{1}{b^3}
				\right] \\
				& = \frac{Q^2}{24 \pi \epsilon_0} \left(\frac{1 + \chi_e}{a^3} - \frac{\chi_e}{b^3}\right)
		\end{align*}
	\end{enumerate}
\end{document}