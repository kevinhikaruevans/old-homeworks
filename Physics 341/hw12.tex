\documentclass{homework}

\title{Homework 12}
\author{Kevin Evans}
\studentid{11571810}
\date{December 9, 2020}
\setclass{Physics}{341}
\usepackage{amssymb}
\usepackage{mathtools}

\usepackage{amsthm}
\usepackage{amsmath}
\usepackage{slashed}
\usepackage{relsize}
\usepackage{threeparttable}
\usepackage{float}
\usepackage{booktabs}
\usepackage{boldline}
\usepackage{changepage}
\usepackage{physics}
\usepackage[inter-unit-product =\cdot]{siunitx}
\usepackage{setspace}

\usepackage[makeroom]{cancel}
%\usepackage{pgfplots}

\usepackage{enumitem}
\usepackage{times}

\usepackage{calligra}
\DeclareMathAlphabet{\mathcalligra}{T1}{calligra}{m}{n}
\DeclareFontShape{T1}{calligra}{m}{n}{<->s*[2.2]callig15}{}
\newcommand{\scriptr}{\mathcalligra{r}\,}
\newcommand{\boldscriptr}{\pmb{\mathcalligra{r}}\,}

\begin{document}
	\maketitle
	\begin{enumerate}
		\item \begin{enumerate}
			\item From the cylinder, its magnetic field is \begin{align*}
				\int \bvec{B} \cdot \dd{\bvec{l}} & = \mu_0 I_\mathrm{enc} \\
					& = \mu_0 \int \bvec{J} \cdot \dd{\bvec{a}} \\
					& = \mu_0 J_0 \pi R^2 \\
				\bvec{B} & = \frac{ \mu_0 J_0 R^2 }{2 s}  \uvec{\phi}\\
			\end{align*}
			Then the torque on the dipole becomes \begin{align*}
				\bvec{N} & = \bvec{m} \cross \bvec{B} \\
					& = m_0 \frac{ \mu_0 J_0 R^2 }{2 s}  (\uvec{z} \cross \uvec{\phi}) \\
					& = - m_0 \frac{ \mu_0 J_0 R^2 }{2 s} \uvec{s}
			\end{align*}
		
			\item $\bvec{F} = 0$, since $\bvec{m} \perp \bvec{B}$.
			
			\item It's not true for the magnetic analogs because it requires that $\curl{\bvec{E}} = 0 \iff \curl{\bvec{B}} = 0$, but this is not true. It's only true in the electostatic case, but for magnetics, this is equal to $\mu_0 \bvec{J}$.
		\end{enumerate}
	
		\item \begin{enumerate}
			\item For the bound volume current, \begin{align*}
				\bvec{J}_b & = \curl{\bvec{M}} \\
					& = 2k \uvec{z}
				\intertext{From Ampere's law, $\oint B \dd{l} = \int J \dd{a}$}
				B & = \begin{cases}
					\mu_0 k s & s < R \\
					\frac{ \mu_0 k R^2 }{s} & s > R
				\end{cases}
			\end{align*}
		
			And the surface charge, $\bvec{K}_b  = \bvec{M} \cross \uvec{n}$, but the normal direction ($\uvec{s}$) is perpindicular to $\bvec{M}$, so there is no bound surface current. So the $B$ contribution is zero.
			
			\item There is no free current, so $\bvec{H} = 0$? Then the magnetic field is \begin{align*}
				B & = \mu\bvec{H} + \mu\bvec{M} \\
					& = \begin{cases}
						\mu k s & s < R \\
						0? & s > R
					\end{cases}
			\end{align*}
		\end{enumerate}
	
		\item Between the tubes, the free current enclosed is $I$, so the azimuthal $H$ and $B$ field is \begin{align*}
			H & = \frac{I}{2 \pi s} \\
			B & = \frac{\mu_0 (1 + \chi_m)I}{2 \pi s}
		\end{align*}
		 	To check, the magnetization is \begin{align*}
		 		\bvec{M} & = \chi_m \bvec{H} \\
		 		 & = \frac{ \chi_m I }{2 \pi s} \uvec{\phi}
		 		\intertext{The bound currents are then}
		 		\bvec{J}_b & = \curl{\bvec{M}} = 0 \\
		 		\bvec{K}_b & = \bvec{M} \cross (\pm \uvec{s}) \\
		 			& = \begin{cases}
		 				\frac{\chi_m I}{2 \pi a} \uvec{z} & s=a \\
		 				\frac{\chi_m I}{2 \pi b} \uvec{z} & s=b
		 			\end{cases} 
	 			\intertext{Between the cylinders, the total current is then}
	 			I_\mathrm{enc} & = I + \int K_b \dd{l} \\
	 				& = I  + \frac{\chi_mI}{2 \pi a} (2 \pi a) = I (1 + \chi_m) 
		 	\end{align*}
	 		From Ampere's law, the magnetic field is \begin{align*}
	 			B & = \frac{\mu_0 I(1 + \chi_m)}{2 \pi s}
	 		\end{align*}
		\item For a (free) current $J_z=ks$, within the wire, the $H$ field is\begin{align*}
			\bvec{H} & =  \frac{ k }{2 \pi s} \int_0^s s^2 \dd{\phi} \uvec{\phi} \\
				& = \frac{k}{2 \pi s} \left(\frac{ 2 \pi s^3 }{3}\right)\uvec{\phi} \\
				& = \frac{ks^2}{3} \uvec{\phi}
			\intertext{Outside the wire, it's the same thing but bounded at $s=a$,}
			\bvec{H} & = \begin{cases}
				\frac{ks^2}{3}\uvec{\phi} & s < a \\
				\frac{ka^3}{3s}\uvec{\phi} & s > a
			\end{cases}
			\intertext{As it's a linear medium,}
			\bvec{B} & = \begin{cases}
			\mu_0 (1 + \chi_m)	\frac{ks^2}{3}\uvec{\phi} & s < a \\
							\mu_0\frac{ka^3}{3s}\uvec{\phi} & s > a
			\end{cases}
		\end{align*}
		For the bound charges, the magnetization $\bvec{M} = \chi_m \bvec{H}$, so \textit{within }the medium,\begin{align*}
			\bvec{J}_b & = \curl{\bvec{M}} = 2k/3 \uvec{z} \\
			\bvec{K}_b & = \bvec{M} \cross \uvec{n} \\
				& = -\frac{ks^2}{3} \uvec{z}
		\end{align*}
		\item Heating it up to its Curie temperature.
	\end{enumerate}
\end{document}