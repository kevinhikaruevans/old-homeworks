\documentclass{homework}

\title{Lab 9}
\author{Kevin Evans}
\studentid{11571810}
\date{April 11, 2020}
\setclass{EE}{352}
\usepackage{amssymb}
\usepackage{mathtools}

\usepackage{amsthm}
\usepackage{amsmath}
\usepackage{slashed}
\usepackage{relsize}
\usepackage{threeparttable}
\usepackage{float}
\usepackage{booktabs}
\usepackage{boldline}
\usepackage{changepage}
\usepackage{physics}
\usepackage[inter-unit-product =\cdot]{siunitx}
\usepackage{setspace}

\usepackage[makeroom]{cancel}
\usepackage{pgfplots}

\usepackage{multicol}
\usepackage{tcolorbox}
\usepackage{enumitem}
\usepackage{times}
\usepackage{mhchem}
\usepackage{graphicx} 
\DeclareSIUnit{\year}{yr}

\begin{document}
	\maketitle
	\vspace{-1em}
	\subsection*{Experiment 1}
	\vspace{-0.2em}
	In this experiment, the 2N3904 BJT was simulated in LTSPICE and the $i_c$-$v_{CE}$ characteristic curves were obtained. From these curves, the current gain $\beta$ was obtained at various collector-emitter voltages and base currents. The data is shown in Table \ref{table:exp1}. At $V_{CE}=1$ V, the expected gain $h_{FE}$ has a minimum of 70 (1 mA) and 100 (10 mA) and a maximum of 300. In the simulation, a current gain of just under 300 was calculated. At $V_{CE}=10$ V, the gain exceeded 300, however an $h_{FE}$ value was not listed for this voltage.
	\vspace{-1em}
	\begin{table}[H]
		\centering
		\caption{Data collected from the $i_c$-$v_{CE}$ curves and the current gains calculated.}
		\label{table:exp1}
		\begin{tabular}{@{}ccccc@{}}
			\toprule
			$V_{CE}$ & Approx. $I_C$ (mA) & $I_C$ (mA) & $I_B$ (uA) & $\beta$ (A/A) \\ \midrule
			1   & 3                       & 2.9872372       & 10          & 298.72372  \\
			1   & 12                      & 11.692551       & 40          & 292.313775 \\
			10  & 3                       & 3.2552434       & 10          & 325.52434  \\
			10  & 12                      & 12.741939       & 40          & 318.548475 \\ \bottomrule
		\end{tabular}
	\end{table}

	\noindent For small AC signals, the gain was found to be $\beta=320.7$. This was determined by fixing $V_{CE} = 10$ V and measuring the collector current $I_C$ at the base currents of $10$ and $20$ mA, \begin{align*}
		\beta_{ac} & = \eval{ \frac{\Delta I_C}{\Delta I_B} }_{V_{CE} = 10} \\
			& = \frac{6.4623146 - 3.2552434 \text{ mA}}{10 \text{ uA}} = 320.71
	\end{align*}
	At the point $(V_{ce}, I_c) \approx (10 \text{ V}, 12 \text{ mA})$, the output resistance was determined using the slope, \begin{align*}
		\text{Slope} & = \frac{12.747768 - 12.736109\; \si{\mA}}{10.05 - 9.95 \; \si{\V}} \\
		r_o & = \SI{8576.92}{\ohm}
	\end{align*}
	The early voltage $V_A$ was determined using this $r_o$, \begin{align*}
		V_A & = r_o I_c - V_{CE} \\
			& = \SI{8.577}{\kohm} \times \SI{12.74}{\mA} - \SI{10}{\V} \\
			& = \SI{99.28}{\V}
	\end{align*}
	Next, the collector-emitter terminals of the transistor were reversed and the reverse current gain $\beta_r$ was calculated at $V_{CE} = 4$ V and at a base current $I_B = \SI{10}{\uA}$, \begin{align*}
		\beta_r & = \frac{39.77}{10} = 3.98
	\end{align*}
	\vspace{-2em}
	\subsection*{Experiment 2}
	\vspace{-0.2em}
	The transistor was attached in a diode configuration, with the base and collector shorted together. Around a base current $I_B = \SI{20}{\uA}$, the slope was measured using two data points and the resistance $r_\pi$ was calculated, \begin{align*}
		\text{Slope} & = \frac{20.032921 - 19.958448 \; \si{\uA}}{702.5 - 702.4 \; \si{\mV}} = \SI{0.000744731}{\A/\V} \\
		r_\pi & = \SI{1342.77}{\ohm}
	\end{align*}
\end{document}