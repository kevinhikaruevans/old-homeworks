\documentclass{homework}

\title{Homework 1}
\author{Kevin Evans}
\studentid{11571810}
\date{September 4, 2021}
\setclass{CptS}{466}
\usepackage{amssymb}
%\usepackage{mathtools}
\usepackage{graphicx}
\usepackage{amsthm}
\usepackage{amsmath}
\usepackage{slashed}
\usepackage{boldline}
\usepackage{physics}
\usepackage[inter-unit-product =\cdot]{siunitx}

\usepackage[makeroom]{cancel}
\usepackage{booktabs}

\usepackage{times}
\usepackage{mhchem}

%\usepackage{calligra}
%\DeclareMathAlphabet{\mathcalligra}{T1}{calligra}{m}{n}
%\DeclareFontShape{T1}{calligra}{m}{n}{<->s*[2.2]callig15}{}
%\newcommand{\scriptr}{\mathcalligra{r}\,}
%\newcommand{\boldscriptr}{\pmb{\mathcalligra{r}}\,}
%\newcommand{\emf}{\mathcal{E}}

\begin{document}
	\maketitle
	\begin{enumerate}
		\item \begin{itemize}
				\item Microcomputer: a small computer with a microprocessor and other I/O within it, what you'd typically think of when you say ``computer.''
				\item Microprocessor: a chip that executes instructions.
				\item Microcontroller: a chip that contains a microprocessor, memory, I/O ports.
			\end{itemize}
		
		\item An embedded system is a computer with a specific purpose, usually for real-time computing. Typically, the parts are minimized for a specific purpose and extraneous hardware is stripped out. 
		
		\item Digital watches, traffic lights, automotive, avionics, industrial (PLCs, etc). 
		
		\item Switches, USB, keyboard, touchscreen.
		
		\item Displays, USB, audio output.
		
		\item Flash ROM has a higher density, since RAM requires refresh lines and ROM does not.
		
		\item \begin{enumerate}[label={\arabic*.}]
			\item C
			\item D
			\item A
			\item B
			\item E
		\end{enumerate}
			
		\item \begin{itemize}
			\item Bit: 1
			
			\item Nibble: 4
			
			\item Byte: 8
			
			\item Hword: 16
			
			\item Word: 32
		\end{itemize}
	
		\item Unsigned numbers use all the bits for the magnitude of a number. A signed number has a specific bit to represent the sign of the number. In two's complement, for a negative number, the magnitude is inverted and $1$ is added to the number to represent the signed number. 
		
		\item \texttt{0100 1110 0001b}
		
			\texttt{2341h}
			
			\texttt{0x4E1}
			
			For a 32-bit sign-extended binary, \texttt{0000 $\dots$ 0100 1110 0001b}
			
		\item Inverting $1249$, we get $\texttt{1111 \dots 1011 0001 1110b}$. Then adding one, $\texttt{1111 \dots 1011 0001 1111b}$.
		
		\item In sign-magnitude, \texttt{1000 $\dots$ 0100 1110 0001b}.
		
		\item \begin{itemize}
			\item $-10+12$: \begin{align*}
				-10 & = (\neg\texttt{10b} + 1) = \neg 1111010_2 + 1 \\
					& = 11110110_2 \\
				12 & =  00001100_2 \\
				-10 + 12 & = 0000 0010 = 2?
			\end{align*}
			I think this overflowed because it has an extra carry bit.
		
			\item $-20+19$: \begin{align*}
				-20 & = \neg 0001 0100_2 + 1 = 1110 1011_2 + 1 \\
					& = 1110 1100_2 \\
				19 & =  0001 0011_2 \\
				-20 + 19 & = 11111111_2 \\
					& = -(0000 0000_2 + 1_{10}) = -1
			\end{align*}
		
			\item $10000000_h + 10000000_h$: \begin{align*}
				0001\dots0000 + 0001\dots0000 & = 0010\dots0000 \\
					& = 20000000_h
			\end{align*}
		\end{itemize}
	\end{enumerate}
\end{document}