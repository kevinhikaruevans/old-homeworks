\documentclass{homework}

\title{Homework 2}
\author{Kevin Evans}
\studentid{11571810}
\date{January 27, 2021}
\setclass{Physics}{465}
\usepackage{amssymb}
\usepackage{mathtools}
\usepackage{graphicx}
\usepackage{amsthm}
\usepackage{amsmath}
\usepackage{slashed}
\usepackage{boldline}
\usepackage{physics}
\usepackage{tcolorbox}
\usepackage[inter-unit-product =\cdot]{siunitx}

\usepackage[makeroom]{cancel}
\usepackage{booktabs}
\usepackage{multirow}

\usepackage{times}
\usepackage{mhchem}
\usepackage{enumitem}
\usepackage[normalem]{ulem}
\usepackage{systeme}
\usepackage{tikz}
\usepackage{mathtools}
\usepackage{tabularx}

%\usepackage{calligra}
%\DeclareMathAlphabet{\mathcalligra}{T1}{calligra}{m}{n}
%\DeclareFontShape{T1}{calligra}{m}{n}{<->s*[2.2]callig15}{}
%\newcommand{\scriptr}{\mathcalligra{r}\,}
%\newcommand{\boldscriptr}{\pmb{\mathcalligra{r}}\,}
%\newcommand{\emf}{\mathcal{E}}

\DeclareSIUnit\eVperc{\eV\per\clight}
\DeclareSIUnit\clight{\text{\ensuremath{c}}}
\DeclareSIUnit\year{yr}

\newcommand{\fm}{\femto\meter}

\begin{document}
	\maketitle
	\begin{enumerate}
		\item \begin{enumerate}
			\item Because the probability involves $\psi_0(t)^* \psi_0(t)$, where the imaginary part cancels out. We're then left with \begin{align*}
				P(t) & = \abs{ \psi_0^* \psi_0 }^2 \\
					& = e^{-\Gamma t / \hbar} .
			\end{align*}
			Taking this at $t=0$, we see that $P(t)=1$. This makes sense as the state is definite at the start time with a probability of surely existing.
			
			The decay constant is $\alpha = \Gamma/\hbar$.
			
			\item \begin{align*}
				g(\omega) & = \int_0^\infty e^{iE_0t / \hbar - \frac{1}{2} \Gamma t / \hbar + i \omega t} \dd{t} \\
					& = \int_0^\infty e^{i (E_0 + E) t / \hbar - \frac{1}{2} \Gamma t / \hbar } \dd{t} \\
					& = \frac{2 \hbar}{\Gamma - 2i(E + E_0)}. && \text{(WolframAlpha)}
			\end{align*}
		
			\item The probability is given by $g(\omega)^* g(\omega)$, so \begin{align*}
				 \frac{2 \hbar}{\Gamma + 2i(E + E_0)}  \frac{2 \hbar}{\Gamma - 2i(E + E_0)} & = \frac{4 \hbar^2}{\Gamma^2 - 4(E+E_0)^2} \\
				 & = \frac{\hbar^2}{(\Gamma/2)^2 - (E + E_0)^2}?
			\end{align*}
		\end{enumerate}
		
		\item % 2.3 8e-13
			For an ideal gas at STP, one liter contains \begin{align*}
					n & = \frac{PV}{RT} = \frac{ \SI{1}{\bar} \times \SI{1}{\L} }{ \SI{0.083145}{\L \bar \per \mol \per \K} \times \SI{273}{\K}} \\
						& = \SI{0.044}{\mol}.
			\intertext{Then for the atoms of carbon,}
			N_\mathrm{carbon} & = n N_A = \SI{0.044}{\mol} \times \SI{6.022e23}{\per \mol} \\
				& = \SI{2.65e22}{atoms}.
			\intertext{The transition rate per second is }
			\omega & = \tau^{-1} \\
				& = 1/\left(
					\SI{8267}{\year}
					\times \SI{365}{\day \per \year}
					\times \SI{86400}{\s\per\day}
				\right) \\
				& = \SI{3.835e-12}{\per\s}.
			\end{align*}
			Then at 5 disintegrations per minute, the atomic fraction of nitrogen to carbon can be found with \begin{align*}
				\dot{N} & = -\omega N \\
				\frac{N_{N-14}}{N_\mathrm{C}} & = \dot{N}_\mathrm{N} / \omega  N_\mathrm{C} \\
					& = \num{8.2e-13}.
			\end{align*}
		
		\item % 2.5, 2.6e15 2.5e15 3.4e15
			During the first reaction, $10^{10}$ neutrons are absorbed per second. The decay constant from Au-198 to Hg-198 is given by \begin{align*}
				\omega & = 1/\tau = \frac{1}{\SI{3.89}{\day} \times \SI{86400}{\s \per \day}} = \SI{2.97e-6}{\per\s}.
				\end{align*}
			
			Then, after six days, the Au-198 atoms present is \begin{align*}
				N(t) & = \frac{\SI{e10}{reactions \per \second}}{\SI{2.97e-6}{\per\s}} \left[
					1 - e^{-\SI{6}{\day} / \SI{3.89}{\day}}
				\right] \\
					& = \SI{2.63e15}{atoms}.
			\end{align*}
			
			After six days, the amount of Hg atoms is given by \begin{align*}
				N_\mathrm{Hg} & = pt - N_\mathrm{Au} \\
					& = \left(\SI{e10}{\per \sec}\right) \left(\SI{6}{\day} \times \SI{86400}{\s \per \day}\right) \\
					& = \SI{2.5e15}{atoms}.
			\end{align*}
			
			The equilibrium number is reached when $t\to\infty$, \begin{align*}
				\lim_{t \to \infty} N(t) & = \frac{\SI{e10}{reactions \per \second}}{\SI{2.97e-6}{\per\s}} \\
					& = \SI{3.36e15}{atoms}.
			\end{align*}
		
		\item % 2.8
			The transition rates are \begin{align*}
				\omega_{235} & = 1/\left(\SI{1.03e9}{\year} \times \SI{3.154e7}{\s\per\year}\right) = \SI{3.07e-17}{\per\s} \\
				\omega_{238} & = 1/\left(\SI{6.49e9}{\year} \times \SI{3.154e7}{\s\per\year}\right) = \SI{4.88e-18}{\per\s}. 
			\end{align*}
			Then for a multimodal decay, \begin{align*}
				N(t) & = N(0) e^{-(\omega_{235} + \omega_{238})t} \\
				\num{7.3e-3} & = e^{-(\omega_{235} + \omega_{238})t} \\
				t & \approx \SI{4.38e9}{\year}.
			\end{align*}
		
		\item % 2.11
			From Problem 2.3, we know the atomic fraction is \begin{align*}
				\frac{N_{14}(t)}{N_{12}} & = \num{8.1e-13}.
				\intertext{And this is equal to }
					& = \frac{ N_{14}(0) }{N_{12}} e^{-t/\tau} \\
					& = 10^{-12} e^{-t/\tau} \\
				\implies t & = \SI{1742}{\year}.
			\end{align*}
			
			
		\item %2.12
			\begin{enumerate}
				\item The total cross section is given as \begin{align*}
					\sigma & = \sigma_e + \sigma_c + \sigma_f \\
						& = \SI{2.7002e-26}{\m\squared}.
				\end{align*}
				The attenuation can be found using \begin{align*}
					1-n \sigma x & = \frac{\SI{1e-1}{\kg\per\m\squared}}
						{\SI{0.235}{\kg \per \mol}} \times \SI{6.02e23}{atoms\per\mol} \times \SI{2.7002e-26}{\m\squared} = 0.993.
				\end{align*}
			
				\item %510
					The total rate is given by the incident particle beam hitting the cross-section, \begin{align*}
						R_\mathrm{total} & = \SI{e5}{\per\s} \times \SI{0.00691}{} \\
							& = \SI{691.7}{\per\s}.
					\end{align*}
					The number of fission reactions is then \begin{align*}
						R_\mathrm{f} & = \frac{\sigma_f}{\sigma} R_\mathrm{total} \\
							& = \SI{512.3}{\per\s}.
					\end{align*}
				\item %4e-5
					Similarly, for elastically scattered particles, \begin{align*}
						R_\mathrm{e} & = \frac{\sigma_f}{\sigma} R_\mathrm{total} \\
							& = 0.051.
					\end{align*}
			\end{enumerate}
	\end{enumerate}
\end{document}