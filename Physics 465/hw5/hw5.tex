\documentclass{homework}

\title{Homework 5}
\author{Kevin Evans}
\studentid{11571810}
\date{February 16, 2021}
\setclass{Physics}{465}
\usepackage{amssymb}
\usepackage{mathtools}
\usepackage{graphicx}
\usepackage{amsthm}
\usepackage{amsmath}
\usepackage{slashed}
\usepackage{boldline}
\usepackage{physics}
\usepackage{hyperref}
\usepackage{tcolorbox}
\usepackage[inter-unit-product =\cdot]{siunitx}

\usepackage[makeroom]{cancel}
\usepackage{booktabs}
\usepackage{multirow}

\usepackage{times}
\usepackage{mhchem}
\usepackage{enumitem}
\usepackage[normalem]{ulem}
\usepackage{systeme}
\usepackage{tikz}
\usepackage{mathtools}
\usepackage{tabularx}

%\usepackage{calligra}
%\DeclareMathAlphabet{\mathcalligra}{T1}{calligra}{m}{n}
%\DeclareFontShape{T1}{calligra}{m}{n}{<->s*[2.2]callig15}{}
%\newcommand{\scriptr}{\mathcalligra{r}\,}
%\newcommand{\boldscriptr}{\pmb{\mathcalligra{r}}\,}
%\newcommand{\emf}{\mathcal{E}}

\DeclareSIUnit\eVperc{\eV\per\clight}
\DeclareSIUnit\clight{\text{\ensuremath{c}}}
\DeclareSIUnit\year{yr}
\newcommand{\M}{\ensuremath{\mathcal{M}}}
\newcommand{\fm}{\femto\meter}

\begin{document}
	\maketitle
	\begin{enumerate}
		\item \begin{enumerate}
			\item The LHS of the equation is the $Q$ of the reaction, it's the available energy that the reaction can yield. 
			
			For this reaction, we know the momentum must be conserved, where $$\bvec{p}_\gamma = \bvec{p}_\mathrm{recoil}.$$
			If we assume non-relativistic motion and use nuclear masses, \begin{align*}
				\bvec{p}_\gamma = pc \implies p_\gamma = E_\gamma / c \\
				E_{D} = p^2 / 2m = \frac{ E_\gamma^2 }{2 \mathcal{M}_D c^2}.
			\end{align*}
			We can equate the RHS of the equation as \begin{align*}
				E_\gamma + E_\mathrm{recoil} = E_\gamma + \frac{ E_\gamma^2 }{2 \mathcal{M}_D c^2}.
			\end{align*}
			
			\item Atomic masses should be OK to use on the left side for $m_p$ and $m_d$, as we're dealing with atoms complete with bound electrons.
			
			Using mass values from the internet, \begin{align*}
					\SI{939.565}{\MeV}
					+ \SI{938.272}{\MeV}
					- \SI{1876}{\MeV} & = E_\gamma + \frac{ E_\gamma^2 }{2 \times \SI{1.875}{\GeV}} \\
					E_\gamma & = \SI{1.84}{\MeV}.
			\end{align*}

			The quadratic term doesn't matter much, as that term is roughly only \SI{900}{\eV}---far less than the \si{\MeV} range.
			
			\item In Williams, $Q_\alpha = \SI{7.834}{\MeV}$. 
				We could expect \begin{align*}
					Q_\alpha & = \left[\SI{213.995186}{\atomicmassunit} - \SI{209.984163}{\atomicmassunit} - \SI{4.001506}{\atomicmassunit}\right]c^2 \\
						& = \SI{8.87}{\MeV},
				\end{align*}
				which is pretty close to the \SI{7.8}{\MeV}, I guess...
				
				I'm not sure how/what to check if it's consistent with the caption of the cloud chamber figure. We see a long-range (high-energy) decay occurring in the cloud chamber. Since the energy of the decay is higher than the other decay mode, shouldn't it be less common? 
		\end{enumerate}
	
		\item Using Table 5.1 and \href{https://physics.nist.gov/cgi-bin/Compositions/stand_alone.pl}{NIST's Atomic Mass Table}, the $Q$ values for each scenario are \begin{enumerate}
			\item $\beta^-$ emission. \begin{align*}
				Q_{\beta^-}& = \left(\M(19, 40) - \M(20, 40)\right)c^2 \\
					& = \left(\SI{39.963998}{\atomicmassunit} - \SI{39.962590}{\atomicmassunit} \right)c^2\\
					& = \SI{1.31}{\MeV}.
			\end{align*}
		
			\item $\beta^+$ emission. \begin{align*}
				Q_{\beta^+} & = \left(
					\M(19, 40) - \M(18, 40) - 2 m_e
				\right)c^2 \\
					& = \left(
						\SI{39.963998}{\atomicmassunit}
						- \SI{39.962383}{\atomicmassunit}
						- 2 \times \SI{5.485799e-4}{\atomicmassunit}
					\right)c^2 \\
					& = \SI{482}{\keV}.
			\end{align*}	
		
			\item $e^-$ capture.
				\begin{align*}
					Q_\mathrm{EC} & =  \left(
					\SI{39.963998}{\atomicmassunit}
					- \SI{39.962383}{\atomicmassunit}
					\right)c^2 \\
					& = \SI{1.52}{\MeV}.
				\end{align*}
		\end{enumerate}
	
		\item In this sequence from \ce{^238 U} to \ce{^206 Pb}, there are 8 $\alpha$-decays and 6 $\beta$-decays (can we use atomic masses even if there are an uneven number of beta and alpha decays?). 
		
			The $Q$ value can be calculated as \begin{align*}
				Q & = \left(
						\M(\text{U}, 938)		
						- \M(\text{Pb}, 206)
						- 8\M(\text{He}, 4)
					\right)c^2 \\
					& = \left(
						\SI{238.050788}{\atomicmassunit}
						- \SI{205.97446}{\atomicmassunit}
						- 8 \times \SI{4.002603}{\atomicmassunit}
					\right) c^2 \\
					& = \SI{51.7}{\MeV} = \SI{8.3e-12}{\J}.
			\end{align*}			
			From room temperature, the total change in temperature required is roughly \SI{1135}{\K}. For one gram, this requires an energy of \begin{align*}
				E & = \SI{0.12}{\J \per \g \per \K} \times \SI{1135}{\K} = \SI{136.2}{\J}.
			\end{align*}
			In terms of decays, this needs $\num{1.644e13}$ decays, and in a gram of \ce{^238 U}, there are \num{2.53e21} atoms. I'm guessing the time required would be solved with \begin{align*}
				\num{1.644e13} & = \num{2.53e21} \exp(-t / \SI{4.5}{\giga\year}) \\
				t & = \SI{85}{\giga\year}?
			\end{align*}
		\item If we consider a particle of energy $E$ tunneling through a potential barrier of energy $U$, its wavefunction is given by \begin{align*}
			\Psi(x) & \propto e^{-kx} \\
			\text{where } k & = \frac{1}{\hbar} \sqrt{2m(U-E)}.
		\end{align*}
		Then, we can assume the rate of tunneling across the spatial interval $[a, b]$ is given by its probability, i.e. \begin{align*}
			\text{tunneling rate } \omega & \propto \abs{\Psi}^2 \\
				& \propto \exp{-\frac{2}{\hbar} \sqrt{2m} \int_a^b \sqrt{U(x) - E} \dd{x}}.
			\intertext{Then, changing the bounds of the integral and taking the log (not sure if I really understand the reasoning of this part...),}
			\ln \omega & = -2 \sqrt{\frac{2m}{\hbar}} \int_0^{r=c/E} \sqrt{ \frac{c}{r} - E } \dd{r} \\
			\intertext{Changing variables to $x=Er/c$,}
				& = \dots \frac{c}{\sqrt{E}} \int_0^1 \sqrt{1/x - 1} \dd{x}.
			\intertext{Using an integration table, we can see that}
			\ln \omega & \propto - 1/\sqrt{E},
			\intertext{as depicted in Figure 6.3.}
		\end{align*}
	\end{enumerate}
\end{document}