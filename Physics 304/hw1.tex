\documentclass{homework}

\title{Homework 1}
\author{Kevin Evans}
\studentid{11571810}
\date{January 17, 2020}
\setclass{Physics}{304}
\usepackage{amssymb}
\usepackage{mathtools}

\usepackage{amsthm}
\usepackage{amsmath}
\usepackage{slashed}
\usepackage{relsize}
\usepackage{threeparttable}
\usepackage{float}
\usepackage{booktabs}
\usepackage{boldline}
\usepackage{changepage}
\usepackage{venndiagram}
\usepackage{icomma}
\usepackage{physics}
\usepackage[inter-unit-product =\cdot]{siunitx}
\usepackage{setspace}

\usepackage[makeroom]{cancel}
\usepackage{pgfplots}

\usepackage{multicol}
\usepackage{tcolorbox}
\usepackage{enumitem}
\usepackage{times}

\begin{document}
	\maketitle
	\subsection*{Questions}
	\begin{enumerate}
		\item[1.] Because a magnetic moment is defined as
			\[ \boldsymbol{\mu} = \frac{q}{2m} \bvec{L} \]
			For an electron, the charge $q=-e$. The direction of $\boldsymbol{\mu}$ will always be opposite of $\bvec{L}$.
			
		\item[2.] The Stern-Gerlach experiment uses an inhomogeneous magnetic field to create a non-zero net force on particles. If the magnetic field were uniform, there would be zero \textit{net} force on the electron's orbit---but instead it would only exert a non-zero torque and the particle would not experience any deflection in its trajectory. This is similar to loops of current in magnetic fields.
		
		\item[3.] No. If the particle had a non-zero net charge, the particle would experience a Lorentz force, $q\bvec{v} \cross \bvec{B}$.
		% -> talk about spin affecting it
	\end{enumerate}

	\subsection*{Problems}
	\begin{enumerate}
		\item[1.] The total magnetic moment is given as \begin{align*}
			\boldsymbol{\mu} & = \boldsymbol{\mu}_0 + \boldsymbol{\mu}_s = \frac{-e}{2m_e} \left\{
				\bvec{L} + g \bvec{S}
			\right\}
			\intertext{Since we're only concerned with the change in energy from spin, we can omit the orbital momentum as it remains constant in both states. Then, we can take the component ($\mu_z$) in the direction of $\bvec{B}$. }
			\boldsymbol{\mu} & = \frac{-e}{2m_e} g \bvec{S} \\
			\mu_z & = \frac{e}{2m_e} g S_z = \frac{\hbar ge}{2m_e} m_s
			\intertext{The change in magnetic quantum numbers is $\Delta m_s = 1$, with the change in energy between the aligned and unaligned states as}
			\Delta E & = \frac{\hbar g e B}{2 m_e} \approx \frac{ \hbar e B }{m_e} \\
			\intertext{Equating this to the energy of a photon,}
			\Delta E & \approx \frac{ \hbar e B }{m_e} = \hbar \omega = 2 \pi \hbar f \\
			f & = \frac{e B}{2 \pi m_e} = \frac{ \left( \SI{1.60e-19}{\coulomb} \right) \left( \SI{0.35}{\tesla} \right) }{2 \pi \cdot \SI{9.11e-31}{\kg}} \\
			& \approx \SI{9.8}{\giga\hertz} \qed
		\end{align*}
	\end{enumerate}
\end{document}