\documentclass{homework}

\title{Homework 2}
\author{Kevin Evans}
\studentid{11571810}
\date{January 31, 2020}
\setclass{Physics}{304}
\usepackage{amssymb}
\usepackage{mathtools}

\usepackage{amsthm}
\usepackage{amsmath}
\usepackage{slashed}
\usepackage{relsize}
\usepackage{threeparttable}
\usepackage{float}
\usepackage{booktabs}
\usepackage{boldline}
\usepackage{changepage}
\usepackage{venndiagram}
\usepackage{icomma}
\usepackage{physics}
\usepackage[inter-unit-product =\cdot]{siunitx}
\usepackage{setspace}

\usepackage[makeroom]{cancel}
\usepackage{pgfplots}

\usepackage{multicol}
\usepackage{tcolorbox}
\usepackage{enumitem}
\usepackage{times}

\begin{document}
	\maketitle
	\section*{Problems}
	\begin{enumerate}
		\item[0.] For three spin-1/2, the total angular momenta are $j_s = s_1 + s_2 + s_3 = 3/2$ and $1/2$.
		
		\begin{tabular}{cccc}
			\toprule
			$j_s$ & $s_1$ & $s_2$ & $s_3$  \\
			\midrule
			$1/2$ & - & - & +  \\
			$1/2$ & - & + & - \\
			$1/2$ & + & - & - \\
			$1/2$ & + & + & - \\
			$3/2$ & + & - & + \\
			$3/2$ & - & + & + \\
			$3/2$ & + & + & + \\
			$3/2$ & - & - & - \\
			\bottomrule
		\end{tabular}
		
		\item[10.] For $n=2$, $\ell \in \{0, 1\}$. Then, $j \in \{ 1/2, 3/2 \}$.
			\begin{align*}
				j = \frac{1}{2} & \implies m_j \in \left\{ -\frac{1}{2}, \frac{1}{2} \right\} \\
				j = \frac{3}{2} & \implies m_j \in \left\{ -\frac{3}{2}, -\frac{1}{2}, \frac{1}{2}, \frac{3}{2} \right\}
			\end{align*}
			
		\item[11.] For a $d$-electron, $l=2$ and $j \in \left\{ \frac{1}{2}, \frac{3}{2}, \frac{5}{2} \right\}$. \begin{align*}
			j = \frac{1}{2} & \implies m_j \in \left\{ \pm \frac{1}{2} \right\} \\
			j = \frac{3}{2} & \implies m_j \in \left\{ \pm \frac{1}{2}, \pm \frac{3}{2} \right\} \\
			j = \frac{5}{2} & \implies m_j \in \left\{ \pm \frac{1}{2}, \pm \frac{3}{2}, \pm \frac{5}{2} \right\}
		\end{align*}
		\item[12.] \begin{enumerate}
			\item $7\mathrm{G}_{9/2}$
			\item $6\mathrm{H}_{11/2}$, $6\mathrm{H}_{9/2}$, $6\mathrm{H}_{7/2}$, $6\mathrm{H}_{5/2}$, $6\mathrm{H}_{3/2}$, $6\mathrm{H}_{1/2}$
		\end{enumerate}
	\pagebreak
		\item[17.] With no potential inside the cube of side length $L$, \begin{align*}
			-\frac{\hbar^2}{2m} \laplacian{\psi}(\bvec{r}) & = E \psi(\bvec{r})
			\intertext{The solutions for $\psi$ would have the form}
			\psi(e_i) & = \sin(k_i \pi e_i) \implies k_i = \frac{n_i \pi}{L}
			\intertext{Then the momentum in each direction and total energy is:}
			p_i & = \hbar k_i = \frac{\pi \hbar}{L} n_i \\
			E_\mathrm{total} & = \frac{\pi^2 \hbar^2}{2m L^2} \left( n_x^2 + n_y^2 + n_z^2  \right) = \frac{\pi^2 \hbar^2 n^2}{2mL^2}
			\intertext{Applying the dimensions given in the problem,}
			E_\mathrm{total} & \approx \frac{ \pi^2 \left(\SI{0.1973}{\eV\um}\right)^2}{2\left(\SI{0.511e6}{\eV \per c^2}\right) \left(\SI{0.2e-3}{\um}\right)^2} \; n^2 \\
			& \approx 9.398 n^2 \quad [\si{\eV}]
		\end{align*}
		
		\begin{enumerate} %395
			\item For electrons, each state can contain two electrons. In the $n_i=1$ ($n^2=3$) state, 6 electrons would be contained, each with energy 
			\[ E_{111} = 3\cross9.398\]
			The remaining two electrons would be in the threefold-degenerate state with $n^2=6$,
			\[ E_{112} = E_{121} = E_{211} = 6 \cross 9.398  \]
			The lowest energy possible is then \begin{align*}
				E_\mathrm{total} & = 6 \cross 3 \cross 9.398 + 2 \cross 6 \cross 9.398 \\
					& = \SI{281.7}{\eV}?
			\end{align*}
			Not sure what part I'm not understanding as the answer given is \SI{395}{\eV}.
			
			\item For bosons, each particle would fall to the ground state, \begin{align*}
				E_\mathrm{total} & = 8 E_{111} \\
					& = \SI{225.6}{\eV}
			\end{align*}
		\end{enumerate}
		\item[21.] \begin{enumerate}
			\item $1s^2 2s^2 2p^4$
			\item \begin{tabular}[t]{cccc}
				\toprule
				$n$ & $\ell$ & $m_\ell$ & $m_s$ \\
				\midrule
				$1$ & $0$ & $0$ & $-1/2$ \\
				$1$ & $0$ & $0$ & $1/2$ \\
				\midrule
				$2$ & $0$ & $0$ & $-1/2$ \\
				$2$ & $0$ & $0$ & $1/2$ \\
				$2$ & $1$ & $0$ & $-1/2$ \\
				$2$ & $1$ & $0$ & $1/2$ \\
				$2$ & $1$ & $\pm1$ & $-1/2$ \\
				$2$ & $1$ & $\pm1$ & $1/2$ \\
				\bottomrule
			\end{tabular}
		\end{enumerate}
	\end{enumerate}

	\section*{Questions}
	\begin{enumerate}
		\item[1.] After $Z=83$, the repulsive Coulomb force between the protons is greater than the nuclear force holding the protons and neutrons together.
		\item[2.] The frequency also doubles as $\omega \propto \abs{\bvec{B}}$.
		\item[3.] Because there's an uneven number of protons and neutrons.
		\item[4.] Y would be more unstable. If X has a higher binding energy, that would mean it there's more energy holding it together.
	\end{enumerate}
\end{document}