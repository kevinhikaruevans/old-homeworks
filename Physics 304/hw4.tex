\documentclass{homework}

\title{Homework 4}
\author{Kevin Evans}
\studentid{11571810}
\date{February 14, 2020}
\setclass{Physics}{304}
\usepackage{amssymb}
\usepackage{mathtools}

\usepackage{amsthm}
\usepackage{amsmath}
\usepackage{slashed}
\usepackage{relsize}
\usepackage{threeparttable}
\usepackage{float}
\usepackage{booktabs}
\usepackage{boldline}
\usepackage{changepage}
\usepackage{venndiagram}
\usepackage{icomma}
\usepackage{physics}
\usepackage[inter-unit-product =\cdot]{siunitx}
\usepackage{setspace}

\usepackage[makeroom]{cancel}
\usepackage{pgfplots}

\usepackage{multicol}
\usepackage{tcolorbox}
\usepackage{enumitem}
\usepackage{times}
\usepackage{mhchem}

\DeclareSIUnit{\curie}{Ci}
\DeclareSIUnit{\year}{yr}
\begin{document}
	\maketitle
	% 26,28,35,41,42,44,46,47,49,50
	\subsection*{Chapter 13}	
	\begin{enumerate}
		\item[26.] If the half-life is \SI{8.1}{\day}, then we can determine the decay constant as
			\[ \lambda = \frac{\ln 2}{T_{1/2}} = \frac{\ln 2}{\SI{8.1}{\day} \cross \SI{86400}{\s\per\day}} =  \SI{9.904e-7}{\per\s} \]
			Then using the current activity, the number of remaining atoms can be found as,
			\begin{align*}
				R & = N \lambda \\
				\SI{0.2e-6}{\curie} \cross \left(\frac{\SI{3.7e10}{\becquerel}}{\SI{1}{\curie}}\right) & = N \left(\SI{9.904e-7}{\per\s}\right) \\
				N & = \num{7.5e9} \text{ atoms remaining}
			\end{align*}
		\item[28.] From eq. (13.10), \begin{align*}
			R & = \abs{ \dv{N}{t} } = N_0 \lambda e^{-\lambda t} = R_0 e^{-\lambda t} \tag{13.10}
			\shortintertext{We can derive the first equation,}
			R & = R_0 e^{-\lambda t} \\
			\ln(\frac{R}{R_0}) & = -\lambda t \\
			-\left( \ln R - \ln R_0 \right) & = \lambda t \\
			\ln(\frac{R_0}{R}) & = \lambda t \\
			\lambda & = \frac{1}{t} \ln(\frac{R_0}{R}) \qed
			\shortintertext{Using the solution above and (13.11),}
			\lambda & = \frac{\ln 2}{T_{1/2}} \\
			\frac{\ln 2}{T_{1/2}} & = \frac{1}{t}\ln(R_0 / R) \\
			\frac{T_{1/2}}{\ln 2} & = \frac{t}{\ln(R_0 / R)} \\
			T_{1/2} & = \frac{(\ln 2) t}{\ln(R_0 / R)} \qed
		\end{align*}
		\pagebreak
		\item[35.] For a half-life $T_{1/2} = \SI{5730}{\year}$, the rate constant would be
			\begin{align*}
				\lambda & = \frac{\ln 2}{5730} \approx \SI{1.21e-4}{\per \year} \\
				& = \frac{\ln 2}{\SI{5730}{\year} \times \SI{365.25}{\day/\year} \times \SI{1440}{\min\per\day}} \approx \SI{2.3e-10}{\per \min}
				\intertext{Then for a gram of Carbon,}
				R & = N_0 \lambda e^{-\lambda t} \tag{13.10} \\
				R & \approx \left(\SI{1}{\g} \text{ Carbon}
					\times \frac{\SI{1}{\mol}}{\SI{12.011}{g}}
					\times \frac{\num{6.022e-23} \text{ atoms} }{\SI{1}{\mol}}
					\times \frac{\num{1.3e-12} \text{ atoms } \ce{^{14} C}}{1 \text{ atom } \ce{^{12} C}}
				\right) \\
				& \qquad \times \left(\SI{2.3e-10}{\per \min}\right) \\
				& \qquad \times \exp(-\SI{1.21e-4}{\per\year} \times \SI{2000}{\year}) \\
				& \approx \num{11.769} ~~ \si{disintegrations\per\minute \cdot \g}
			\end{align*}
		\item[41.] From Table 13.6, the energy released during the decay is the change in mass between the parent and daughter nuclei, \begin{align*}
			\Delta m & = 238.050785-234.043593-4.002603 = \SI{0.004589}{\amu}\\
			Q & = \Delta m c^2 = \SI{0.004589}{\amu} \times \SI{931.494}{\MeV/\amu} \\
				& \approx \SI{4.27}{\MeV}
		\end{align*}
		\item[42.] \begin{enumerate}
			\item For a photon with energy $\Delta E$, we can use the energy-momentum relation and find the photon's momentum, \begin{align*}
			p & = \frac{\Delta E}{c}
			\intertext{Then, for any non-relativistic particle of mass $M$, its kinetic energy can be written as}
			E & = \frac{ p^2 }{2M}
			\intertext{As the momentum is conserved, the particle must have an equal momentum (in magnitude),}
			E_r & = \frac{(\Delta E)^2}{2Mc^2}
			\end{align*}
			
			\item For a \ce{^{57} Fe} nucleus and \SI{14.4}{\keV} $\gamma$-emission,
				\begin{align*}
					E_r & = \frac{\left(\SI{14.4e-3}{\MeV} \right)^2}{2\left(\SI{57}{\amu} \times \SI{931.494}{\MeV/\amu} \right)} \\
						& = \SI{28.11}{\micro\electronvolt}
				\end{align*}
		\end{enumerate} 
		\item[44.] For $\ce{^{220}_{86} Rn} \to \ce{^{216}_{84} Po} + \ce{^4_2\alpha}$, if we assume all of the disintegration energy goes into the $\alpha$ particle's kinetic energy, \begin{align*}
			\Delta m & = \left(220.011368-216.001888-4.002603 \right) = \SI{0.006877}{\amu} \\
			Q & = \Delta mc^2 = \SI{0.006877}{\amu} \times \SI{931.494}{\MeV/\amu} = \SI{6.4059}{\MeV}
		\end{align*}
		\pagebreak
		\item[46.] \begin{enumerate}
			\item It's forbidden because the mass/energy of the free proton is less than the mass of a neutron.
			\item It's possible since the mass/energy of the proton bound within a nucleus is greater than its resultants.
			\item For the reaction in (b) and assuming the $\nu$-energy is negligible,
				\begin{align*}
					Q & = \left(13.005739 - 13.003355 - 2\times 0.000549\right) \times 931.494 \\
						& = \SI{1.1979}{\MeV}
				\end{align*}
		\end{enumerate}
		\item[47.] For the nucleus of \ce{^{13} N} (from the last problem), its radius is given as \begin{align*}
			r & \approx \left( \SI{1.2}{\femto\meter} \right) 13^{1/3} \\
				& \approx \SI{2.82}{\femto\meter}
			\intertext{From the uncertainty principle, for an electron exists within the nucleus, the minimum uncertainty in its momentum can be determined as}
			\Delta p & \approx h / \Delta x \\
				& \approx \left( \SI{6.626e-34}{\joule \cdot \s} \right) / \left( \SI{2.82}{\femto\meter} \right) \\
				& \approx \SI{2.35e-19}{\J\s\per\meter}
			\intertext{Using the relativistic energy-momentum relation,}
			E^2 & \approx (pc)^2 + (m_e c^2)^2 \\
			& = \left(\SI{2.35e-19}{\J\s\per\meter} \times c\right)^2
				+ \left(\SI{9.11e-31}{\kg} \times c^2\right)^2 \\
			E & \approx \SI{7.1e-11}{\J} \\
			 & \gtrsim \SI{400}{\MeV}
		\end{align*}
		The energy of an electron within the nucleus would far exceed the energy of electrons emitted during beta decay.
		
		I definitely looked at the answers in the back of the book on that last part. I'm not sure I would've gotten the connection to that energy and the usual energy of electrons during $\beta$ decay.
		\pagebreak
		\item[49.] I'm going to omit the neutrino mass and disregard the $c^2$ term (and just add it on at the end).
		
		Since we're looking at the $Q$-values for the nuclei energies of mass $M_N$, we can relate that to the mass of the atom of mass $M$ as, \begin{align*}
			M( \ce{^A_Z X} ) & = M_N(\ce{^A_Z X} ) + Z m_e
			\intertext{Or, in terms of a nucleon $M_N$,}
			M_N(\ce{^A_Z X} ) & = M( \ce{^A_Z X} ) - Z m_e
			\intertext{The $Q$ value for $\beta$-decay is}
			Q & = M_N( \ce{^A_Z X} ) - M_N( \ce{^A_{Z + 1} X}) + m_e + m_\nu 
			\intertext{Substituting the nuclei masses for the atomic masses and neglecting the $\nu$ mass,}
			Q & = \left[ M( \ce{^A_Z X} ) - Z m_e \right]
				- \left[ M( \ce{^A_{Z+1} Y} ) - (Z+1) m_e \right]
				- m_e \\
				& = \left[ M( \ce{^A_Z X} ) -  M( \ce{^A_{Z+1} Y} ) \right] c^2 \qed
			\intertext{Using a similar approach for positron emission,}
			Q & = M_N( \ce{^A_Z X} ) - M_N( \ce{^A_{Z - 1} Y}) - m_e \\
				& = \left[ M( \ce{^A_Z X} ) - Z m_e \right]
				- \left[ M( \ce{^A_{Z-1} Y} ) - (Z-1) m_e\right]
				- m_e \\
				& = \left[ M( \ce{^A_Z X} ) - Z m_e \right]
				- \left[ M( \ce{^A_{Z-1} Y} ) - Z m_e + m_e\right] 
				- m_e \\
				& =  \left[ M( \ce{^A_Z X} ) - M( \ce{^A_{Z-1} Y} )  - 2m_e \right] c^2 \qed
				\intertext{For electron capture, the $Q$ value would be}
			Q & = M_N( \ce{^A_Z X} ) + m_e - M_N( \ce{^A_{Z - 1} Y}) \\
				& = \left[ M( \ce{^A_Z X} ) - Z m_e \right]
				+ m_e
				- \left[ M( \ce{^A_{Z-1} Y} ) - (Z-1) m_e\right] \\
				& =  \left[ M( \ce{^A_Z X} ) -  M( \ce{^A_{Z-1} Y} ) \right] c^2 \qed
		\end{align*}
		
		\item[50.]
	\end{enumerate}
\end{document}