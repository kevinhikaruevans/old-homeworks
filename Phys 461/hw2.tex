\documentclass{homework}

\title{Homework 2}
\author{Kevin Evans}
\studentid{11571810}
\date{January 31, 2021}
\setclass{Physics}{461}
\usepackage{amssymb}
\usepackage{mathtools}

\usepackage{amsthm}
\usepackage{amsmath}
\usepackage{slashed}
\usepackage{relsize}
\usepackage{threeparttable}
\usepackage{float}
\usepackage{booktabs}
\usepackage{boldline}
\usepackage{changepage}
\usepackage{physics}
\usepackage[inter-unit-product =\cdot]{siunitx}
\usepackage{setspace}

\usepackage[makeroom]{cancel}
%\usepackage{pgfplots}

\usepackage{enumitem}
\usepackage{times}

\usepackage{amsthm}
\newtheorem*{ident}{Identity}

\renewcommand\qedsymbol{$\blacksquare$}

\begin{document}
	\maketitle
	\begin{enumerate}
		\item For the relative velocity $\bvec{v}_\mathrm{rel} = \bvec{v}_2 - \bvec{v}_1$, if we take the first derivative and substitute the opposing forces, \begin{align*}
			\bvec{\dot{v}}_\mathrm{rel} & = \bvec{\dot{v}}_2 - \bvec{\dot{v}}_1 \\
				& = -\frac{ \bvec{F} }{m_2} - \frac{ \bvec{F} }{m_1} = -\bvec{F} \left(m_2^{-1} + m_1^{-1}\right) \\
				& = -\frac{ \bvec{F} }{\mu}
			\intertext{where}
			\mu & = \left(m_2^{-1} + m_1^{-1}\right)^{-1} \\
				& = \frac{m_1 m_2}{m_1 + m_2}
		\end{align*}
	
		For the hydrogen atom, it's a pretty good approximation as the mass of the electron is miniscule relative to the proton, \begin{align*}
			\mu & = \frac{ m_p m_e }{m_p + m_e} \\
				& \approx \frac{m_p m_e}{m_p} = m_e
		\end{align*}
		This approximation would not be valid for positronium, as the masses are equal.
		
		\item Starting with the rightmost and evaluating by chain rule (and mixing notations), we can get to the rightmost equation, \begin{align*}
			\frac{1}{r} \dv[2]{r} r R(r) & = \frac{1}{r} \left[
				r R''(r) + 2R'(r) + R(r) \dv[2]{r} r
			\right] \\
			& = R''(r) + \frac{2}{r}R'(r) + 0 \\
			& = \left[ \dv[2]{r} + \frac{2}{r} \dv{r} \right] R(r) 
		\end{align*}
		Then, starting with the middle equation and evaluating by chain rule, \begin{align*}
			\frac{1}{r^2} \dv{r} r^2 \dv{r} R(r) & = \frac{1}{r^2} \dv{r} r^2 R'(r) \\
				& = \frac{1}{r^2} \left[ 2rR'(r) + r^2 R''(r) \right] \\
				& = \left[\dv[2]{r} + \frac{2}{r} \dv{r}\right] R(r)
		\end{align*}
	
		\item The radial Schrodinger equation with zero potential is given by \begin{align*}
			-\frac{\hbar^2}{2m} \left[ \frac{1}{r} \dv[2]{r} r \phi(r) - \frac{\ell(\ell + 1)}{r^2}  \phi(r)\right] & = E \phi(r)
			\intertext{Using the wavefunction $\phi(r) = Ae^{ikr} / r$ in the Schrodinger equation,}
			-\frac{\hbar^2}{2m} \left[ \frac{-k^2}{r} \left(\frac{r}{r}\right) e^{ikr}  - \frac{\ell(\ell + 1)}{r^2} \phi(r)\right] & = E \phi(r) \\
			-\frac{\hbar^2}{2m} \left[ -k^2 \phi(r) - \frac{\ell(\ell + 1)}{r^2} \phi(r)\right] & = E \phi(r)  \\
			\frac{\hbar^2}{2m} \left[k^2 + \frac{\ell(\ell + 1)}{r^2}\right] & = E
		\end{align*}
		As discussed in class, the angular momentum term has an $r$ dependence that is not on the RHS. Thus $\ell(\ell + 1)$ must be zero, and $\ell = 0$. The energy is then $$E = \frac{\hbar^2 k^2}{2m}$$
		
		\item For an electron trapped in an infinite spherical well, the Schrodinger equation looks something like \begin{align*}
			- \frac{\hbar^2}{2 m} \left[ \frac{1}{r} \dv[2]{r} r \phi(r) - \frac{\ell(\ell+1)}{r^2} \phi(r)\right] & = E \phi(r)
			\intertext{For the $\ell = 0$ ground state and multiplying by $r$,}
			- \frac{\hbar^2}{2 m} \left[ \frac{1}{r} \dv[2]{r} r \phi(r)\right] & = E \phi(r) \\
			-\frac{\hbar^2}{2m} \dv[2]{r} r \phi(r) & = E r \phi(r)
		\end{align*}
		As the second derivative of the wavefunction must be negative, the ansatz\footnote{I'm not sure if I'm using \textit{ansatz} correctly here, but the textbook uses that word copiously.} will have the form \begin{align*}
			\phi(r) & = A \frac{ \sin(kr) }{r} + B \frac{\cos(kr) }{r}
			\intertext{For the wavefunction to be well-defined at the origin, $B \to 0$, and we're left with a $\mathrm{sinc}$-like function,}
			\phi(r) & = A \frac{ \sin(kr) }{r} \\
			\phi''(r) & = -Ak^2 \phi(r)
			\intertext{Plugging this back into the Schrodinger equation to find $k$,}
			\frac{\hbar^2 k^2}{2m} & = E \\
			k & = \frac{\sqrt{2mE}}{\hbar}
			%\intertext{At the boundary $r=a$, the $\abs{\phi}^2 = 0$ and}
			%\phi(r=a) & = A \frac{\sin(ka)}{a} = 0 \\
			%& 
		\end{align*}
	
		\item For $R(r) = Ae^{-\beta r}$, \begin{align*}
			-\frac{\hbar^2}{2m} \left[\frac{1}{r} \left(\beta^2 r - 2\beta\right)R(r) - \frac{\ell(\ell + 1)}{r^2} R(r)\right] - \frac{e^2}{4 \pi \epsilon_0 r} R(r)  & = Er \frac{ R(r) }{r} \\
			-\frac{\hbar^2}{2m} \left[\beta^2 r - 2 \beta - \frac{ \ell(\ell + 1) }{r}\right] - \frac{e^2}{4 \pi \epsilon_0} & = Er
		\end{align*}
		I'm not entirely sure what I'm supposed to be doing here.
		
		\item For the $n=1$ hydrogen wavefunction \begin{align*}
			\phi(r) & = C e^{-r/a_0}
			\intertext{It's normalized as}
			1 & = \int \abs{ \phi(r) }^2 \dd[3]{\bvec{r}} \\
				& = 4 \pi C^2 \int_{0}^\infty e^{-2r/a_0} r^2 \dd{r} \\
				& = 4 \pi C^2 \left(a_0^3 / 4\right) \qquad \text{(WolframAlpha'd the integral)} \\
			C & = \sqrt{\frac{a_0^3}{\pi }}
		\end{align*}
	
		\item \begin{enumerate}
			\item For $l=1$, the associated $Y_{lm}$ is \begin{align*}
				Y_{lm} & = \begin{cases}
					\mp \frac{1}{2} \sqrt{\frac{3}{2\pi}} \sin \theta e^{\pm i \phi} & m = \pm 1 \\
					\frac{1}{2} \sqrt{ \frac{3}{\pi} } \cos \theta & m = 0
				\end{cases}
			\end{align*}
		
			\item 
		\end{enumerate}
		\item Letting $\rho = Zr$ in the Schrodinger equation results in \begin{align*}
			-\frac{\hbar^2}{2m} \left[ \frac{1}{\rho} \dv[2]{\rho} \rho R(\rho) - \frac{\ell(\ell + 1)}{\rho^2} R(\rho)\right] - \frac{Ze^2}{\rho} & = E R(\rho)
		\end{align*}

		\item In the two dimensional case and assuming a separable wavefunction, $\Psi(r, \phi) = R(r) \Phi(\phi)$, \begin{align*}
			-\frac{\hbar^2}{2M} \left[ \pdv[2]{\Psi}{r} + \frac{1}{r} \pdv{\Psi}{r} + \frac{1}{r^2} \pdv[2]{\Psi}{\phi} \right] + U(r) \Psi & = E \Psi \\
			- \frac{\hbar^2}{2M} \left[ \Phi(\phi) R''(r) + \frac{1}{r} \Phi(\phi) R'(r) + \frac{1}{r^2} R(r) \Phi''(\phi)\right] + U(r) \Psi & = E \Psi \\
			\intertext{Dividing by $\Psi$ then multiplying by $r^2$ and rearranging,}
			-\frac{\hbar^2}{2M} \left[\frac{1}{R(r)} \left(R''(r) + \frac{R'}{r}\right)  + \frac{\Phi''(\phi)}{r^2 \Phi(\phi)} \right] + U(r) & = E \\
			- \frac{\hbar^2}{2M} \frac{r}{R(r)} \left(rR''(r) + R'(r)\right) + r^2 \left(U(r) - E\right) & = \frac{\hbar^2}{2M} \frac{\Phi''(\phi)}{\Phi(\phi)} = C
		\end{align*}
As the LHS depends only on $r$ and the RHS is a function on $\phi$, these both must be equal to a constant. We can assign this constant as
$$				C  = -\frac{\hbar^2 m^2}{2M}$$
		\item Starting from the last problem, \begin{align*}
			\frac{ \hbar^2 }{2M} \frac{\Phi(\phi)}{\Phi(\phi)} & = - \frac{\hbar^2 m^2}{2M} \\
			\Phi''(\phi) & = -m^2\Phi(\phi) \\
			\Phi(\phi) & = Ae^{im\phi}
		\end{align*}
	Since $\Phi$ will have an argument $m \phi$, $m$ can be any integer (probably? or would it need to be a multiple of $2\pi$?). Compared to the spherical case, it's similar in that they're both sinusoidal functions.
	\end{enumerate}
\end{document}