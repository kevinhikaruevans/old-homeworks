\documentclass{homework}

\title{Homework 10}
\author{Kevin Evans}
\studentid{11571810}
\date{March 31, 2021}
\setclass{Physics}{461}
\usepackage{amssymb}
%\usepackage{mathtools}
\usepackage{graphicx}
\usepackage{amsthm}
\usepackage{amsmath}
\usepackage{slashed}
\usepackage{boldline}
\usepackage{physics}
\usepackage[inter-unit-product =\cdot]{siunitx}

\usepackage[makeroom]{cancel}
\usepackage{booktabs}

\usepackage{times}
\usepackage{mhchem}

%\usepackage{calligra}
%\DeclareMathAlphabet{\mathcalligra}{T1}{calligra}{m}{n}
%\DeclareFontShape{T1}{calligra}{m}{n}{<->s*[2.2]callig15}{}
%\newcommand{\scriptr}{\mathcalligra{r}\,}
%\newcommand{\boldscriptr}{\pmb{\mathcalligra{r}}\,}
%\newcommand{\emf}{\mathcal{E}}

\begin{document}
	\maketitle
	\begin{enumerate}
		\item \begin{enumerate}
			\item Not allowed, as $\Delta \ell = 2$.
			\item Not allowed, as $\Delta j = 2$.
			\item Allowed, as $\Delta \ell = 1$.
			\item Allowed.
			\item Not allowed, as $\Delta j = 2$.
		\end{enumerate}
	
		\item \begin{enumerate}
			\item For a dampened classical oscillator, \begin{align*}
				0 & \ddot{x} + \gamma \dot{x} + \omega_0^2 x
				\intertext{The solution can be approximated as}
				x(t) & \approx x_0 e^{-(\gamma/2) t} \cos \omega_0 t
				\intertext{Taking the Fourier transform, we can find the amplitude in the frequency domain as}
				A(\omega) & = \frac{1}{\sqrt{2\pi}} \int_\mathbb{R}
						x_0 e{-(\gamma / 2)t} \cos \omega_0 t e^{-j\omega t} \dd{t} \\
						& = \frac{x_0}{\sqrt{8 \pi}} \left[
							\frac{1}{j(\omega_0 - \omega) + \gamma / 2}
							+ \frac{1}{j(\omega_0 + \omega) + \gamma / 2}
						\right]
			\end{align*}
		
			\item 
		\end{enumerate} 
		
		\vspace{1em}
		\hrule
		
		\item[1.] \begin{enumerate}
			\item For the \ce{3^2 P_{3/2}} state, the energy is \SI{3.4e-19}{\J}. The total emitted energy is found by integrating power with respect to time, \begin{align*}
				P & = P_0 \int_0^\infty e^{-t/\tau} \dd{t} = P_0 \tau \\
				P_0 & = (10^8 \times \SI{3.4e19}{\J})\times \SI{1.6e-8}{\s} \\
					& = \SI{2.1e-3}{\W}
			\end{align*}
		
			\item From the problem, the angular distribution is \begin{align*}
				I(\theta) & = I_0 \sin[2](\theta)
				\intertext{Integrating this,}
				W & = 2\pi W_0 \int \sin[2](\theta) \dd{\theta} \\
					& = \pi^2 W_0 \\
				W_0 & = W_\mathrm{total} / \pi^2
			\end{align*}
		\end{enumerate}
	
		\item[2.] \begin{enumerate}
			\item For hydrogen, the molar mass is \SI{1}{\g/\mol}. The Doppler width is then \begin{align*}
				\delta \nu_D & = \num{7.16e-7} \left(\SI{2.47e15}{\Hz}\right) \left(300 / 1\right)^{1/2} \\
				& = \SI{30.6}{\GHz}
			\end{align*}
		
			\item The collimation ratio (this is Fraunhofer diffraction, right?) is \begin{align*}
				\epsilon & = \frac{b}{2d} = 1 / 200
			\end{align*}
			From (a), the reduced Doppler width is then \begin{align*}
				\delta_\mathrm{D-beam} & = \delta_D \sin \epsilon = \SI{150}{\MHz}
			\end{align*}
			
			\item For $\tau(2p) = \SI{1.2}{\ns}$, the natural linewidth is \begin{align*}
				\delta \nu_N = \frac{1}{2 \pi \tau} = \SI{132}{\MHz}
			\end{align*}
			which is on the same order of magnitude as (b).
			
			\item Yes, as the hyperfine splitting is just the 21 cm line, \SI{1400}{\MHz}.
		\end{enumerate}
	
		\item[11.] \begin{enumerate}
			\item For the 21 cm line, the Einstein coefficient $A_{ik} = \SI{2.9e-15}{\per\s}$, the natural line width is given by \begin{align*}
				\delta \nu_v & = \frac{A_{ik}}{2 \pi} = \SI{5e-16}{\per\s}
			\end{align*}
			The Doppler width is given by (7.72b), \begin{align*}
				\delta \nu_D & = \num{7.16e-7} \times \nu_0 \sqrt{T/M} \\
					& = \num{7.16e-7} \times \frac{c}{\SI{21}{\centi\meter}} \sqrt{10 / 1} \\
					& = \SI{3234}{\per\s}
			\end{align*}
			The collision broadening is given by \begin{align*}
				\delta \nu_\mathrm{coll} & = \frac{n \sigma}{2 \pi} \sqrt{ \frac{8 kT}{\pi m} }  \\
					& = \SI{7.3e-20}{\per\s}
			\end{align*}
		
			As for the Lyman $\alpha$ line, $A_{ik} = \SI{10e9}{\per\s}$ and the natural line width is \begin{align*}
				\delta \nu_n & = \SI{1.6e8}{\per\s} \\
				\delta \nu_D & = \SI{5.6e9}{\per\s} \\
				\delta \nu_\mathrm{coll} & = \SI{7.3e-13}{\per\s}
			\end{align*}
		
			\item I don't understand this problem at all. At \SI{10}{\K}, the ratio of the populations is \begin{align*}
				\frac{N(F=1)}{N(F=0)} & = 3 ^{-h \nu / kT} \approx 3 \cdot 0.994 \\
				\Delta N & = N(F=0) - \frac{1}{3} N(F=1) \\
					& = 0.006 N(F=0)
				\intertext{The absorption coefficient is neglible as}
				\alpha & = \Delta N \cdot \sigma_\mathrm{abs} \\
					& = \num{5.4e-6}
				\intertext{For the Lyman $\alpha$-line,}
				\alpha & = \sigma N = \num{1e-15} \num{10e6} = \num{10e-9}
			\end{align*}
		
			\item The natural linewidth and Doppler linewidths can be calculated as \begin{align*}
				\delta \nu_N & = \frac{1}{2 \pi \tau} = \frac{1}{2 \pi \times \SI{20}{\ms}} \approx \SI{8}{\Hz} \\
				\delta \nu_D & = \num{7.16e-7} \frac{c}{\lambda} \sqrt{T/M} \\
					& = \SI{274}{\MHz}
			\end{align*}
			The pressure broadening can be calculated with mean velocity \begin{align*}
				\bar{v} & = \sqrt{ \frac{8kT}{\pi m} } = \SI{630}{\m\per\s} \\
				\delta \nu_\mathrm{trans} & = \frac{1}{2 \pi (0.01 / 630)} \approx \SI{100}{\kHz}
			\end{align*}
		\end{enumerate}
		
		\item[12.] The matrix element (for $Z=1$) is \begin{align*}
			M_{ik} & = e \int \psi(2s) \bvec{r} \psi(1s) \dd{\tau} \\
				& = \frac{1}{4 \pi \sqrt{2} a_0^3} \int \left(2 - r/a_0\right) e^{-r/2a_0} \bvec{r} e^{-r/a_0} \dd\tau \\
				& = \left(\dots\right) \int_0^{2 \pi} \cos \phi \dd{\phi} = 0 \qed
		\end{align*}
		
		\vspace{1em}
		\hrule
		
		\item[1.] \begin{enumerate}
			\item As the statistical weights $g_i = 2 J_i  + 1$, then the relative population ratio is \begin{align*}
					\frac{N_i}{N_k} & = 3e^{-h\nu / kT} = \num{6.6e-42}
				\end{align*}
			
			\item The relative absorption of the incident wave is \begin{align*}
				A & = \frac{I_0 - I_t}{I_0} \\
					& = \frac{I_0 - I_0 e^{-\alpha L}}{I_0} \\
					& \approx \alpha L = N_k \sigma_{ki} L
			\end{align*}
		
			\item
		\end{enumerate}
	\end{enumerate}
\end{document}