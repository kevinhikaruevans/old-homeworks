\documentclass{homework}

\title{Homework 11}
\author{Kevin Evans}
\studentid{11571810}
\date{April 15, 2021}
\setclass{Physics}{461}
\usepackage{amssymb}
%\usepackage{mathtools}
\usepackage{graphicx}
\usepackage{amsthm}
\usepackage{amsmath}
\usepackage{slashed}
\usepackage{boldline}
\usepackage{physics}
\usepackage[inter-unit-product =\cdot]{siunitx}

\usepackage[makeroom]{cancel}
\usepackage{booktabs}

\usepackage{times}
\usepackage{mhchem}

%\usepackage{calligra}
%\DeclareMathAlphabet{\mathcalligra}{T1}{calligra}{m}{n}
%\DeclareFontShape{T1}{calligra}{m}{n}{<->s*[2.2]callig15}{}
%\newcommand{\scriptr}{\mathcalligra{r}\,}
%\newcommand{\boldscriptr}{\pmb{\mathcalligra{r}}\,}
%\newcommand{\emf}{\mathcal{E}}

\begin{document}
	\maketitle
	\begin{enumerate}
		\item  The absorption coefficient is given by \begin{align*}
				\alpha(\nu_{12}) & = \left[
					N_1 - (g_1 / g_2) N_2
				\right] \sigma(\nu_{12})
			\end{align*}
			where in the $2 \to 1$ transition, the statistical weights are $5$ and $3$ respectively. As the rotational energy is defined as $E_J = B_e J(J+1)$ where $B_e$ is the rotational constant specific to the molecule, then $\Delta E = hc B_e \left[(J_2 (J_2 + 1) - J_1(J_1 + 1))\right] = $. Additionally, the populations of the two levels are related as \begin{align*}
				\frac{N_2}{N_1} & = \frac{g_2}{g_1} e^{-h \nu / kT} \\
				\implies \frac{N_2}{N_1} - \frac{g_1}{g_2} N_2 & = N_1 - \frac{g_1}{g_2} \left[
					\frac{g_2}{g_1} (1 - \Delta E / kT)
				\right] N_1
			\end{align*}
		
		\item The Fourier transform of $S(t)$ can be written \begin{align*}
			I(\omega) & = \frac{1}{\sqrt{2 \pi}} \int_{-\infty}^\infty S(t) e^{i\omega t} \dd{t}
			\intertext{From the relation $e^{i\omega t} = \cos(\omega t) + i \sin(\omega t)$, we can convert this to a one-sided Fourier transform}
			I(\omega)	& = \frac{2}{\sqrt{2 \pi}} \int_0^\infty S(t) \cos(\omega v t / c) \dd{t}
		\end{align*}
	
		\item For 1200 grooves per mm, the distance between grooves is \begin{align*}
			d & = \SI{833.3}{\nm}
			\intertext{For $m=1$, the angle of each reflection is}
			\beta_1 & = \sin[-1](\frac{\lambda}{d} - \sin \alpha) \\
				& = \SI{0.208209}{\radian} \\
			\beta_2 & = \SI{0.208945}{\radian} \\
			\Delta \beta & = \SI{0.00073598}{\radian} \\
			\implies \Delta s & = \beta f \\
				& = \SI{0.736}{\mm}
		\end{align*}
	
		\item[5.] From Beere's law, the transmitted power is \begin{align*}
			P & = P_0 e^{-\alpha x} \approx P_0 (1 - \alpha x)
			\intertext{Then the absorbed power per length is}
			\Delta P & = \alpha P_0  = 10^{-7} \quad \si{\W\per\centi\meter}
			\intertext{Therefore, the power absorption rate is}
			N & = \frac{\Delta P}{E} = \frac{\Delta P}{h \nu} \\
				& \approx \num{2.5e11} \text{ photons per s}
			\intertext{For the given angle, the detector receives the fraction }
			\frac{ \Delta \Omega }{4 \pi} N = \frac{0.2}{4 \pi} \num{2.5e11} & = \num{4e9} \quad \text{ photons per s}
			\intertext{Then for the given efficiency, the current is just the fraction of photons per second and}
			I & = ( \eta  G) I_0 = \SI{0.13}{\mA}
		\end{align*}
	\end{enumerate}
\end{document}