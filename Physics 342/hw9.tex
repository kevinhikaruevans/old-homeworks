\documentclass{homework}

\title{Homework 9}
\author{Kevin Evans}
\studentid{11571810}
\date{April 14, 2021}
\setclass{Physics}{342}
\usepackage{amssymb}
\usepackage{mathtools}

\usepackage{amsthm}
\usepackage{amsmath}
\usepackage{slashed}
\usepackage{relsize}
\usepackage{threeparttable}
\usepackage{float}
\usepackage{booktabs}
\usepackage{boldline}
\usepackage{changepage}
\usepackage{physics}
\usepackage[inter-unit-product =\cdot]{siunitx}
\usepackage{setspace}

\usepackage[makeroom]{cancel}
%\usepackage{pgfplots}

\usepackage{enumitem}
\usepackage{times}
\usepackage{mhchem}

\usepackage{calligra}
\DeclareMathAlphabet{\mathcalligra}{T1}{calligra}{m}{n}
\DeclareFontShape{T1}{calligra}{m}{n}{<->s*[2.2]callig15}{}
\newcommand{\scriptr}{\mathcalligra{r}\,}
\newcommand{\boldscriptr}{\pmb{\mathcalligra{r}}\,}
\newcommand{\emf}{\mathcal{E}}

\begin{document}
	\maketitle
	\begin{enumerate}
		\item The vector potential can be written \begin{align*}
			\bvec{A} & = -\frac{\mu_0 p_0 \omega}{4 \pi r} \sin(\omega(t - r/c)) \left[
				\cos \theta \uvec{r}
				- \sin \theta \uvec{\theta}
			\right]
			\intertext{To check if the potentials satisfy the Lorenz gauge condition, we can find the divergence of the vector potential,}
			\div{\bvec{A}} & =  -\frac{\mu_0 p_0 \omega}{4 \pi}  \left\{
				\frac{1}{r^2} \pdv{r} \left[
					r \sin(\omega(t - r/c)) \cos \theta
				\right]
				- \frac{\sin(\omega(t - r/c)) }{r^2 \sin \theta} \pdv{\theta} \left[
					\sin[2](\theta)
				\right]
			\right\} \\
				& =  -\frac{\mu_0 p_0 \omega}{4 \pi}  \left\{
					\frac{\cos \theta}{r^2} \left[
						\sin(\omega(t - r/c)) - \frac{ r \omega \cos(\omega (t - r/c)) }{c}
					\right]
					- \frac{2\sin(\omega(t - r/c)) \cos \theta}{r^2 } 
				\right\} \\
				& = -\frac{\mu_0 p_0 \omega}{4 \pi}  \left\{
				\frac{\cos \theta}{r^2} \left[
				- \frac{ r \omega \cos(\omega (t - r/c)) }{c}
				\right]
				- \frac{\sin(\omega(t - r/c)) \cos \theta}{r^2 } 
				\right\}
		\end{align*}
		Next, taking the time derivative of the scalar potential \begin{align*}
			\pdv{V}{t} & = \frac{p_0 \cos \theta \omega}{4 \pi \epsilon_0 r} \left[
				-\frac{\omega}{c} \cos(\dots) 
				+ \frac{1}{r} \sin(\dots)
			\right]
		\end{align*}
		Rearranging, it becomes clear that the potentials obey the Lorenz gauge condition, 
		\[ \div{\bvec{A}} = -\mu_0 \epsilon_0 \pdv{V}{t}. \]
		\item Starting from the power radiated via the Poynting vector and the current flowing in the ``wire'', eq. (11.22) and (11.15), \begin{align*}
			\expval{P} & = \frac{\mu_0 p_0^2 \omega^4}{12 \pi c} = \expval{I^2}_T R \\
			R & = \frac{\expval{P}}{\expval{I^2}_T} \\
				& = \frac{\mu_0 (q_0 d)^2 \omega^4}{12 \pi c} \frac{2}{q_0^2 \omega^2} \\
				& = \frac{\mu_0 d^2 \omega^2}{6 \pi c} \\
				& = \frac{\mu_0 d^2 \pi c}{3 \lambda^2} \propto (d/\lambda)^2
		\end{align*}
		For a \SI{1}{\m} antenna operating at $\lambda = \SI{30}{\centi\m}$, the resistance would be \SI{4.4}{\kohm}. This is quite high for an antenna.
		
		
		\item From (11.40), the average power is \begin{align*}
			\expval{P} & = \frac{\mu_0 m_0^2 \omega^4}{12 \pi c^3} = I^2 R \\
			R & = \frac{\mu_0 \omega^4}{12 \pi c^3} (\pi b^2 I_0)^2 \frac{1}{\expval{I^2}} \\
				& = \frac{\mu_0 \omega^4 b^4}{6 c^3}
		\end{align*}
		...as the time average of $\expval{I^2} = I_0 / 2$.
		\item Starting from (11.33), \begin{align*}
			\bvec{A} & = \frac{\mu_0 m_0}{4 \pi} \frac{\sin \theta}{r} \left\{
				\frac{1}{r} \cos\left[\omega(t - r/c)\right]
				- \frac{\omega}{c} \sin\left[\omega(t - r/c)\right]
			\right\} \uvec{\phi}
		\end{align*}
		Assuming there is no net charge, the scalar potential is zero. The electric field is then \begin{align*}
			\bvec{E} & = -\pdv{\bvec{A}}{t} \\
				& = -\frac{\mu_0 m_0 }{4 \pi} \frac{\sin\theta \omega}{r} \left\{
					-\frac{1}{r} \sin\left[\omega(t - r/c) \right]
					- \frac{\omega}{c} \cos\left[\omega(t - r/c) \right]
				\right\} \uvec{\phi} \\
				& = \frac{\mu_0 m_0 }{4 \pi} \frac{\omega \sin\theta}{r} \left\{
				\frac{1}{r} \sin\left[\omega(t - r/c) \right]
				+ \frac{\omega}{c} \cos\left[\omega(t - r/c) \right]
				\right\} \uvec{\phi} \\
		\end{align*}
		The magnetic field is the curl of the vector potential. The non-zero terms are \begin{align*}
			\bvec{B} & = \curl\bvec{A} \\
				& = \frac{1}{r \sin \theta} \pdv{\theta} \sin \theta A_\theta \uvec{r}
					- \frac{1}{r}
						\pdv{r} r A_\phi
					\uvec{\theta} \\
				& = \frac{\mu_0 m_0}{4 \pi} \left[
					\frac{1}{r \sin \theta} \pdv{\theta} \left(
						\frac{ \sin[2](\theta) }{r} \left\{\dots\right\} 
					\right) \uvec{r} \:
					- \frac{\sin \theta}{r} \pdv{r}
						\left(
							\frac{1}{r} \cos(\dots)
							- \frac{\omega}{c} \sin(\dots)
						\right) \uvec{\theta}
				\right] \\
				& = \frac{\mu_0 m_0}{4 \pi} \left[
					\frac{2 \cos \theta}{r} \left\{ \dots \right\} \uvec{r}
					- \frac{\sin \theta}{r} \left(
						\frac{
							c^2 \cos\left[\omega(t - r/c) \right]
							+ (r^2 \omega^2 - cr \omega) \sin\left[\omega(t - r/c) \right]
					 }{c^2 r^2}
					\right)\uvec{\theta}
				\right]
		\end{align*}
		The Poynting vector is found using \begin{align*}
			\bvec{S} & = \frac{1}{\mu_0} \bvec{E} \cross \bvec{B}
		\end{align*}
		I've given up on this problem.
		
		\item \begin{enumerate}
			\item Not sure if this is valid in this case, but if we reuse the setup from p. 480 and take the power radiating through a giant spherical surface, the total radiated power is \begin{align*}
				P_\mathrm{rad} & \cong \frac{\mu_0}{6 \pi c} \ddot{p}^2
				\intertext{where the capacitor is approximately a dipole,}
				p(t) & = Q_0 d e^{-t/RC} \\
				\ddot{p}(t) & = \frac{ Q_0 d }{(RC)^2} e^{-t/RC}
				\intertext{Integrating to find the total radiated energy,}
				E_\mathrm{rad} & = \int P(t) \dd{t} \\
					& = \frac{\mu_0}{6 \pi c} \left( \frac{Q_0 d}{(RC)^2} \right)^2 \int_0^\infty  e^{-t/RC} \dd{t} \\
					& = \frac{\mu_0}{6 \pi c} \left( \frac{Q_0 d}{(RC)^2} \right)^2 RC \\
					& = \frac{\mu_0 Q_0^2 d^2}{6 \pi c R^3 C^3}
				\intertext{As a fraction of its initial energy,}
				\frac{E_\mathrm{rad}}{E_\mathrm{initial}} & = \frac{\mu_0 d^2}{3 \pi c R^3 C^2}
			\end{align*}
			
			\item Plugging all the values in, the fraction is \begin{align*}
				\frac{E_\mathrm{rad}}{E_\mathrm{initial}} & \approx \num{4.44e-9}
			\end{align*}
			which is pretty tiny.
		\end{enumerate}
	\end{enumerate}


\end{document}