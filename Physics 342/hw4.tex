\documentclass{homework}

\title{Homework 4}
\author{Kevin Evans}
\studentid{11571810}
\date{February 17, 2021}
\setclass{Physics}{342}
\usepackage{amssymb}
\usepackage{mathtools}

\usepackage{amsthm}
\usepackage{amsmath}
\usepackage{slashed}
\usepackage{relsize}
\usepackage{threeparttable}
\usepackage{float}
\usepackage{booktabs}
\usepackage{boldline}
\usepackage{changepage}
\usepackage{physics}
\usepackage[inter-unit-product =\cdot]{siunitx}
\usepackage{setspace}

\usepackage[makeroom]{cancel}
%\usepackage{pgfplots}

\usepackage{enumitem}
\usepackage{times}

\usepackage{calligra}
\DeclareMathAlphabet{\mathcalligra}{T1}{calligra}{m}{n}
\DeclareFontShape{T1}{calligra}{m}{n}{<->s*[2.2]callig15}{}
\newcommand{\scriptr}{\mathcalligra{r}\,}
\newcommand{\boldscriptr}{\pmb{\mathcalligra{r}}\,}
\newcommand{\emf}{\mathcal{E}}

\begin{document}
	\maketitle
	\begin{enumerate}
		\item \begin{enumerate}
			\item The momentum is \begin{align*}
				\bvec{p} & = \mu_0 \epsilon_0 \int_V \bvec{S} \dd{\tau} \\
				& = \epsilon_0 EB \left(\uvec{z} \cross \uvec{x}\right) Ad \\
				& = \epsilon_0 A EB d \uvec{y}
			\end{align*}
			\item From the magnetic force on the wire, \begin{align*}
				F & = Id B \\
				p & = \int_0^t F \dd{t} \\
					& = dB \int_0^t I(t) \dd{t} = dB Q_\mathrm{tot}?
			\end{align*}
		\end{enumerate}
	
		\item \begin{enumerate}
			\item By Gauss's law, \begin{align*}
				\bvec{ E } & = \frac{ Q_\mathrm{enc} }{2 \pi \epsilon_0 s L} \uvec{s}\\
					& = -\frac{\lambda }{2 \pi \epsilon_0 s} \uvec{s}
			\end{align*}
		
			\item The total charge must be zero outside of the cylinder, so \begin{align*}
				Q_\mathrm{enc} & = \sigma A = \lambda L \\
				\sigma & = \frac{ \lambda L }{2 \pi a L} \\
					& = \frac{\lambda }{2 \pi a} 
			\end{align*}
		
			\item Omitting the $z$ axis from the integral, it should give the momentum per unit length as \begin{align*}
				\bvec{p} / L & = \mu_0 \epsilon_0 \iint_A \bvec{S} s \dd{s} \dd{\phi} \\
					& = \mu_0 \epsilon_0 B_\mathrm{ext} \left(-\frac{\lambda}{2 \pi \epsilon_0}\right) \left(\uvec{s} \cross \uvec{z}\right) \int_0^a s^2 \dd{s} \int_0^{2\pi} \dd{\phi} \\
					& = \frac{\mu_0 B_\mathrm{ext} \lambda a^3}{3} \uvec{\phi}
			\end{align*}
		\end{enumerate}
	
		\item For $f(z, t) = A \sin[2](\alpha z+\beta t)$, \begin{align*}
			\pdv[2]{f}{z} & = 2Aa^2 \cos(2(\alpha z + \beta t)) \\
			\pdv[2]{f}{t} & = 2A\beta^2 \cos(2(\alpha z + \beta t))
			\intertext{Applying the wave equation and removing terms on both sides,}
			\alpha^2 & = \frac{1}{v^2} \beta^2 \\
			v & = \beta/\alpha
		\end{align*}
	
		\item \begin{enumerate}
			\item $-1$
			\item $i$
			\item $\frac{1}{\sqrt{2}}(1 + i)$
		\end{enumerate}
	
		\item \underline{Proposition}: $\sin(u+v) = \sin u \cos v + \cos u \sin v$ and $\cos(u+v) = \cos u \cos v - \sin u \sin v$.
			\begin{proof}
				By Euler's formula, \begin{align*}
					e^{i(u+v)} & = e^{iu}e^{iv} = \cos(u+v) + i \sin(u+v) \\
						& = \left(\cos u + i \sin u\right)\left(\cos v + i \sin v \right) \\
						& = \cos u \cos v + i \cos u \sin v + i \sin u \cos v - \sin u \sin v
					\intertext{Separating the real and imaginary parts,}
					\sin(u+v) & = \cos u \sin v + \sin u \cos v \\
					\cos(u+v) & = \cos u \cos v - \sin u \sin v					
				\end{align*}
			\end{proof}
	\end{enumerate}
\end{document}