\documentclass{homework}

\title{Homework 5}
\author{Kevin Evans}
\studentid{11571810}
\date{September 30, 2020}
\setclass{Physics}{341}
\usepackage{amssymb}
\usepackage{mathtools}

\usepackage{amsthm}
\usepackage{amsmath}
\usepackage{slashed}
\usepackage{relsize}
\usepackage{threeparttable}
\usepackage{float}
\usepackage{booktabs}
\usepackage{boldline}
\usepackage{changepage}
\usepackage{physics}
\usepackage[inter-unit-product =\cdot]{siunitx}
\usepackage{setspace}

\usepackage[makeroom]{cancel}
%\usepackage{pgfplots}

\usepackage{enumitem}
\usepackage{times}

\usepackage{calligra}
\DeclareMathAlphabet{\mathcalligra}{T1}{calligra}{m}{n}
\DeclareFontShape{T1}{calligra}{m}{n}{<->s*[2.2]callig15}{}
\newcommand{\scriptr}{\mathcalligra{r}\,}
\newcommand{\boldscriptr}{\pmb{\mathcalligra{r}}\,}

\begin{document}
	\maketitle
	\begin{enumerate}
		\item \begin{enumerate}
			\item Integrating over the sphere's volume, \begin{align*}
				W & = \frac{1}{2} \int_v \rho V \dd{\tau} \\
					& = \frac{1}{2} \frac{3q}{4 \pi R^3} \int_v \frac{q}{4 \pi \epsilon_0}
					\frac{1}{2 R} \left(3 - \frac{r^2}{R^2}\right) \dd{\tau} \\
					& = \frac{3q^2}{64 \pi^2 \epsilon_0 R^4} \left(4 \pi \right) \int_0^R \left(3 - \frac{r^2}{R^2}\right) \left(r^2 \dd{r}\right) \\
					& = \frac{12q^2}{64 \pi \epsilon_0 R^4} \left[\frac{3r^3}{3} - \frac{r^5}{5R^2}\right]_0^R \\
					& = \frac{3q^2}{20 \pi \epsilon_0 R}
			\end{align*}
		
			\item Integrating over all space, \begin{align*}
				W & = \frac{\epsilon_0}{2} \int_{\mathbb{R}^3} E^2 \dd{\tau} \\
					& = \frac{4 \pi \epsilon_0}{2} \left[\int_0^R {E_\mathrm{in}}^2 \; r^2\dd{r}
					+ \int_R^\infty {E_\mathrm{out}}^2 \; r^2 \dd{r}
					 \right] \\
					& = 2\pi\epsilon_0 \left[\int_0^R \left(\frac{qr}{4 \pi \epsilon_0 R^3} \right)^2 r^2 \dd{r} + \int_{R}^{\infty} \left(\frac{q}{4 \pi \epsilon_0 r^2}\right)^2 r^2 \dd{r} \right] \\
					& = \frac{q^2}{8 \pi \epsilon_0} \left[ \frac{1}{R^6} \int_0^R r^4 \dd{r} + \int_R^\infty r^{-2} \dd{r} \right] \\
					& = \frac{q^2}{8 \pi \epsilon_0} \left[ \frac{R^5}{5R^6} + \left(0 + \frac{1}{R} \right) \right] = \frac{6q^2}{40 \pi \epsilon_0 R} \\
					& = \frac{3q^2}{20 \pi \epsilon_0 R}
			\end{align*}
		
			\item From (b), we can reuse much of the calculation, but instead use $a$ as the upper bound of the $E_\mathrm{out}$ integral, leading to \begin{align*}
				W & = \frac{\epsilon_0}{2} \left( \int_v E^2 \dd{\tau} + \oint_S V \bvec{E} \cdot \dd{\bvec{a}} \right) 
					 = \frac{q^2}{8 \pi \epsilon_0} \left[ \frac{R^5}{5R^6} + \left(\frac{1}{a} + \frac{1}{R}\right)\right] + \frac{\epsilon_0}{2} \oint_S V\bvec{E} \cdot \dd{\bvec{a}}
				\intertext{And since $\dd{\bvec{a}}$ does not include a $\dd{r}$ term, the area integral becomes the surface area of a sphere with radius $a$,} 
				W & = \frac{q^2}{8 \pi \epsilon_0} \left[ \frac{R^5}{5R^6} + \left(\frac{1}{a} + \frac{1}{R}\right)\right] + \frac{\epsilon_0}{2} \left(\propto \frac{1}{a}\right) \left( \propto \frac{1}{a^2}\right) \left(4 \pi a^2\right)
				\intertext{At the limit $a \to \infty$, $1/a \to 0$ and we're left with the expected expression}
				& =  \frac{q^2}{8 \pi \epsilon_0} \left[ \frac{R^5}{5R^6} + \frac{1}{R}\right] = \frac{3q^2}{20 \pi \epsilon_0 R}
			\end{align*}
		\end{enumerate}
	
		\item \begin{enumerate}
			\item If we take a Gaussian surface, the only region with a non-zero enclosed charge is between the two spheres. By Gauss's law, the electric field between the spheres is \begin{align*}
				\bvec{E} & = \frac{q}{4 \pi r^2} \uvec{r}
				\intertext{The energy of the configuration is then}
				W & = \frac{\epsilon_0}{2} \int_{\mathbb{R}^3} E^2 \dd{\tau} = \frac{4 \pi \epsilon_0}{2} \int_a^b \left(\frac{q}{4 \pi \epsilon_0 r^2}\right)^2 r^2 \dd{r} \\
					& = \frac{q^2}{8 \pi \epsilon_0} \int_a^b \frac{1}{r^2} \dd{r} \\
					& = \frac{q^2}{8 \pi \epsilon_0} \left(a^{-1} - b^{-1}\right)
			\end{align*}
		
			\item Using the alternative formula, the electric field of each individual charge distribution is only non-zero outside of its radius, \begin{align*}
				\bvec{E}_1 & = \frac{q}{4 \pi \epsilon_0 r^2} \uvec{r} \\
				W_1 & = \frac{\epsilon_0}{2} \int_{\mathbb{R}^3} {E_1}^2 \dd{\tau} \\
					& = \frac{4 \pi \epsilon_0}{2} \int_a^\infty \left( \frac{q}{4 \pi \epsilon_0 r^2} \right)^2 r^2 \dd{r} \\
					& = \frac{q^2}{8 \pi \epsilon_0} a^{-1}
				\intertext{Similarly for the outer charge,}
				W_2 & = \frac{q^2}{8 \pi \epsilon_0} b^{-1}
			\intertext{For the dot product, it is only non-zero after $b$,}
			\epsilon_0 \int_v \bvec{E}_1 \cdot \bvec{E}_2 \dd{\tau} & = 4 \pi \epsilon_0 \int_b^\infty \left(\frac{q}{4 \pi \epsilon_0 r^2}\right) \left(-\frac{q}{4 \pi \epsilon_0 r^2}\right) r^2 \dd{r} \\
				& = -\frac{q^2}{4 \pi \epsilon_0} \left(b^{-1}\right)
			\intertext{Summing these work terms, it equals the work found in (a),}
				W & = \frac{q^2}{8 \pi \epsilon_0} a^{-1} +  \frac{q^2}{8 \pi \epsilon_0} b^{-1}  -\frac{q^2}{4 \pi \epsilon_0} b^{-1} \\
					& = \frac{q^2}{8 \pi \epsilon_0} \left(a^{-1} - b^{-1}\right)
			\end{align*}
		\end{enumerate}
	
		\item \begin{enumerate}
			\item Since the inner sphere is a conductor, $q$ will be uniformly distributed on the surface, \begin{align*}
				\sigma_R & = \frac{q}{4 \pi R^2} \\
				\intertext{Similarly on the outer shell, the inner sphere will induce a uniform surface charge,}
				\sigma_a & = \frac{-q}{4 \pi a^2} \\
				\sigma_b & = \frac{q}{4 \pi b^2}
			\end{align*}
		
			\item The potential can be found by splitting the integrals: beyond the outer shell and between the two spheres, \begin{align*}
				V & = -\int_o^r \bvec{E} \cdot \dd{\vec{\ell}} \\
					& = - \frac{q}{4 \pi \epsilon_0} \int_\infty^b r^2 \dd{r} - \frac{q}{4 \pi \epsilon_0} \int_a^R r^2 \dd{r} \\
					& = \frac{q}{4 \pi \epsilon_0} \left( b + R - a\right)
			\end{align*}
		
			\item For (a), Only $\sigma_b$ is changed, as it would equal zero, since the charges on $\sigma_a$ would ``come from'' ground, instead of coming from the zero net charge.
			
			For (b), the outer sphere and reference point is now an equipotential, so the potential difference in the first term of $V$ in (b) would equal zero, i.e. 
			\[V = \frac{q}{4 \pi \epsilon_0} \left(R - a\right)\]
		\end{enumerate}
	
		\item Using (2.48), $\bvec{E} = \frac{\sigma}{\epsilon_0} \uvec{n}$ and since $\sigma = Q / A$, the pressure \[ P = \frac{\epsilon_0}{2} E^2 = \frac{Q^2}{2A^2 \epsilon_0} \]
		
		\item From Gauss's law, the electric field between the two cylinders (of arbitrary length $l$) is \begin{align*}
			\bvec{E} & = \frac{Q}{2 \pi \epsilon_0 s l } \uvec{s}
			\intertext{The potential difference between the two tubes and capacitance per length is found as}
			V & = -\int_b^a \bvec{E} \cdot \dd{\vec{\ell}} = - \frac{Q}{2 \pi \epsilon_0 l}\int_b^a s^{-1} \dd{s} \\
				& = \frac{Q}{2 \pi \epsilon_0 l} \ln(\frac{b}{a}) \\
			C/l & = \frac{Q}{2 \pi \epsilon_0} \ln(\frac{b}{a})
		\end{align*}
	\end{enumerate}
\end{document}