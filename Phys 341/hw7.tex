\documentclass{homework}

\title{Homework 7}
\author{Kevin Evans}
\studentid{11571810}
\date{October 21, 2020}
\setclass{Physics}{341}
\usepackage{amssymb}
\usepackage{mathtools}

\usepackage{amsthm}
\usepackage{amsmath}
\usepackage{slashed}
\usepackage{relsize}
\usepackage{threeparttable}
\usepackage{float}
\usepackage{booktabs}
\usepackage{boldline}
\usepackage{changepage}
\usepackage{physics}
\usepackage[inter-unit-product =\cdot]{siunitx}
\usepackage{setspace}

\usepackage[makeroom]{cancel}
%\usepackage{pgfplots}

\usepackage{enumitem}
\usepackage{times}

\usepackage{calligra}
\DeclareMathAlphabet{\mathcalligra}{T1}{calligra}{m}{n}
\DeclareFontShape{T1}{calligra}{m}{n}{<->s*[2.2]callig15}{}
\newcommand{\scriptr}{\mathcalligra{r}\,}
\newcommand{\boldscriptr}{\pmb{\mathcalligra{r}}\,}

\begin{document}
	\maketitle
	\begin{enumerate}
		\item On the inside of the sphere, we can assume the potential to have form (as $B=0$ to prevent $V \to \infty$). Outside the sphere $A = 0$, allowing the voltage to tend toward zero at infinity. \begin{align*}
			V(r, \theta) & = \begin{cases*}
				\displaystyle\sum_l A_l r^l P_l (\cos \theta) & inside \\
				\displaystyle\sum_l \frac{B_l}{r^{l+1}} P_l (\cos \theta) & outside
			\end{cases*}
			\intertext{The given voltage can be written in terms of Legendre polynomials for $l=0$ and $l=2$,}			
			V_0(\theta) & = k \cos 2 \theta \\
				& = \frac{4k}{3} \left(\frac{3 \cos[2](\theta) - 1}{2} - \frac{1}{4}\right) \\
				& = \frac{4k}{3} \left(P_2(\cos \theta) - \frac{1}{4} P_0(\cos \theta)\right)
			\intertext{Inside the sphere, the coefficients can be found as}
			A_l & = \frac{2l + 1}{2 R^l} \int_0^\pi V_0(\theta) \sin \theta \dd{\theta} \\
			A_0 & = \frac{-k}{6} \int_0^\pi \sin(\theta) \dd{\theta} = -\frac{2k}{6} \\
			A_2 & = \frac{5}{2R^2} \int_0^\pi \frac{4k}{3} \left[P_2(\cos(\theta))\right]^2 \sin(\theta) \dd{\theta} = \frac{4k}{3R^2} \\
			\intertext{Plugging in these $A_l$ values and expanding the sum, inside the sphere, the potential is}
			\Aboxed{ V_\mathrm{in}(r, \theta) & = -\frac{2k}{6} + \frac{4kr^2}{3R^2} \left(\frac{3 \cos[2](\theta) - 1}{2}\right) }
			\intertext{Outside the sphere, we can apply the relation (3.75) from Griffith's, $B_l = -A_l R^{2l + 1}$}
			B_0 & = \frac{2Rk}{6} \\
			B_2 & = \frac{-4R^3 k}{3} \\
			\Aboxed{ V_\mathrm{out}(r, \theta) & = \frac{2Rk}{6r} - \frac{4R^3k}{3r^3} \left(\frac{3 \cos[2](\theta) - 1}{2}\right) }
		\end{align*}
		Since the normal derivatives are discontinuous at $r=R$ by the surface charge, i.e. 
		\begin{align*}
			\eval{\pdv{V_\mathrm{out}}{r} - \pdv{V_\mathrm{in}}{r}}_{r=R} & = -\frac{\sigma(\theta)}{\epsilon_0} \\
			\left[\frac{4R^3 k}{r^4} \left(\frac{3 \cos[2](\theta) - 1}{2}\right) - \frac{8kr}{3R^2} \left(\frac{3 \cos[2](\theta) - 1}{2}\right)\right]_{r=R} & = -\frac{\sigma(\theta)}{\epsilon_0} \\
			\sigma(\theta) & = \epsilon_0 \left( \frac{8k}{3R} - \frac{4k}{R} \right)\left(\frac{3 \cos[2](\theta) - 1}{2}\right) \\
			\Aboxed{ \sigma(\theta)	& = -\frac{4k\epsilon_0}{3R} \left(\frac{3 \cos[2](\theta) - 1}{2}\right) }
		\end{align*}	
	
		\item Following the steps on page 148 of Griffith's, \begin{align*}
			A_l & = \frac{1}{2 \epsilon_0 R^{l - 1}} \left[
				\int_0^{\pi / 2} \sigma_0 P_l(\cos \theta) \sin(\theta) \dd{\theta}
				+ \int_{\pi / 2}^{\pi} \left(- \sigma_0 \right)P_l(\cos \theta) \sin(\theta) \dd{\theta}
			\right]
			\intertext{The first 6 coefficients can be found as }
			A_0 & = \frac{\sigma_0}{2 \epsilon_0 R^{-1}} \left[
				\int_0^{\pi / 2} \sin(\theta) \dd{\theta}
				-\int_{\pi / 2}^{\pi} \sin(\theta) \dd{\theta}
			\right] = 0 \\
			A_1 & =  \frac{\sigma_0}{2 \epsilon_0} \left[
			\int_0^{\pi / 2} \cos(\theta) \sin(\theta) \dd{\theta}
			-\int_{\pi / 2}^{\pi} \cos(\theta)\sin(\theta) \dd{\theta}
			\right] = \frac{\sigma_0}{2 \epsilon_0} \\
			A_2 & = 0 \\
			A_3 & \approx -10.84 \sigma_0 / \epsilon_0 && \text{...used a calculator for these}  \\
			A_4 & = 0\\
			A_5 & \approx -188.89 \sigma_0 / \epsilon_0
		\end{align*}
	
		\item For the charge density \begin{align*}
			\rho(r, \theta) & = k \left(\frac{R^2}{r^2} - 3\right) \cos \theta
			\intertext{The monopole potential is }
			V_\mathrm{mono} (\bvec{r}) & = \frac{1}{4 \pi \epsilon_0} \frac{k}{z} \int_{V} \left(\frac{R^2}{r^2} - 3\right) \cos \theta \dd{\tau} \\
				& = \frac{1}{4 \pi \epsilon_0} \frac{k}{z} \left(\dots\right) \int_0^\pi \cos(\theta) \sin(\theta) \dd{\theta} \\
				& = 0 \quad \text{as the $\cos\theta \sin\theta$ integral evaluates to zero}
		\end{align*}
		For the dipole potential, \begin{align*}
			V_\mathrm{dipole} (\bvec{r}) & = \frac{1}{4 \pi \epsilon_0 r^2} \left[ \int_V r' \cos \theta \rho{(\bvec{r}')} \dd{\tau} \right] \\
				& = \frac{k}{4\pi \epsilon_0 r^2} \left[
					\int_0^{2\pi} \dd{\phi}
					\int_0^{\pi} \cos^2\!\theta \sin{\theta} \dd{\theta}
					\int_0^R \left( \frac{R^2}{r^2} - 3\right) r^2 \dd{r}
				\right] \\
				& = \frac{k}{2 \epsilon_0 r^2}  \left(\frac{2}{3}\right) \left(0\right) \\
				& = 0
		\end{align*}
		For the quadrupole term, \begin{align*}
			V_\mathrm{quad}(\bvec{r}) & = \frac{1}{4 \pi \epsilon_0 r^3} \int_V (r')^2 \left(\frac{3}{2} \cos[2](\alpha) - \frac{1}{2}\right) \rho(\bvec{r}') \dd{\tau}'
			\intertext{As the observation points are along the $z$-axis, the angle $\alpha = \theta$,}
			V_\mathrm{quad}(\bvec{r}) & = \frac{k}{4 \pi \epsilon_0 r^3} \int_V (r')^2 	\left(\frac{3}{2} \cos[2](\theta) - \frac{1}{2}\right)
				\left(\frac{R^2}{r'^2} - 3\right) \cos \theta
			\dd{\tau}' \\
				& = \frac{k}{4 \pi \epsilon_0 r^3} \left[
					\int_0^{2 \pi} \dd{\phi}
					\int_0^\pi 
					\int_0^R
						\left(\frac{3}{2} \cos[2] (\theta) - \frac{1}{2}\right)
						\left(\frac{R^2}{r'^2} - 3 \right) \cos(\theta) \sin(\theta) (r')^4 \dd{r} \dd{\theta}
				\right] \\
				& = 0 \quad \text{Due to the $\theta$ integral evaluating to zero}
		\end{align*}
	
		\item The monopole term is found by summing all the charges near the origin, \begin{align*}
			V_\mathrm{mono} & = \frac{q}{4 \pi \epsilon_0 r}
		\end{align*}
		
		The dipole term can be found by first summing the dipole moments, \begin{align*}
			\bvec{p} & = \sum_i q_i r_i' \\
				& = q \left( (2 - 1)a\uvec{z} + 0 \uvec{x} \right) = qa \uvec{z} \\
				& = qa \uvec{r} \\
			V_\mathrm{dipole} & = \frac{qa}{4 \pi \epsilon_0 r^2}
		\end{align*}
		
		\item The monopole term is zero as the total net charge is zero.
		
		For the dipole term, \begin{align*}
			\bvec{p} & = qa \left( -2\uvec{z} + (1 - 1) \uvec{x} \right) \\
				& = -2qa \uvec{z} \\
			V_\mathrm{dipole} & = -\frac{2qa}{4 \pi \epsilon_0 r^2}
		\end{align*}
	\end{enumerate}
\end{document}