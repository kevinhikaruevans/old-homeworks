\documentclass{homework}

\title{Homework 1}
\author{Kevin Evans}
\studentid{11571810}
\date{January 24, 2021}
\setclass{Physics}{461}
\usepackage{amssymb}
\usepackage{mathtools}

\usepackage{amsthm}
\usepackage{amsmath}
\usepackage{slashed}
\usepackage{relsize}
\usepackage{threeparttable}
\usepackage{float}
\usepackage{booktabs}
\usepackage{boldline}
\usepackage{changepage}
\usepackage{physics}
\usepackage[inter-unit-product =\cdot]{siunitx}
\usepackage{setspace}

\usepackage[makeroom]{cancel}
%\usepackage{pgfplots}

\usepackage{enumitem}
\usepackage{times}

\usepackage{amsthm}
\newtheorem*{ident}{Identity}

\renewcommand\qedsymbol{$\blacksquare$}

\begin{document}
	\maketitle
	\begin{enumerate}
		\item \begin{enumerate}
			\item If they're both real, there's no phase shift and it's linearly polarized.
			\item As long as they're equal, it's still linearly polarized.
			\item There needs to be a phase shift of some multiple of $\pi/2$ to be circularly polarized.
		\end{enumerate}
	
		\item \begin{enumerate}
			\item From the basis $$\uvec{e}_\pm = \frac{1}{\sqrt{2}} \left(\uvec{e}_x \pm i \uvec{e}_y\right)$$
			\begin{ident}
				$\uvec{e}^*_\pm \cdot \uvec{e}_\pm = 1$
			\end{ident}
			\begin{proof}
				For $\uvec{e}_+$ and ignoring the normalization constant, $\uvec{e}_+^* = \evec{x} - i \evec{y}$. Multiplying this out, \begin{align*}
					\left(\evec{x} - i \evec{y}\right) \left( \evec{x} + i\evec{y}\right) & = \evec{x}^2 - i^2 \evec{y}^2 = 2
				\end{align*}
				For $\evec{-}$, it's the same thing but the order of the products are flipped. Since multiplication is communicative, it must be equal.
			\end{proof}
			%%%%%%%%%%%%%%%%%%%	
			\vspace{1em}
			\begin{ident}
				$\uvec{e}^*_\pm \cdot \uvec{e}_\mp = 0$
			\end{ident}
			\begin{proof}
				Conjurgating $\evec{-}$, it results in $\evec{+}$, so we're left with \begin{align*}
					\left(\evec{x} + i\evec{y}\right)\left(\evec{x} + i\evec{y}\right) & = \evec{x}^2 + i^2 \evec{y}^2 = 0
				\end{align*}
				And it's the same thing if we had conjurgated $\evec{+}$ instead, as the product of the two negatives would be a positive.
			\end{proof}
			%%%%%%%%%%%%%%%%%%%	
			\vspace{1em}
			\begin{ident}
				$\evec{\pm}^* \cross \evec{\pm} = \pm i \uvec{z}$
			\end{ident}
			\begin{proof}
				For $\evec{+}$ and ignoring the normalization constant again, \begin{align*}
					\left(\evec{x} - i\evec{y}\right) \cross \left(\evec{x} + i \evec{y}\right) & = \left(\evec{x} \cross i\evec{y}\right) + \left(-\evec{y} \cross \evec{x}\right) \\
						& = 2 \left(\evec{x} \cross i \evec{y}\right) \\
						& = i \uvec{z} && \text{Renormalizing}
				\end{align*}
			\end{proof}
			%%%%%%%%%%%%
			\vspace{1em}
			\begin{ident}
				$i \uvec{z} \cross \evec{\pm} = \pm \evec{\pm}$
			\end{ident}
			\begin{proof}
				For either $\evec{\pm}$ vector, \begin{align*}
					i \uvec{z} \cross \left(\evec{x} \pm i \evec{y}\right) & = \left(i \uvec{z} \cross \evec{x}\right) \pm \left(i \uvec{z} \cross i \evec{y}\right) \\
						& = \left(i \evec{y}\right) \mp \left(-\evec{x}\right) \\
						& = \pm \left( \evec{x} \pm  i \evec{y}\right) = \pm \evec{\pm}
				\end{align*}
			\end{proof}
		
			\item Since $E_\pm = \bvec{E}^* \cdot \evec{\pm}$ (maybe?), \begin{align*}
				E_+ & = \overbrace{ \left(E_x \evec{x} - E_y \evec{y}\right) }^{\bvec{E}^*} \cdot \evec{+} \\
					& = E_x \left(\evec{x} \cdot \evec{+}\right) - E_y \left(\evec{y} \cdot \evec{+}\right) \\
					& = \frac{1}{\sqrt{2}} \left( E_x + i E_y \right)\\
				E_- & = E_x \left(\evec{x} \cdot \evec{-}\right) - E_y \left(\evec{y} \cdot \evec{-}\right) \\
					& = \frac{1}{\sqrt{2}} \left( E_x - i E_y\right)
			\end{align*}
		\end{enumerate}
	
		\item The product of the two real functions gives \begin{align*}
			A(t)B(t) & = \Re{Ae^{-i \omega t}} \Re{B e^{-i \omega t}} \\
				& = \frac{1}{4} \left(Ae^{-i \omega t} + A^*e^{i \omega t}\right)
					 \left(Be^{-i \omega t} + B^*e^{i \omega t}\right) \\
				& = \frac{1}{4} \left[AB e^{-2 i \omega t} + A^* B^* e^{2 i \omega t} + AB^* + A^*B\right]
			\intertext{Taking the time average, the oscillatory component goes to zero, leaving a real-valued thing}
			\expval{A(t)B(t)}	& = \frac{1}{4} \left( AB^* + A^*B\right) \\
				& = \frac{1}{2} \Re{A^*B}
		\end{align*}\qed
	\end{enumerate}

\end{document}